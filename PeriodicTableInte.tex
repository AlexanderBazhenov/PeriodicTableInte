\documentclass[a5paper,openany]{book}
%\documentclass[11pt]{article}



\input{BookSty.tex}



%%%%%%%%%%%%%%%%%%%%%%%%%%%%%%%%%%%%%%%%%%%%%%%%%%%%%%%%%%%%%%%%%%%%%%%%%%%%%%%%%%%%%%%%

\begin{document}


\begin{center}
	\hfill \break
	\Large{Министерство образования науки Российской Федерации}\\
	\hfill \break
	\Large{	САНКТ-ПЕТЕРБУРГСКИЙ \\
		ПОЛИТЕХНИЧЕСКИЙ УНИВЕРСИТЕТ ПЕТРА ВЕЛИКОГО}\\ 
	%\hfill \break
	%\hfill \break
	%	\large{Институт прикладной математики и механики}\\
	%\hfill \break		
	%\hfill \break
	%	\large{Кафедра «Прикладная математика»}\\
	%\hfill \break
	%\hfill \break
	\hfill \break
	\hfill \break
	
	
	
	
	
	\Large{\it А.Н.Баженов\\
		\hfill \break		\hfill \break}
	{\huge 	Интервальная таблица Менделеева и изотопы на Земле.}\\
	\hfill \break \hfill \break
	\Large{	Учебное пособие	
	}\\
	\hfill \break \hfill \break
\end{center}

\hfill \break
%\hfill \break
%\hfill \break
\begin{center}\Large{Санкт-Петербург \\
		\hfill \break
		2022} \end{center}
\thispagestyle{empty} % выключаем отображение номера для этой страницы



\newpage
УДК 519.9
Р32

%Р е ц е н з е н т ы:\\
%Доктор физико-математических наук, профессор НГУ С.П.Шарый
А в т о р :\\
А.Н.Баженов.
Интервальная таблица Менделеева и изотопы на Земле. – СПб., 2022. – 83 с.
\hfill \break

Учебное пособие соответствует образовательному стандарту высшего
образования Федерального государственного автономного образовательного учреждения высшего образования «Санкт-Петербургский политехнический университет Петра Великого» по направлению подготовки бакалавров 01.03.02 <<Прикладная математика и информатика>>, по дисциплине «Интервальный анализ» и по направлению подготовки  магистров 01.04.02 <<Прикладная математика и информатика>> по дисциплине «Анализ данных с интервальной неопределённостью».


В издании рассмотрена современная версия таблицы Менделеева с интервальными значениями атомных весов, разработанная Международным союзом теоретической и прикладной химии IUPAC. 
Также приводятся сведения об изотопных распределениях в живой и неживой природе на Земле.
Рассмотрены понятия, методы и инструменты анализа данных с интервальной неопределённостью применительно к тематике изотопов.


Пособие адресовано всем, кто интересуется современным естествознанием в различных областях и применением математики к решению практических задач.


Материал апробирован в учебных курсах для студентов 
СПбПУ Петра Великого и аспирантов первого года обучения ФТИ им. А.Ф.Иоффе РАН.
\hfill \break
\hfill \break
\hfill \break
\copyright 
Санкт-Петербургский политехнический университет Петра Великого, 2022

%	\maketitle 

% УДК  518+519.4+519.6, 658.562.012+519.254  
%\\
%\textbf{\textit{Ключевые слова: }} измерение, неопределённость, интервальный анализ.
%\end{abstract}

%%%%%%%%%%%%%%%%%%%%%%%%%%%%%%%%%%%%%%%%%%%%%%%%%%%%%%%%%%%%%%%%%%%%%%%%%%%%%%%%%%%%%%%%

\tableofcontents      %  Содержание  

%%%%%%%%%%%%%%%%%%%%%%%%%%%%%%%%%%%%%%%%%%%%%%%%%%%%%%%%%%%%%%%%%%%%%%%%%%%%%%%%%%%%%%%%
\listoffigures
\listoftables

\chapter{Введение}
\addcontentsline{toc}{chapter}{Введение}   

\bigskip
%В издании приводятся примеры применения интервальных величин в науке и технике. 

В пособии дается информация о современной версии таблицы Менделеева с интервальными значениями атомных весов, разработанная Международным союзом теоретической 
и прикладной химии IUPAC. Также приводятся необходимые сведения из ядерной физики и о синтезе ядер химических элементов. Приведены примеры изотопных распределений в живой и неживой природе на Земле.

Издание задумано как дополнение к курсам лекций автора по интервальному анализу для студентов кафедры <<Прикладная математика>> Института Прикладной Математики и Механики СПбПУ и аспирантов первого года обучения ФТИ им. А.Ф.Иоффе РАН. В значительной степени тематика этого курса представлена в учебном пособии автора <<Интервальный анализ. Основы теории и учебные примеры>> \cite{Bazhenov2020} и <<Обобщение мер совместности для анализа данных с интервальной неопределённостью>> \cite{Bazhenov2022}.

Для углубленного знакомства с интервальным анализом и интервальной статистикой рекомендуются фундаментальная монография С.П.\,Шарого <<Конечномерный интервальный анализ>> \cite{SharyBook} и коллективный труд А.Н.\,Баженова, С.И.\,Жилина, С.И.\,Кумкова и С.П.\,Шарого <<Обработка и анализ данных с интервальной неопределённостью>> \cite{BookIntStat}.

\section{Историческая справка.}

Истории создания Периодической таблицы элементов посвящена обширная библиография \cite{Trifonov1974, Scerri2019}.
В середине 19 века многие исследователи пытались найти закономерности химических свойств элементов. Наиболее удачной оказалась система Д.И.Менделеева. Статья <<Соотношение свойств с атомным весом элементов>> была оубликова на в 1869 г. \cite{Mendeleev1869ru}. В то же время таблица была аннотирована в Германии \cite{Mendeleev1869} и стала достоянием научной общественности.

\begin{figure}[ht] 
	\centering\small
	\unitlength=1mm
	\includegraphics[width=70mm]{Figures/1869Mendeleev.png} 
	\caption{Периодическая таблица 1869 \cite{Mendeleev1869}} 
	\label{f:1869Mendeleev}
\end{figure}	

\begin{figure}[h] 
	\centering\small
	\unitlength=1mm
	\includegraphics[width=80mm]{Figures/140538_PerTab3.jpg} 
	\caption{Таблица Менделеева, XIX век \cite{MendeleevTable} }
	\label{f:MendeleevTable}
\end{figure}
% https://www.chemistryworld.com/opinion/is-this-the-worlds-oldest-classroom-periodic-table/3009960.article

%T.B. Coplen and N. E. Holden Atomic Weights—No Longer Constants of Nature, 2011 \cite{IUPACIntervalHistory}.

\paragraph{Развитие учения о периодичности.}
Приведём некоторые исторические вехи развития \cite{Trifonov1974, Scerri2019, Bekman} \ldots Переход от исследования веществ и эмпирического исследования атомов элементов к исследованию строения атомов и элементарных частиц

{\small	
	%{\tiny 
		\begin{tabular}{l|l|l}
			Год & Автор	& Проблема \\
			\hline
			1885 & Ридберг & {\color{red}Атомные веса не могут рассматриваться} \\
			~ & ~ & {\color{red} в качестве независимой переменной} \\	
			~ & Балмер & Формула для спектральных линий водорода \\	
			1886 & Крукс & {\color{red}Атомные веса не одинаковы для всех атомов } \\
			1888 & ~ &   {\color{red}элемента, а существует распределение }\\
			1890 & Ридберг & Обобщение формулы Бальмера на разные элементы \\	
			1895 & Рентген & Открытие X-лучей \\
			~ & Стрэтт, Рамзай & Аргон --- новая составляющая часть атмосферы \\
			1896 & Беккерель & Радиоактивность урановых соединений  \\
			~ & ~ &  и металлического урана\\
			1897 & Томсон & Катодные лучи --- носитель отрицательного заряда  \\
			~ & ~ &  для всех веществ, в 1800 раз легче водорода\\	
			~ & Ридберг & {\color{red}Атомный вес элементов $M = N+D$,} \\	
			~ & ~ &  {\color{red}$N$ --- целое, $D$ --- малая периодическая функция}\\			
			\hline
		\end{tabular}
	}
	
	\paragraph{Исследования атома и ядра.} По мере осознания закономерностей строения атома и накопления эмпирического материала \ldots
	
	
	
	
	
	
	
	
	В начале XX века произошло много открытий, относящихся к физике ядра. Приведём самые главные, имеющие отношение к элементам и изотопам \cite{Bekman}.
	
	\begin{table}[h]
		{\small	
			%{\tiny 
				\begin{tabular}{l|l|l}
					Год & Автор	& Открытие \\
					\hline
					1913 & Дж.Дж.Томсон & Открытие {\color{red}изотопов} неона с массой 20 и 22\\
					~ & А.Ван-ден-Брук & {\color{red}Порядковый номер элемента в Периодической}  \\
					~ & ~ &  {\color{red}системе равен заряду ядра его 	атомов} \\
					~ & Ф.Содди & Понятие изотопа у радиоактивных
					элементов   \\
					1914, & Н.Бор & Свойства элементов периодической системы  \\
					1921 & ~ & суть  функции зарядов ядер их атомов \\
					1914 & Г.Мозли & Зависимость  частоты характеристического	  \\
					~ & ~ & излучения от порядкового номера элементов \\
					1916 & У.Д.Харкинс & {\color{red}Правило большей распространенности элементов}\\
					~ & ~ & {\color{red}с четными порядковыми номерами} \\
					1917 & Ф.Содди  & Изотопы высшего порядка --- ядерные изомеры\\
					1918 & Дж.Дж.Томсон  &  Доказательство существования изотопов среди\\
					~ & ~ & продуктов радиоактивного распада \\
					1918 & А.У.Стюарт  &  Открытие {\color{red}изобар}\\
					1919 &Э.Резерфорд &   Открытие протона; доказательство  \\
					~ & ~ & наличия в ядрах элементов протонов; \\
					~ & ~ & первая искусственная ядерная реакция \\
					~ & ~ & --- превращение азота в кислород \\
					1920 & Г.Хевеши & Явление изотопного обмена\\
					1921 & Н.Бор & Строение атомов --- связь периодичности  \\
					~ & ~ & их химических и спектральных свойств с \\
					~ & ~ &  характером формирования электронных конфигураций \\
					~ & ~ &  по мере роста заряда ядра \\
					1921 & О.Ган  & Открытие изомера урана\\
					1921 & Ф.Астон  & {\color{red}212 природных изотопов различных элементов;} \\
					~ & ~ &   {\color{red}Массы изотопов --- целые числа} \\
					\hline
				\end{tabular}
			}	
			\caption{Открытия и введение понятий в науке об изотопах}
			\label{t:IsotopeHistory}
		\end{table}
		
		<<Понятие атомного веса восходит ко временам Джона Дальтона в начале 19 века. Большая часть химии в первой половине того века была связана с измерением и анализом атомных весов. Многие ученые, в первую очередь Дмитрий Менделеев, проанализировали и разделили атомный вес элементов на триады, октавы и спирали, основываясь на сходстве химических и физических свойств этих элементов. Менделеев предоставил периодическую таблицу вместе с предсказаниями новых элементов, чтобы заполнить пробелы в своей таблице, и эти элементы впоследствии были открыты.>>  \cite{IUPACIntervalHistory}. \index{Менделеев Д.И.} \index{Дальтон Дж.}
		
		Важным шагом на пути к открытию изотопов стало обнаружение явления радиоактивности
		А.Беккерель, П.Кюри, М.Кюри) и сформулированная Э.Резерфордом и Ф.Содди \index{гипотеза радиоактивного распада}
		гипотеза радиоактивного распада: радиоактивность есть не что иное, как распад атома на заряженную
		частицу и атом другого элемента, по своим химическим свойствам отличающийся от исходного. В
		результате возникло представление о радиоактивных рядах или радиоактивных семействах, \index{радиоактивные ряды}, \index{радиоактивные семейства}
		в начале
		которых есть первый материнский элемент, являющийся радиоактивным, и в конце – последний
		стабильный элемент. Анализ цепочек превращений показал, что в их ходе в одной клеточке
		периодической системы могут оказываться одни и те же радиоактивные элементы, отличающиеся лишь
		атомными массами. Фактически это и означало введение понятия изотопов. 
		
		В 1906—07 выяснилось, что продукт радиоактивного распада урана — ионий ($Jo, ^{230}Th$) и
		продукт радиоактивного распада тория — радиоторий ($RdTh$, $^{228}Th$) имеют те же химические свойства,
		что и торий ($^{232}Th$), однако отличаются от последнего атомной массой и характеристиками
		радиоактивного распада. Более того, как было обнаружено позднее, все три элемента имеют
		одинаковые оптические и рентгеновские спектры. Такие вещества, идентичные по химическим
		свойствам, но различные по массе атомов и некоторым физическим свойствам, по предложению
		английского учёного Фредерика Содди, стали называть изотопами. (По образному выражению
		Ф.Содди, атомы изотопов одинаковы «снаружи», но различны «внутри»). \index{изотоп}
		
		После того как изотопы были обнаружены у тяжёлых радиоактивных элементов, начались поиски
		изотопов у стабильных элементов. В 1913 английский физик Дж. Томсон в экспериментах с пучками
		положительно заряженных частиц (каналовые лучи), выходящих из разрядной трубки, обнаружил
		изотопы у неона (20Ne и 22Ne). Разработанный им метод парабол позволял определить отношение массы
		иона к его заряду по отклонению в параллельно направленных электрическом и магнитном полях
		тонкого пучка положительных ионов, получаемых в высоковольтном электрическом разряде. Наряду с
		атомами 20Ne Томсон наблюдал небольшую примесь более тяжёлых атомов. Однако убедительных
		доказательств того, что вторая компонента более тяжёлых атомов является изотопом неона, получено не
		было. 
		
		В 1919 Астон сконструировал прибор, названный масс-спектрографом (или масс-спектрометром). В качестве источника ионов по-прежнему использовалась разрядная трубка как в оптытах Дж.Томсона, однако
		Астон нашел способ, при котором последовательное отклонение пучка частиц в электрическом и
		магнитном полях приводило к фокусировке частиц с одинаковым значением отношения заряда к массе
		(независимо от их скорости) в одной и той же точке на экране. Наряду с Астоном масс-спектрометр
		несколько другой конструкции в те же годы был создан американцем Демпстером. В результате
		последующего использования и усовершенствования масс-спектрометров усилиями многих
		исследователей к 1935 году была составлена почти полная таблица изотопных составов всех известных
		к тому времени химических элементов.
		
		\begin{figure}[ht] 
			\centering\small
			\unitlength=1mm
			\includegraphics[height=150mm]{Figures/1925AstonTableElemIsotopes.png} 
			\caption{Таблица элементов и изотопов Астона-Демпстера 1925 г.} 
			\label{f:Aston1925}
		\end{figure}	
		
		
		С помощью масс-спектрографа, были получены надёжные доказательства существования двух
		изотопов 20Ne и
		22Ne, относительное содержание (распространённость) которых в природе составляет
		приблизительно 91% и 9%. В дальнейшем был обнаружен изотоп 21Ne с распространённостью 0,26%,
		изотопы хлора, ртути и ряда других элементов. 
		
		Примерно к 1940 изотопный анализ был осуществлен
		для всех существующих на Земле элементов. В результате этого к 1950 были выявлены и
		идентифицированы практически все стабильные и долгоживущие радиоактивные изотопы природных
		элементов. 
		
		
		
		В 1920-х годах были созданы первые ускорители элементарных частиц.
		Ядернофизические исследования пошли широким фронтом. Постепенно были открыты и тщательно изучены изотопы всех элементов на Земле. 
		В середине XX века были созданы ядерные реакторы, получена возможность получения интенсивных потоков нейтронов. По мере развития ускорительной техники появилась возможность заглянуть внутрь ядра, осуществлять синтез неизвестных ранее элементов. Началось изучение космических лучей и далёкого космоса. 
		
		Развитие экспериментальной техники и методов работы с радиоактивными элементами ввело изотопы в науки о Земле. Создание масс-спектрометров и других приборов позволило напрямую определять изотопный состав. Во второй половине XX века изотопы вошли в биологию и медицину.


%%%%%%%%%%%%%%%%%%%%%%%%%%%%%%%%%%%%%%%%%%%%%%%%%%%%%%%%%%%%%%%%%%%%%%%%%%%%%%%%%%%%

\chapter{Изотопы нуклидов во Вселенной и на Земле}\label{s:IntNature}

Интервальность явлений в природе проявляется начиная с самых основ строения веществ, образующих нашу планету. В этой главе мы рассмотрим Периодический закон расположения элементов, а далее рассмотрим ряд изотопных распределений на Земле.


\section{Сведения из ядерной физики}\label{s:NuclPhys} 

Вещества в природе состоят из химических элементов.
Под химическим элементом понимают совокупность атомов с одинаковым зарядом атомных ядер. \index{химический элемент}

Атомное ядро состоит из протонов, число которых равно атомному номеру элемента, и нейтронов, число которых может быть различным.  В связи с этим, для понимания закономерностей строения ядер атомов необходимы некоторые сведения из ядерной физики, см., например книгу \cite{NuclPhys}.

Изотопы (от isos – одинаковый и tópos — место), разновидности атомов одного химического элемента,
атомные ядра которых содержат одинаковое число протонов и различное число нейтронов; имеют одинаковое
число электронов в атомной оболочке и занимают одно место в Периодической Системе Элементов, отличаясь
массами атомов. Различают устойчивые (стабильные) и радиоактивные изотопы. \index{изотоп}

Изобары – \index{изобары} атомные ядра с одинаковым числом нуклонов $А$, но разным числом протонов $Z$  и нейтронов $N$ .

Нуклид – вид атомов, характеризующихся определенным числом протонов (зарядом ядра или атомным номером \index{нуклид}
$Z$) и нуклонов (массовое число $А$).   Обозначают символом соответствующего химического элемента с индексами $А$ --- слева вверху и $Z$ --- слева внизу, например, $_6С^{12}C$.

Нуклиды одного элемента называются изотопами. Нуклиды, претерпевающие радиоактивный распад – радионуклидами.

Радионуклид --- \index{радионуклид} атомное ядро, способное к радиоактивному распаду. 

Нуклон --- общее название протона и нейтрона, являющихся составными частями атомных ядер. 

\index{нуклон}

Химические свойства атомов определяются электрическим зарядом ядра, то есть числом протонов в ядре. При этом скорость протекания химических и физических процессов также зависит от массы ядра, то есть от числа нейтронов в ядре. Поэтому  различные изотопы атомов имеют различные свойства: с разной скоростью участвуют в химических ракциях, по-разному накапливаются в организмах, стратифицируются в зависимости от высоты земной поверхности и многое другое. Кроме того, не все изотопы ядер стабильны, и это тоже сказывается на их распределении в природе.

\subsection{Распределение ядер в природе}\label{s:NZ}

Акиуальным источником данных является база данных МАГАТЭ \cite{NUBASE2020}.
Интерактивные модели различных свойств данных размещены на сайте МАГАТЭ  \cite{IAEA}.
%Таблица свойств ядер --- https://www-nds.iaea.org/amdc/ame2020/mass_1.mas20.txt

Обычно в ядрах нейтронов больше чем протонов. Этот факт иллюстрирует так называемая диаграмма $N-Z$ атомных ядер. \index{диаграмма $N-Z$ атомных ядер}

\begin{figure}[ht] 
	\centering\small
	\unitlength=1mm
	\includegraphics[width=100mm]{Figures/NZdiagram.png} 
	\caption{$N-Z$ диаграмма  атомных ядер \cite{NuclPhys}} 
	\label{f:NZdiagram}
\end{figure}
Черным цветом выделены стабильные ядра --- долина стабильности атомных ядер.  \index{долина стабильности атомных ядер}
Справа от нее располагаются ядра, испытывающие $\beta^{-}$-распад, слева - ядра, испытывающие $\beta^{+}$-распад и $e$-захват. В области больших $A = N+Z$ находятся ядра, испытывающие $\alpha$-распад, и спонтанно делящиеся ядра. 
Линия $B_p = 0$ (proton drip-line) ограничивает
область существования атомных ядер слева, линия $B_n = 0$ (neutron drip-line) --- справа.

На Рис.~\ref{f:NZdiagram} большинство элементов лежит ниже прямой $N=Z$.
Также на Рис.~\ref{f:NZdiagram} выделены ядра с определёнными числами протонов и нейтронов. Поясним этот факт.

Cвойства ядер весьма сложны. Протоны и нейтроны участвуют во всех видах известных взаимодействий, сильных, слабых, электромагнитных и гравитационных. Поэтому нет такой теоретической модели ядер, которая бы успешно количественно объясняла все их наблюдаемые свойства.

Для начального представления о моделях следует упомянуть капельную и оболочечную модели ядра.

\subsection{Капельная и оболочечная модели ядра}\label{s:NuclModels}


\paragraph{Капельная модель ядра}
% https://ru.wikipedia.org/wiki/%D0%9A%D0%B0%D0%BF%D0%B5%D0%BB%D1%8C%D0%BD%D0%B0%D1%8F_%D0%BC%D0%BE%D0%B4%D0%B5%D0%BB%D1%8C_%D1%8F%D0%B4%D1%80%D0%B0
Формула Вайцзеккера \index{формула Вайцзеккера} \cite{Weizsacker}
\begin{equation}\label{Weitzekker}
 E_{c}=\alpha A-\beta A^{2/3}-\gamma {\frac {Z^{2}}{A^{1/3}}}-\varepsilon {\frac {(A-2Z)^{2}}{A}}+\delta.
\end{equation}
Первый член \eqref{Weitzekker} растёт линейно с массовым числом $A$. Следующие члены соответствуют различным поправкам:
\begin{itemize}
	\item Поправка на эффект поверхностного натяжения
	\item Поправка на кулоновское отталкивание
	\item Поправка на протон-нейтронную асимметрию
	\item Поправка на влияние чётности
\end{itemize}
Величина поправки чётности в полной энергии связи для четных-четных и нечетных-нечетных ядер, как функция массового числа

Можно разбить все ядра на три группы:
\begin{itemize}
	\item 1) чётно-чётные ядра ($A$ --- чётное)
	\item (2) нечётно-чётные и чётно-нечётные ($A$--- нечётное)
	\item (3) нечётно-нечётные ($A$ --- чётное)
\end{itemize}

Увеличение или уменьшение числа протонов или нейтронов на единицу скачком переводит ядро из одной группы в другую, соответственно скачком должна при этом изменяться энергия связи. Этот экспериментальный факт учитывается введением в формулу члена $\delta$ следующим образом:
\begin{equation}
 \delta =
 \begin{cases}+\left|\delta \right|&(1)\\0&(2)\\-\left|\delta \right|&(3)
\end{cases}
\end{equation}
Было экспериментально установлено что значение  $\delta$  зависит от массового числа:
\begin{equation}
\left|\delta \right|=\chi A^{k}
\end{equation}
Значение $k$
обычно берут либо -1/2, либо -3/4.

Численные значения коэффициентов в формуле Вайцзеккера
\begin{itemize}
	\item $\alpha =15.56$
	\item $\beta =17.23$
	\item $\gamma =0.71$
	\item $\varepsilon =23.7$
\end{itemize}
$ \chi =12, k=-1/2$ и 
$ \chi =34, k=-1/4$


Формула Вайцзеккера позволяет вычислять энергию связи ядра по известным $A$ и $Z$ с точностью ~10 МэВ. При $A\approx 100$ это даёт относительную погрешность $10^{-2}$. Массу любого ядра можно вычислять с точностью $10^{-4}$:
\begin{equation}
	M = Z m_{p}+(A-Z) m_{n}\left[\alpha A-\beta A^{2/3} -\gamma {\frac {Z^{2}}{A^{1/3}}}-\varepsilon {\frac {(A-2Z)^{2}}{A}}+\delta \right]/c^{2},
\end{equation}	
где $m_{p}$ --- масса протона, $m_{n}$ --- масса нейтрона, $c$ --- скорость света.

Так как капельная модель является макроскопической теорией,
то она не учитывает микроскопического строения ядра, например, распределения ядерных оболочек.
Поэтому формула Вайцзеккера плохо применима для магических ядер.
В рамках капельной модели считается, что ядро должно делиться на два фрагмента равной массы,
но это наблюдается лишь с вероятностью около 1 \%
(обычно один из осколков деления тяжёлых ядер стремится обладать магическим числом 50 или 82,
то есть массы фрагментов будут различаться примерно в 1,5 раза).
%Также капельная модель непригодна для количественного описания спектров энергий возбуждённых состояний ядер.

\paragraph{Оболочечная модели ядра.}
Экспериментальные исследования атомных ядер выявили некоторую периодичность в изменении
индивидуальных характеристик (энергии связи, спины, магнитные моменты, четности, некоторые особенности $\alpha$- и $\beta$-распадов) основных и возбужденных состояний атомных ядер. \index{$\alpha$- и $\beta$-распады}

Обнаруженная периодичность подобна периодичности свойств электронных оболочек атома и определяется так называемыми \emph{магическими} числами нейтронов и протонов. 

В частности, было обнаружено, что наибольшую энергию связи имеют ядра с так называемыми \emph{магическими} числами нейтронов и протонов, равными \index{магические числа нейтронов и протонов}
\begin{align}
		N & \quad 2, 8, 20, 28, 50, 82, 126, 184(?) \label{MagicN} \\
		Z & \quad 2, 8, 20, 28, 50, 82, 114(?) \label{MagicP}
\end{align}

 Особой стабильностью характеризуются так называемые дважды магические ядра, в которых количества и протонов, и нейтронов составляют магические числа. В природе существуют следующие дважды магические ядра: \index{дважды магические ядра}
 \begin{equation*}
^A_Z Element ^N :  \ ^4_2He^2, \ ^{16}_8O^8, \ ^{40}_{20}Ca^{20}, \ ^{48}_{20}Ca^{28}, \ ^{208}_{82}Pb^{126}.
 \end{equation*}




Результатом работы по систематизации и обобщения огромного
количества экспериментальных данных было создание в середине XX века
модели оболочек атомных ядер. \index{модель оболочек атомных ядер}  
Эта модель позволяет не только объяснить многие закономерности свойств ядер, но и предсказывать возможные пути синтеза новых элементов.

Что касается распространённости изотопов в природе, то имеется ряд закономерностей \cite{Bekman}.
Оболочечная модель согласуется с эмпирическим правилом повышения устойчивости ядер,
содержащих чётное число протонов и нейтронов --- Табл.~\ref{t:IsotopeHistory}. Менее стабильны ядра с нечётным числом протонов, но
чётным числом нейтронов (и наоборот). Неустойчивыми являются ядра, в которых число протонов и
нейтронов нечётное. Последняя закономерность хорошо иллюстрируется распределением естественных
известных 273-х природных изотопов по правилу чётности. 

\begin{table}[!h]
{\small 
\begin{center}
\begin{tabular}{|c|c|c|c|c|}
	\hline
	Число изотопов & N & Z & Примеры Табл.~\ref{t:IsotopeCounts}\\
	\hline
	166 & чётное & чётное & ~ \\
	55 & чётное & нечётное & 1 изотоп --- $_{9}F^{19} \ldots \, _{83}Bi^{109} $ \\
	~ & ~ & ~ &  2 изотопа --- $_{1}H^{1}  \ldots \, _{81}Tl^{203, 205}   $ \\
	47 & нечётное & чётное & ~  \\
	5 & нечётное & нечётное & $_{1}H^{2},   \,  _{3}Li^{6},  \,  _{5}B^{10},   \, _{7}N^{14},  \,  _{47}Ag^{109} $   \\
	\hline
\end{tabular}
\end{center}	
}	
\caption{Распределение естественных изотопов по правилу чётности нуклонов}
\label{t:IsotopeOddEven}
\end{table}

Для 24 элементов известно лишь по одному устойчивому изотопу. Такие элементы называют
моноизотопными. Из них 23 элемента - с нечётными $Z$. 
У других элементов, преимущественно с чётными порядковыми номерами, число устойчивых изотопов доходит до 10 --- Табл.~\ref{t:IsotopeCounts} \S\ref{s:IsotopeCounts}. \index{моноизотопные элементы}

\subsection{Изобары}\label{s:Isobars}

Изобары --- нуклиды разных элементов, имеющие одинаковое массовое число; например, изобарами являются $^{40}Ar, ^{40}K, ^{40}Ca$ --- см. Табл.~\ref{t:IsotopeHistory}. \index{изобары}

Хотя массовое число (то есть число нуклонов) $A = N + Z$ в ядрах-изобарах одинаково, числа протонов $Z$ и нейтронов $N$ различаются: 
\begin{equation*}
	Z_{1}\neq Z_{2}, N_{1}\neq N_{2}. 
\end{equation*}

Существуют 58 примордиальных изобарных пар и 9 примордиальных изобарных триад, которые в основном включают в себя стабильные изотопы элементов с чётными $Z$, отличающимися на 2 единицы. Если учитывать только стабильные нуклиды, то существуют 48 изобарных пар.


\begin{table}[!h]
	{\small 
		\begin{center}
			\begin{tabular}{|c|c|c|c|c|c|}
				\hline
				Номер	& Массовое & Изобарная  &	Номер	& Массовое & Изобарная \\
				~	& число	& пара &	~	& число	&  пара\\
				\hline
				1	&36& $_{16}S$ $_{18}Ar$ &	~	& ~	&  ~ \\				
				\hline
			\end{tabular}
		\end{center}	
	}	
	\caption{Примордиальные изобарные пары}
	\label{t:IsobarPairs}
\end{table}

\subsection{Изобарические цепочки}\label{IsobarChains}

Совокупность нуклидов с одинаковым $A$, но разным $Z$ называют \emph{изобарической цепочкой}. \index{изобарическая цепочка} В то время как массовое число изобаров одинаково, их атомные массы совпадают лишь приближённо. Зависимость атомной массы (или избытка массы) от $Z$ в изобарической цепочке показывает направление возможных бета-распадов. Эта зависимость в первом приближении представляет собой параболу (см. формула Вайцзеккера \eqref{Weitzekker}) --- сечение долины стабильности плоскостью $A = const$. 

Для изобаров справедливо \emph{правило Щукарева — Маттауха}, объясняющее, в частности, отсутствие стабильных изотопов у технеция. \index{правило Щукарева — Маттауха}
Суть правила заключается в том, что в природе не могут существовать два стабильных изобара, заряды ядра которых отличаются на единицу. Другими словами, если у какого-либо химического элемента есть устойчивый изотоп, то его ближайшие соседи по таблице не могут иметь устойчивых изотопов с тем же массовым числом. \index{технеций}

Правило объясняет, в частности, отсутствие стабильных изотопов у технеция несмотря на то, что он находится в таблице Менделеева задолго до свинца: соседние с ним молибден и рутений имеют стабильные изотопы с массовыми числами $92, 94, 95, 96, 97, 98, 100$ и $96, 98, 99, 100, 101, 102, 104$, соответственно --- см. Табл.~\ref{t:IsotopeCounts}.
Подробная публикация --- \cite{Technetium2017}.



\section{Синтез элементов во Вселенной}\label{s:Nucleosynthesis}

Для понимания распределения изотопов элементов в природе необходимо знать, как они образовывались. Наиболее  полное издание по этому вопросу --- книга \cite{Nucleosynthesis}, краткое изложение вопроса содержится в статье \cite{ElementsOrigin}. 

По современным представлениям, Вселенная развивается согласно \emph{стандартной космологической модели} (Большого взрыва). Согласно этой модели, более 15 млрд. лет назад началось и продолжается поныне расширение Вселенной из некоторого сверсжатого и сверхгорячего состояния. \index{стандартная космологическая модель}

Образование различных элементов происходит на разных этапах развития Вселенной. Самые лёгкие элементы --- водород и гелий --- рождаются на начальном этапе, вскоре после Большого взрыва. Первичной базовой реакцией является слияние протона и нейтрона --- образование легчайшего составного ядра  дейтерия \index{процесс образования ядра  дейтерия}
\begin{equation}\label{npdgamma}
	n + p \longrightarrow d + \gamma.
\end{equation}
Энергия связи дейтона равна 2.22 МэВ, и весьма мала по меркам ядерной физики. Поэтому в горячей Вселенной дейтон быстро разрушается под дейтсвием гамма-квантов. По мере охлаждения Вселенной при её расширении, обратная к \eqref{npdgamma} реакция \index{процесс фоторасщепления ядра  дейтерия}
\begin{equation}\label{gammad}
		\gamma +	d  \longrightarrow n + p 
\end{equation}
становится менее вероятной, и появляется возможность старта процесса \emph{нуклеосинтеза}. \index{нуклеосинтез}

В ещё горячей Вселенной, а затем в звёдах образуются за счёт термоядерных реакциий лёгкие элементы, и далее элементы средней массы вплоть до железа. Энергия связи этого элемента максимальна. Дальнейший синтез тяжёлых элементов невыгоден из-за того, что энергия связи в более тяжёлых элементах становится меньше.

Синтез более тяжёлых элементов идёт уже в ядерных реакциях захвата нейтронов ядрами. При этом образуются более тяжелые изотопы существующих ядер, в основном в процессах радиационного захвата нейтронов \index{процесс радиационного захвата нейтронов}
\begin{equation}\label{neutroncapture}
	^N_ZA(n, \gamma)^{N+1}_ZA
\end{equation}
а также ядра с б\'{о}льшим зарядом за счёт $\beta$-распада  \index{процесс $\beta$-распада}
\begin{equation}\label{betadecay}
	^N_zA \longrightarrow ^N_{Z+1}B+e^{-}.
\end{equation}
При этом возможны и процессы распада получившихся ядер в зависимости от их стабильности по отношению к различным распадам, о чём говорится в \S\ref{s:NuclPhys}.

На Рис.~\ref{f:NuclidsN} представлена относительная распространенность нуклидов от атомной массы.
Обозначения указывают на различные процессы синтеза, наиболее значимые из них мы рассмотрим далее.
\begin{figure}[ht] 
	\centering\small
	\unitlength=1mm
	\includegraphics[width=120mm]{Figures/NuclidsN.png} 
	\caption{Относительная распространенность нуклидов от атомной массы \cite{ElementsOrigin}.
	Обозначения указывают на различные процессы синтеза } 
	\label{f:NuclidsN}
\end{figure}

Важно связать информацию Рис.~\ref{f:NuclidsN} с закономерностями строения ядра \S\ref{s:NuclPhys}. Возьмём менее подробное представление распространенностb  нуклидов, показанное на Рис.~\ref{f:NuclidsNSchematics}.
\begin{figure}[ht] 
	\centering\small
	\unitlength=1mm
	\includegraphics[width=60mm, height=50 mm]{Figures/NuclidsNSchematics.png} 
	\caption{Схематическое изображение распространенности нуклидов от атомной массы \cite{ElementsOrigin}.
		Обозначения указывают на различные процессы синтеза } 
	\label{f:NuclidsNSchematics}
\end{figure}

Прокомментируем Рис.~\ref{f:NuclidsNSchematics}. Распространённость элементов тяжелее гелия в целом экспоненциально снижается с ростом атомной массы $A$ до 100 и после $A >120$ практически постоянна. Наблюдается локальный максимум группы железа $Fe-Co-Ni$. 
Он отвечает максимальной энергии связи нуклонов в этих ядрах. 

В дальнейшем ходе зависимости имеется 3 локальных двойных максимумов. Одни из элементов пар лежат в области магических нейтронных чисел $N = 50, 80, 126$ --- см. \eqref{MagicN}. Массы ядер при этом равны $A = 90, 138, 208$. Другие элементы сдвинуты примерно на 10-15 массовых числел. Эти особенности помечены  буквами $s$ и $r$.

С точки зрения конечного состава планетного вещества важно понимать пути нуклеосинтеза тяжелых элементов и равновесное состояние на настоящее время. Общая схема образования элементов от железа до актинидов начинается с ядра железа.


Возможны различные траектории (схемы) нуклеосинтеза. Основными из них являются реакции  медленного нейтронного захвата ($s$-процесс) и быстрого нейтронного захвата ($r$-процесс). Рассмотрим кратко эти схемы, следуя \cite{Nucleosynthesis, ElementsOrigin}.

\begin{figure}[ht] 
	\centering\small
	\unitlength=1mm
	\includegraphics[width=100mm]{Figures/NucleosythesisTrajectory.png} 
	\caption{Иллюстрация хода $s$- и $r$-процессов \cite{ElementsOrigin}} 
	\label{f:NucleosythesisTrajector}
\end{figure}

Образование этих элементов в результате взаимодействия заряженных частиц сильно подавлено из-за кулоновского барьера. Фактор, который также необходимо принять во внимание, состоит в том, что большинство тяжелых элементов являются бета-радиоактивными.

\subsection{Процесс медленного нейтронного захвата} \index{процесс медленного нейтронного захвата ($s$-процесс)}

По современным представлениям тяжелые элементы образуются в реакциях захвата нейтронов. Обычно различают быстрый (r) и медленный (s) процессы захвата нейтронов (от английских слов rapid и slow). Эти два механизма различаются отношением скорости захвата нейтронов (реакция $(n, \gamma)$) к скорости бета-распада. При условии 
\begin{equation}\label{sprocess}
	\frac{\tau_{\beta}} {\tau_{(n, \gamma)}}  \ll  1 
\end{equation}
в цепочку процессов образования тяжелых элементов будут вовлечены только стабильные и бета-радиоактивные ядра с большими периодами полураспада. То есть образование элементов будет происходить вдоль долины бета-стабильности. Нейтроны добавляются к ядрам последовательно. При этом могут образоваться только сравнительно устойчивые ядра. Ядра с малыми периодами полураспада исчезают раньше, чем они успевают захватить следующий нейтрон. Поэтому ясно, что образование тяжелых элементов должно заканчиваться свинцом и висмутом.

\subsection{Процесс быстрого нейтронного захвата} \index{процесс быстрого нейтронного захвата ($r$-процесс)}

В настоящее время общепризнанно, что многие ядра тяжелее железа, включая все ядра тяжелее $^{209}Bi$, образуются в $r$-процессе путем быстрого последовательного захвата большого количества нейтронов. Главное условие - скорость захвата нейтронов должна быть больше скорости бета-распада. Захват нейтронов происходит до тех пор, пока скорость реакции $(n, \gamma)$ не станет меньше скорости распада изотопа. 

\begin{equation}\label{rprocess}
	\frac{\tau_{\beta}} {\tau_{(n, \gamma)}}  >  1 
\end{equation}
Образующееся ядро распадается затем в результате бета-распада и вновь начинается последовательный захват нейтронов.
Линия, вдоль которой происходит образование ядер в $r$-процессе, смещена от дорожки стабильности (трека $s$-процесса) в направлении нейтроноизбыточных изотопов --- Рис.~\ref{f:NuclidsNSchematics}.

\section{Распространённость изотопов на Земле и их фракционирование}\label{s:IsotopeCounts}

В \S\ref{s:Nucleosynthesis} показано, что распределение изотопов в космическом веществе имеет ряд закономерностей. Особенности конкретной планеты, цикла её формирования и развития накладывают свой отпечаток на элементный и изотопный состав различных пород и географических положений. 

В \cite{IsotopeGeoChem} в удобной приведены данные о количестве стабильных изотопов различных элементов на Земле. Приведём часть данных из этой публикации.
\begin{table}
{\scriptsize 
\begin{tabular}{cp{6cm}cc}
Число  & Элементы, $ \ _{Z}E^{N_1, N_2, \ldots, N_k}$, $z \ $ --- заряд ядра,  &  Число  & Общее  \\
стабильных  & $N_j$ --- атомный вес изотопа $j$ &  элементов & число  \\
изотопов, $k$ & ~ &  ~ &  элементов \\
\hline 
~ & ~ &  ~ &  ~\\
1 &  $_{4}Be^{9}, \, _{9}F^{19}, \, _{11}Na^{23}, \, _{13}Al^{27}, \, _{15}P^{31}, \,  _{21}Sc^{45},  \,  _{23}V^{51},  $ & 24 & 24 \\ [1mm]
~ &  $_{25}Mn^{55}, \, _{27}Co^{59}, \, _{33}As^{75}, \, _{39}Y^{89}, \, _{41}Nb^{93}, \,  _{45}Rh^{103},  $ & ~  & ~ \\ [1mm]
~ &  $ _{53}I^{127}, \, _{55}Cs^{133}, \, _{57}La^{139}, \, _{59}Pr^{141}, \, _{65}Tb^{159},  \, _{67}Ho^{165}, $ & ~  & ~ \\ [1mm]
~ &  $ _{69}Tm^{169}, \, _{71}Lu^{175}, \, _{75}Ta^{181}, \, _{79}Au^{197}, \, _{83}Bi^{109}  $ & ~  & ~ \\ [1mm]
%\hline
2 &  $_{1}H^{1, 2},  \,  _{2}He^{3, 4},  \,  _{3}Li^{6, 7},  \,  _{5}B^{10, 11},  \, _{6}C^{12,13},  \, _{7}N^{14,15}, $ & 19 & 38 \\ [1mm]
~ &  $_{17}Cl^{35, 37},  \,  _{19}K^{39, 41},  \,  _{29}Cu^{63, 65},  \,  _{31}Gs^{61, 71},  \, _{35}Br^{79,81},  $ & ~ & ~\\ [1mm]
~ &  $_{37}Rb^{85, 87},  \,  _{47}Ag^{107, 109},  \,  _{49}In^{113, 115},  \,  _{63}Eu^{151, 153},    $ & ~ & ~\\ [1mm]
~ &  $ _{76}Re^{185, 187}, \,  _{77}Ir^{191, 193},  \,  _{81}Tl^{203, 205} $ & ~ & ~ \\ [1mm]
3 &  $_{8}O^{16, 17, 18}, \, _{10}Ne^{20, 21, 22}, \,  _{12}Mg^{24, 25, 26} $ & 6 & 18 \\ [1mm]
~ &  $_{14}Si^{28, 29, 30}, \, _{16}Ar^{36, 38, 40}, \,  _{58}Ce^{138, 140, 142} $ & ~ & ~ \\ [1mm]
4 &  $_{16}S^{32, 33, 34, 36}, \, _{24}Cr^{50, 52, 53, 54}, \,  _{28}Fe^{54, 56, 57, 58} $ & 5 & 20 \\ [1mm]
~ &  $_{38}Sr^{84, 86, 87, 88}, \,  _{82}Pb^{204, 206, 207, 208} $ & ~ & ~ \\ [1mm]
5 &  $_{22}Ti^{46-50}, \, _{28}Ni^{58, 60, 61, 62, 64}, \, _{30}Zn^{64, 66, 67, 68, 70} $ & 6 & 30 \\ [1mm]
~ &  $_{32}Ge^{70, 72, 73, 74, 76}, \, _{40}Zr^{90, 91, 92, 94, 96}, $ & ~ & ~ \\ [1mm]
~ &  $_{74}W^{180, 182, 183, 184, 186} $ & ~ & ~ \\ [1mm]
6 &  $_{20}Ca^{40, 42, 43, 44, 46, 48}, \, _{34}Se^{74, 76, 77, 78, 80, 82} $ & 7 & 42 \\ [1mm]
~ &  $_{36}Kr^{78, 80, 82, 83, 84, 86},  \,_{46}Pd^{102, 104, 105, 106, 108, 110} $ & ~ & ~ \\ [1mm]
~ &  $_{68}Er^{162, 164, 166, 167, 168, 170},  \, _{72}Hf^{174, 176, 177, 178, 179, 180} $ & ~ & ~ \\ [1mm]
~ &  $_{78}Pt^{190, 192, 194, 195, 196, 198} $ & ~ & ~ \\ [1mm]
7 &  $_{42}Mo^{92, 94-98, 100}, \, _{44}Ru^{96, 98-102, 104}, \, $ & 10 & 70 \\ [1mm]
~ &  $_{56}Ba^{130, 132, 134-138}, \, _{60}Nd^{142-146, 148, 150}, $ & ~ & ~ \\ [1mm]
~ &  $_{62}Sm^{144, 147-150, 152, 154}, \, _{64}Gd^{152, 154-158, 160},$ & ~ & ~ \\ [1mm]
~ &  $_{66}Dy^{156, 158, 160-164}, \, _{70}Yb^{168, 170-174, 176} $ & ~ & ~ \\ [1mm]
~ &  $_{76}Os^{184, 186-190, 192}, \, _{80}Hg^{196, 198-202, 204} $ & ~ & ~ \\ [1mm]
8 &  $_{48}Cd^{106, 108,  110-114}, \, _{52}Te^{120, 122, 123-126, 128, 130}$ & 2 & 16 \\ [1mm]
9 &  $_{54}Xe^{124, 126, 128-132, 134, 136}$ & 1 & 9 \\ [1mm]
10 &  $_{50}Sn^{112, 114-120, 122, 124}$ & 1 & 10 \\ [1mm]
\hline
~ & ~ &  ~ &  ~\\
	~ & ~ & 81 & 276
	\end{tabular}
}
\caption{Количество стабильных изотопов различных элементов на Земле}
\label{t:IsotopeCounts}
\end{table} 


%\input{IsotopeCounts.tex} 

Как видим по данным Табл.~\ref{t:IsotopeCounts}, имеется большое многообразие количества изотопов элементов. Веществ с одним изотопом всего 24. Их принято называть \emph{моноизотопными элементами}. 



Известно 276 стабильных изотопов, принадлежащих 81 природному элементу, и около 2500
радиоактивных 105 природных и синтезированных элементов.
Анализ соотношений между числами нейтронов и протонов для различных изотопов одного и того же
элемента показывает, что ядра стабильных изотопов и радиоактивных изотопов, устойчивых по отношению к
бета-распаду, содержат на каждый протон не менее одного нейтрона. Исключение из этого правила
составляют лишь два нуклида - $^1H$ и $^3He$. По мере перехода ко всё более тяжёлым ядрам отношение числа
нейтронов к числу протонов в ядре растет и достигает 1.6 для урана и трансурановых элементов.

\subsection{Связь строения ядра и распространённости изотопа в природе.}
Данные Табл.~\ref{t:IsotopeCounts} полезно сопоставить с Табл.~\ref{t:IsotopeOddEven} со сведениями по оболочечной модели ядра из раздела \S\ref{s:NuclPhys}   \cite{NuclPhys, Bekman}.

Элементы с нечётным $Z$ имеют не более двух стабильных изотопов. Как правило, число нейтронов $N$ в
таких ядрах чётное, и, следовательно, массовое число $А$ --- нечётное. 

Есть всего 5 изотопов, которые имеют нечётное число протонов и нейтронов:  $_{1}H^{2},   \,  _{3}Li^{6},  \,  _{5}B^{10},   \, _{7}N^{14},  \,  _{47}Ag^{109}$. Следует заметить, что изотопы $Li^{6}$ и $B^{10}$ очень охотно поглощают нейтроны. В природе нет стабильных изотопов технеция $_{43}Tc$, прометия $_{61}Pm$ и актиния $_{89}Ac$, это <<дыры>> в периодической таблице. Также именно на нечётных $Z$ завершаются списки стабильных элементов --- нет ядер $_{84}Po$ и $_{93}Np$.

Большинство элементов с чётным $Z$ имеет несколько стабильных изотопов, из которых не более двух с нечётным $A$. 
Есть только один моноизотопный элемент с чётным значением $Z$ --- $_{4}Be^{9}$.
Число элементов с 3-6 изотопами имеют соответственно 6, 5, 6 и 7 элементов,  10 элементов имеют 7 изотопов: молибден, рутений, барий, неодим, самарий, гадолиний, диспрозий, иттербий, осмий, ртуть, 8 изотопов --- у кадмия и теллура, 9 --- у ксенона.  Наибольшее число изотопов (10) имеет олово.

Широкие вариации в числе стабильных изотопов у различных элементов обусловлены сложной
зависимостью энергии связи ядра от числа протонов и нейтронов в ядре. По мере изменения числа нейтронов
N в ядре с данным числом протонов Z энергия связи ядра и его устойчивость по отношению к различным
типам распада меняются. При добавлении нейтронов ядро становится неустойчивым по отношению к
испусканию электрона с превращением одного нейтрона в ядре в протон. Поэтому нейтронообогащённые
изотопы всех элементов $\beta^{-}$-активны. Наоборот, при обеднении нейтронами ядро получает возможность или
захватить электрон из оболочки атома, или испустить позитрон. 
При этом один протон превращается в
нейтрон и оптимальное соотношение между числом протонов и нейтронов в ядре восстанавливается.

Нейтронообеднённые изотопы всех элементов испытывают или электронный захват или позитронный распад.
У тяжёлых ядер наблюдаются также альфа-распад и самопроизвольное (спонтанное) деление ядер.

Получение нейтроноизбыточных изотопов элементов возможно несколькими способами. Один из них -
реакция захвата нейтронов ядрами стабильных изотопов. Другой - деление тяжёлых ядер под действием
нейтронов или заряженных частиц, в результате которого из одного тяжёлого ядра с большим относительным
содержанием нейтронов образуются два нейтронообогащённых ядра. Нейтронообогащённые изотопы лёгких
элементов эффективно образуются в реакциях многонуклонного обмена при взаимодействии ускоренных
тяжёлых ионов с веществом. Синтез нейтроно-дефицитных изотопов осуществляется в ядерных реакциях под
действием ускоренных заряженных лёгких частиц или тяжёлых ионов.

\subsection{Изотопы в живой природе.}
Примечательной особенностью распределения стабильных изотопов различных элементов на Земле является тот факт, что все \emph{биогенные элементы}, немоноизотопны. \emph{Биогенными} называют элементы, постоянно входящие в состав организмов и выполняющие определённые биологические функции. Элементы и их соединения, требующиеся биоте в больших количествах, называют макробиогенными (в первую очередь --- $C, O, N, H$ из которых строятся все огранические молекулы, а также $Ca, P, S$), а в малых количествах --- микробиогенными.  \index{биогенные элементы}
Следствия из этого факта огромны:
имеется предпосылка для большого разнообразия вариантов накопления и обеднения изотопного состава различных пород, осадков, водных акваторий и иных природных ландшафтов и ниш.

В ходе геохимических процессов происходит разделение изотопов.
Причины разделения изотопов различны. Можно выделить среди них такие группы как геологические процессы, физическо-химические процессы в атмосфере и водной среде, а также биохимические процессы.

Физическо-химические процессы (изотопное фракционирование) включают такие группы как:  
\begin{list}{}{\itemsep=0pt\itemindent=12pt\topsep=2pt\parsep=2pt} 
\label{fractioning}
	\item [$\circ$] Диффузия
 %(Скорость диффузии He сквозь твёрдые материалы в три раза выше, чем у воздуха, и приблизительно на 65 \% выше, чем у водорода. Проникает сквозь кв. стекло)
\item  [$\circ$] Испарение 
\item  [$\circ$] Дистилляция  % (лат.distillatio—стекание каплями—перегонка, испарение жидкости с последующим охлаждением и конденсацией паров).
\item  [$\circ$] Кристаллизация 
\end{list} 

В качестве примера физическо-химического фракционирования на Рис.~\ref{f:EvaporationCondensation} представлен процесс измененения изотопного состава при испарении воды с поверхности океана и конденсации на континенте \cite{StableIsotopeDeltas}.
\begin{figure}[ht] 
	\centering\small
	\unitlength=1mm
	\includegraphics[width=100mm]{Figures/EvaporationCondensation.png} 
	\caption{Изотопные измененения при испарении и конденсации \cite{StableIsotopeDeltas} }
	\label{f:EvaporationCondensation}
\end{figure}
% Stable isotope deltas: Tiny, yet robust signatures in nature. March 2012Isotopes in Environmental and Health Studies 48(3):393-409 W.A. Brand,T.B. Coplen

Биохимические процессы также имеют огромное влияние на планету, начиная с формирования ландшафтов (береговые линии сухопутных и морских водоёомов), так и на образование и трансформацию пород и общепланетарный оборот веществ \cite{Zhuravlev2019}. \index{изотопное фракционирование биохимическое} 

Кроме того, изотопные методы позволяют исследовать и доказательно определять последствия антропогенных и техногенных процессов: археология и миграция людей, контроль лекарств, наркотиков и допинга, аутентичность продуктов питания и напитков. В основном для этого используются методы изотопной масс-спектрометрии \cite{ MS2013, Lebedev2003}, которые получили огромное развитие в последние десятилетия.
Практические примеры приводятся в  \S\ref{s:IsotopesEarth} <<Изотопы на Земле>> .


\section{Изотопы на Земле}\label{s:IsotopesEarth}


\subsection{Изотопная подпись}\label{s:IsotopeSignature}

Представленный выше материал является основой для понимания ряда новых методик исследования неживой и живой природы. 
В последние несколько десятилетий в науку прочно вошёл новый термин <<изотопная подпись>>. \index{изотопная подпись}

Википедия определяет изотопную подпись следующим образом \cite{IsotopeSignatureWiki}. 
<<Изотопная подпись  (или изотопная сигнатура) --- специфическое соотношение нерадиоактивных <<стабильных изотопов>> или относительно стабильных радиоактивных изотопов или неустойчивых радиоактивных изотопов определённых химических элементов в исследуемом материале. Соотношения изотопов в образце исследуют при помощи изотопной масс-спектрометрии.>>

Вхождение нового термина в научное обращние стало одним из результатов длительной работы представителей различных специальностей: биологов, географов, геологов, палеонтологов. Список можно расширить.

Результатом стала не просто фиксация различных изотопных соотношений в зависимости от происхождения исследуемого материала, а использования этих соотношений как инструмента исследования.

Например, для кислорода в документе \cite{IUPAC2018} приводятся следующие данные.
На рисунке \ref{f:OxygenNature} приведены вариации атомного веса и изотопного состава ряда материалов, содержащих кислород приведены для атмосферного воздуха, воде, углекислом газе, карбонатах, оксиде азота, других химических соединениях, растениях и животных на Земле. 

\begin{figure}[ht] 
	\centering\small
	\unitlength=1mm
	\includegraphics[width=100mm]{Figures/OxygenNature.png} 
	\caption{Вариации атомного веса и изотопного состава ряда материалов, содержащих кислород \cite{IUPAC2018} на Земле.} 
	\label{f:OxygenNature}
\end{figure}


Подобные данные есть и для других биогенных материалов: водорода, углерода, азота, фосфора. 
По представленному материалу видно, что изотопные соотношения для биогенных материалов и различные комбинации  иотопных соотношений можно использовать для определения происхождения неизвестного вещества, определять среду в которой он формировался, определять, был ли это живой организм или минерал, и многое другое. \index{биогенные материалы}

В доступном для неспециалиста изложении многочисленные примеры использования изотопных подписей приводятся в книге палеонтолога А.Ю.~Журавлёва <<Сотворение Земли. Как живые организмы создали наш мир>> \cite{Zhuravlev2019}.

Экологические материалы со ссылками на оргинальные статьи можно найти на сайте elementy.ru \cite{OpaevIsotope}.

Как правило, дефицит или избыток конкретного изотопа измеряют по отношению к общепринятому стандарту. Например,
\begin{equation*}
	\delta^{13}C_{\text{sample}} = \left( \frac{^{13}C/^{12}C_{\text{sample}}}{^{13}C/^{12}C_{\text{standard}}} - 1\right) \cdot 1000 \ \permil
\end{equation*}
Используется обозначение $\delta^{\text{mass isotope}}\text{Element}$, единицей измерения служит промилле, $\permil$, одна тысячная доля.
\index{промилле, $\permil$}

%Также в систему современных научных представлений входит понятие <<изотопная ниша>>.

\subsection{Изотопная ниша}\label{s:IsotopeNiche}

В последние годы в практику вошёл новый термин, <<изотопная ниша>>, относящийся к применению изотопов в биологии. Он конкретизирует широко используемый термин экологическая ниша. \index{изотопная ниша}

Изотопная ниша — это пространство, занимаемое видом в многомерном пространстве признаков, которые в этом случае являются значениями индексов $\delta^{13}С, \delta^{15}N, \delta^{18}O \text{ и } \delta^{2}H$. 

В популярной статье \cite{OpaevIsotope} приводятся результаты исследований музейных экспонатов 254 особей 12 видов птиц, оригинальная публикация \cite{Rader2017}.  Группа водяных печников распространена в Южной Америке, где разные виды населяют диапазон высот от 0 до 5000 м над уровнем моря.
\begin{figure}[ht] 
	\centering\small
	\unitlength=1mm
	{\includegraphics[width=45mm]{Figures/BirdsArealLeft.png}} 
	\caption {Слева — ареалы 12 видов водяных печников (Cincloides). \cite{Rader2017}.} %Справа — представители рода Cincloides: сверху вниз — водяные печники островной C. antarcticus, белокрылый C. atacamensis и полосатокрылый C. fuscus. \cite{Rader2017}.} 
\label{f:BirdsAreal}
\end{figure}

Фракционирование углерода происходит в природе разными путями. В частности, при фотосинтезе возможны 3 основных варианта: \index{тип фотосинтеза}
\begin{table}[h!]
\centering
\caption{Отношение $\delta^{13}C$ для разных механизмов фотосинтеза.} 
\medskip 
\begin{tabular}{|c|c|c|}
	\hline
	Тип фотосинтеза & $\delta^{13}C$ & Пример \\
	\hline 
	$C_4$ & [-16, -10] & зерно \\
	$CAM$ & [-20, -10] & фрукты\\
	$C_3$ & [-33, 24] & бобовые \\			
	\hline
\end{tabular}
\label{13Cplants}
\end{table} 
\index{тип фотосинтеза С3}
\index{тип фотосинтеза С4}
\index{тип фотосинтеза СAM}

Было определено соотношение тяжелых и легких изотопов углерода, азота, кислорода и водорода в  перьях птиц. Птицы меняют перья во время линьки, обычно приуроченной к определенному периоду года и длящейся 1–2 месяца. Поэтому изотопный состав перьев может рассказать о том, чем птица в это время питалась. 
\begin{figure}[ht] 
\centering\small
\unitlength=1mm
{\includegraphics[width=120mm]{Figures/IsotopesNiches.png}} 
\caption{Изотопные ниши различных видов птиц по углероду и азоту (слева) и по кислороду и водороду (справа) \cite{Rader2017}.} 
\label{f:IsotopesNiches}
\end{figure}


Изотопная ниша по углероду и азоту в какой-то степени является нишей трофической, так как характеризует питание. А ниша по кислороду и водороду — пространственная, так как зависит от местообитания (ведь изотопный состав воды — поставщика этих элементов — различается в разных местах). Чем шире изотопная ниша по углероду и азоту (то есть больше площадь соответствующего эллипса), тем больший спектр кормов потребляет данное животное, тем шире его трофическая ниша. \index{трофическая ниша}

\begin{figure}[ht] 
\centering\small
\unitlength=1mm
{\includegraphics[width=45mm]{Figures/IsotopesNichesWid.png}} 
\caption{Взаимосвязь ширины изотопной ниши по углероду и азоту (C/N) и по кислороду и водороду (O/H) для 12 видов водяных печников. Ширина ниши данного вида — это площадь соответствующего эллипса на рис. \ref{f:IsotopesNiches} \cite{Rader2017}.} 
\label{f:IsotopesNichesWid}
\end{figure}

Аналогично, чем больше площадь эллипса по кислороду и водороду, тем в более широком спектре местообитаний можно найти особей этого вида. Оказалось, что ширина ниши (то есть площадь эллипса) по углероду и азоту, с одной стороны, и по кислороду и водороду, с другой, положительно связаны между собой.


\subsection{ Изотопные ландшафты }\label{s:IsoScapes}

Изотопные ландшафты (Isoscapes) --- это географические карты, в легенду которых входит содержание тех или иных изотопов. На карту могут быть нанесены результаты измерений или моделирования. \index{изотопные ландшафты}

Приведём примеры из публикации \cite{Bowen2010}. Для изотопа  $\delta^{15}N$ в растениях данные представлены на рисунке \ref{f:Plant15N} и для изотопа  $\delta^{18}O$ в морской воде --- на рисунке \ref{f:SeaWater18O}.
\begin{figure}[ht] 
\centering\small
\unitlength=1mm
{\includegraphics[width=80mm]{Figures/Plant15N.png}} 
\caption{Изотопное отношение для $\delta^{15}N$ в растениях \cite{Bowen2010}.} 
\label{f:Plant15N}
\end{figure}


\begin{figure}[ht] 
\centering\small
\unitlength=1mm
{\includegraphics[width=80mm]{Figures/SeaWater18O.png}} 
\caption{Изотопное отношение для  $\delta^{18}O$ в морской воде \cite{Bowen2010}.} 
\label{f:SeaWater18O}
\end{figure}

%\section*{Заключение}
\paragraph{Изотопные интервалы в природе.}

Развитие изотопного анализа в последние несколько десятилетий обогатило исследователей различными возможностями. Многие науки и отрасли деятельности человека существенно изменились. Появилась возможность получать принципиально новые виды информации, строить новые логические связи между явлениями. Появились новые понятия  и концепции. С получением новых данных идёт обогащение идеями, ставятся новые вопросы.

При этом математика помогает описывать данные с интервальной неопределённостью  и работать с интревальнозначными величинами. Принципиальныо новым шагом стало введение IUPAC интервальных границ стабильных изотопов химических элементов в периодической системе.  Увеличение числа исследований в естественных науках неизбежно потребует и развития математических методов для  эффективной работы с данными. 




\section{Применение изотопов}\label{IsotopeApplications}
% https://www.ecp.ru/activity/nuclear/isotopes-use

Приведём примеры применения промышленно выпускаемых изотопов \cite{isotopes-use}.
%  АО «ПО «Элек­тро­хи­ми­чес­кий за­вод» вы­пус­ка­ет ста­биль­ные и ра­дио­ак­тив­ные изо­то­пы раз­лич­ных хи­ми­чес­ких эле­мен­тов,
С сайта производителя \ldots

Считается, что изотопный состав большинства химических элементов во всех материалах одинаков или варьируется в небольшом диапазоне известных значений. Однако, некоторые физические процессы в природе приводят к заметному изменению изотопного состава элементов, и обнаружение таких изменений используется в научных исследованиях. Более того, в технологической деятельности люди научились изменять изотопный состав химических элементов для получения каких-либо специфических свойств материалов. 

1. МЕТОД «МЕЧЕНЫХ» АТОМОВ

Метод «меченых» атомов — это использование стабильных и радиоактивных изотопов в качестве индикаторов для изучения распределения и путей перемещения вещества в разнообразных системах.

В биологии — исследование процессов биосинтеза, обмена веществ, изучение структуры и функций сложных биологических молекул.

В медицине — изучение динамики активности тех или иных органов, диагностика заболеваний, радиоимунный анализ, авторадиография, сцинтиграфия и т.д.

В сельском хозяйстве — изучение процессов фотосинтеза, изучение усвояемости удобрений и определение эффективности использования растениями азота, фосфора, калия, микроэлементов, водных ресурсов; определение солевого режима почв, исследование эффективности применения, переноса и динамики распада агрохимикатов и пестицидов.

В экологии — определение причин и масштаба загрязнения, исследование глобальных путей переноса, накопления и динамики распада загрязняющих веществ в воде, воздухе и почвах, исследование причин возникновения парникового эффекта.

В гидрогеологии — изучение подземных и поверхностных вод, определение механизмов подпитки подземных вод, получение данных о литологии, пористости и проницаемости водоносных горизонтов, динамика озёр и водоёмов, утечка через дамбы, измерение расхода рек, перенос донных и взвешенных отложений.

В физике — исследование кристаллизации веществ, структуры и однородности высокотемпературных сверхпроводников, тонких плёнок, изучение диффузии примесей в полупроводниках.

В химии — определение качественного и количественного состава веществ, определение растворимости, плотности насыщенных паров, коэффициентов диффузии, исследования в области химической кинетики и электрохимии, изучение гетерогенного катализа, изучение механизмов и кинетики органических реакций.

В промышленности — определение расхода материалов, скорости и длительности протекания технологических процессов, исследование процессов диффузии, сорбции, фазовых превращений, разделения смесей, изучение процессов смешивания, определение однородности смесей, времени смешивания и характеристик смесителей, обнаружение течей, исследование коррозии, износа, процессов смазки.

2. ИЗОТОПЫ КАК ИСТОЧНИКИ РАДИОАКТИВНОГО ИЗЛУЧЕНИЯ

Нестабильные изотопы используются как источники радиоактивного ($\alpha$-, $\beta$-, $\gamma$- и нейтронного) излучения, используемого для решения практических задач в различных областях науки и техники.

В биологии --- мутации живых организмов, индуцированные радиоактивным излучением.

В медицине --- диагностическое оборудование, терапевтическая ядерная медицина (телетерапия, брахитерапия), радиационная стерилизация медицинского оборудования.

В сельском хозяйстве --- разведение высококачественных, хорошо адаптированных к местным условиям и устойчивых к болезням сельскохозяйственных и садовых растений с помощью радиационно-индуцированных мутаций, борьба с вредными насекомыми путём их стерилизации или генетического изменения с помощью излучения, уменьшение потерь урожая с помощью облучения.

В геологии --- радиометрический полевой анализ, активационные методы поиска и разведки в геофизике.

В физике --- ядерно-физические эксперименты.

В технике --- методы измерений, основанные на измерении поглощения радиоактивного излучения (толщиномеры, измерители длины, измерители уровня), активационные методы (измерители плотности, влажности), активационное выявление взрывных устройств, гамма-радиография, гамма-дефектоскопия, нейтронная радиография, детекторы дыма, образцовые источники разных типов излучения для калибровки детекторов, радиоактивные ионизаторы среды для снятия статического электричества, светосоставы длительного действия, датчики уровня, толщины и др.

В ядерной энергетике --- радионуклидный термоэлектрический генератор (РИТЭГ). Действие РИТЭГ основано на преобразовании тепла, выделяемого при радиоактивном распаде, в электрическую энергию постоянного тока с помощью полупроводниковой термоэлектрической батареи. РИТЭГи применимы как источники энергии для автономных систем, удалённых от традиционных источников электроснабжения и нуждающихся в сравнительно небольшой электрической мощности (несколько десятков-сотен ватт) при очень длительном времени работы. РИТЭГи применялись в космических аппаратах, зондах (если условия их работы делали  неэффективным или невозможным использование солнечных батарей);  в навигационных маяках, радиомаяках, метеостанциях (в частности, обслуживавших Северный морской путь, СССР), в морских буях и подводных установках (США)..

В микроэлектронике — радиационный отжиг полупроводников.

3. ИНЫЕ ОБЛАСТИ ПРИМЕНЕНИЯ ИЗОТОПОВ

В биологии, физике, химии — исследования структуры и свойств веществ методами ядерного магнитного резонанса.

В физике — ядерно-физические эксперименты, мёссбауэровская спектроскопия на основе ядерного гамма-резонанса (ЯГР, или эффект Мёсбауэра).

В лазерной физике — сужение линий генерации лазеров и перестройка частоты их излучения.

В химии — нейтронно-активационный анализ, используемый для определения концентраций химических элементов в твёрдых телах, жидкостях, суспензиях, растворах и газах. После бомбардировки нейтронами в образце образуются элементы с радиоактивными изотопами, обладающими коротким периодом полураспада, по спектрам излучения которых и определяются концентрации элементов. Особое преимущество метода в том, что он не разрушает образец, а продолжительность наведенной радиации обычно составляет от нескольких наносекунд до часов. Метод часто используется для анализа произведений искусства и исторических артефактов.

В ядерной энергетике — топливо для атомных электростанций.

В реакторостроении — конструкционные материалы, замедлители и поглотители нейтронов (используются малые или, наоборот, большие сечения поглощения нейтронов теми или иными изотопами).

В производстве изотопов — получение других стабильных и радиоактивных изотопов в ядерных реакциях на ускорителях и реакторах.

В медицине — томография с использованием ядерного магнитного резонанса (ЯМР), позитронная томография (ПЭТ), магнитно-резонансная томография (МРТ).

\newpage
\section*{Заключение}
Итак, наличие изотопов, от 2 до 10, у элементов является не экзотикой, а нормой.  Этот факт был осознан во второй половине прошлого века. Постепенно понимание важности изотопного состава и применение изотопов в различных сферах человеческой деятельности привело и к осознанию расширения понимания периодического закона.

\chapter{Атомные веса элементов}

%Начнём представление интервалов в Природе с наиболее фундаментальной составляющей всего сущего --- с атомов.
Международный союз теоретической \index{Периодическая таблица Менделеева}
и прикладной химии IUPAC (International Union of Pure and Applied Chemistry) занимается, в частности, 
стандартизацией атомного веса элементов через один из старейших постоянных комитетов — Комиссии по изотопному и атомному весу (Commission on Isotopic Abundances and Atomic Weights, CIAAW) \cite{CIAAW}.
\index{The International Union of Pure and Applied Chemistry, IUPAC}
\index{Commission on Isotopic Abundances and Atomic Weights, CIAAW}


\section{Интервальные атомные веса в IUPAC}

% INORGANIC CHEMISTRY DIVISION Atomic weights of the elements 1969
% Pure Appl. Chem., 1970, Vol. 21, No. 1, pp. 91-108 
% http://dx.doi.org/10.1351/pac197021010091


В отчете 1969 г. \cite{AtomicWeights1969} Комиссия CIAAW впервые признала, что:\\
«Открытие того, что большинство химических элементов существует в природе в виде изотопных смесей,
из которых, как известно, различаются по составу, делает необходимым изменить историческую
понятие атомных весов как констант природы. Несмотря на то, что для некоторых элементов  [стабильные] изотопы имеют не найдены в природе (в настоящее время их 21), кажется логичнее считать, что \footnote{выделение автора}
\begin{center}
	изотопные смеси представляют собой нормальное, а не
	исключительное состояние элемента. 
\end{center} 

Комиссия считает, что такое отношение будет
способствовать осознанию того, что неопределенности в значениях, приведенных в Международной таблице
уже не следует, как в прежние времена, считаться следствием только ошибок в
измерении значений, но что они возникают из-за естественных вариаций изотопных
составов \ldots Чтобы получить рекомендуемое значение атомного веса,
Комиссия будет использовать процедуры определения весов, так чтобы значения были оптимизированы для
материалов в мировой науке, химической технологии и торговле, а не представлять расчетное среднее геохимическое значение».

Далее в документе \cite{AtomicWeights1969} приводятся аргументы в пользу использования симметричной формы представления атомного веса, хотя это и не соответствует фактической асиметрии распределения атомных масс изотопов элементов. Первый аргумент заключается в том, что большинство
компьютерных программ не смогут должным образом обрабатывать асимметричные неопределенности.

Также асимметричные неопределенности неудобны в связи по следующим причинам\\
1. Учащиеся и часто другие лица обычно неверно истолковывают значение неопределенности стандартного атомного вес как неопределенность измерения, и они задаются вопросом, почему стандартные атомные веса
точнее определить невозможно.\\
2. В связи с постоянным получением новых данных вариации изотопных распределений изменяются и, как правло расширяются. Таким образом, требуется регулярная коррекция стандартных атомных весов.\\
3. Обычно читатели ожидают, что стандартное значение атомного веса отражает
Гауссово распределение, что в болшинстве случаев неудовлетворительно отражает би- и мультимодальные распределение для ряда элементов, например, бор и сера.\\
4. Метрологические стандартные образцы, как правило, не соответствуют стандартной атомной массе. Например, не имеется водородосодержащего материала с атомным весом 1,00794.



С 2009 года атомные веса некоторых элементов в периодической системе химических 
элементов Д.И.\,Менделеева стали выражаться интервалами \cite{IUPAC}. Это событие 
стало итогом длительного, продолжительностью более полувека, процесса осознания 
химиками неизбежной и неустранимой изменчивости величины атомных масс элементов 
в зависимости от того, где и как взята их проба. С середины XX века вместе 
с развитием измерительной техники и экспериментальных методик постепенно стало ясно, 
что различие результатов измерений атомных масс в различных пробах веществ носит 
принципиальный характер. 

Дело в том, что почти каждый химический элемент представлен в природе смесью своих \index{изотоп}
изотопов, отличающиеся массой ядер. Относительная 
доля различных изотопов может существенно меняеться в зависимости от места и характера 
взятия пробы. Например, в тканях живых организмов преобладают более лёгкие изотопы 
химических элементов, нежели в неживой природе. Отличаются друг от друга 
относительные доли изотопов элементов на суше и в морях и т.\,п. --- подробнее см. \S\ref{s:IntNature}, \S\ref{s:IsotopesEarth}.


Первоначально в 2009 году интервалы атомных весов были назначены для 10 химических 
элементов, но далее в 2015,2017, 2019 и 2021 годах работа по <<интервализации>> продолжилась, 
и теперь в периодической таблице Д.И.\,Менделеева имеется 14 элементов, атомные веса 
которых выражаются интервалами. Среди них --- такие широко распространённые и важные 
элементы как водород, углерод, азот, кислород, кремний, сера и др. Интервалы дают 
двусторонние границы значений атомного веса для любой пробы <<нормального материала>>
включающего эти элементы. При этом особо подчёркивается \cite{IUPAC}, что внутри 
заданных интервалов не предполагается наличия какого-либо вероятностного распределения. 

На настоящий момент текущим документом является \cite{IUPAC2021}, фиксирующая результаты на 2021 год. Технические детали вычислений приведены в документе 2018 года \cite{IUPACUncertainty}.

Результатом работы Комииссии на текущий момент является введение обозначения $A_{r}{\circ}(E)$ для стандартного атомного веса элемента с тем, чтобы отличать его от атомного веса элемента $E$ в конкретном  веществе  $P$, что обозначается как $A_r(E, P)$. \index{$A_{r}{\circ}(E)$, стандартный атомный вес элемента} \index{$A_r(E, P)$, атомный вес элемента $E$ в веществе $P$}
Также изменены значения атомных масс для 5 элементов:
\begin{tabular}{llcl}
Ar (argon): & c $39.948 \pm 0.001$ & на & $[ 39.792, 39.963 ]$ \\
Hf (hafnium): & c $178.49 \pm 0.02$ & на & $178.486 \pm 0.006$ \\
Ir (iridium): & c $192.217 \pm 0.003$ & на & $192.217 \pm 0.002$ \\
Pb (lead): & c $207.2 \pm 0.1$ & на & $[ 206.14, 207.94 ]$ \\
Yb (ytterbium): & c $173.054 \pm 0.005$ & на & $173.045 \pm 0.010$
\end{tabular}

Можно заметить, что в для Hf и Ir неопределённость значений уменьшена, в случае Yb --- увеличена, а в случаях Ar и Pb введена интервальная неопределённость значений атомных весов.

Также произведено уточнение значений для следующих элементов: Al, Au,
Co, F, Ho, Mn, Nb, Pa, Pr, Rh, Sc, Tb, Tm, Y.

\section{Обозначения и определения IUPAC}




\subsection{Общие обозначения и определения.}


\emph{Атомная масса} $\mbf{m_a(^iE)}$ несвязанного нейтрального атома нуклида $^iE$ элемента $E$ с массовым числом $i$  определяется как <<масса покоя атома в его основном состоянии>> \cite{IUPACGoldBook}. Обычно используемой единицей является универсальная атомная единица массы  или Дальтон \cite{IUPACGoldBook}. \index{универсальная атомная единица массы, Дальтон} 

{\bf Атомный вес} (далее этот термин используется в тексте) или {\bf относительная атомная масса}, $A_r(^iE)$, {\bf атома} (нейтрального нуклида в свободном состоянии) $^iE $ элемента $E$ определяется как <<отношение массы атома к универсальной атомной единице массы>>. \index{атомный вес} \index{относительная атомная масса}
Атомная массовая постоянная $m_u$ равна дальтону (Да) или универсальной атомной единице массы $u$ и определяется через массу атома углерода-12: \index{атомная массовая постоянная}
\begin{equation}
	m_u = 1 u = 1 Da = m_a(^{12}C)/12.
\end{equation}
Таким образом, атомный вес есть безразмерная величина:
\begin{equation} \label{ArE}
	A_r(^iE) = m_a(^{i}E)/ \left[ m_a(^{12}C)/12 \right] 
\end{equation}

{\bf Атомный вес элемента} $E$ в веществе $P$, $A_r(E, P)$ это средневзвешенное значение атомных весов
$A_r(^iE)$ изотопов (нуклидов) $^iE$ этого элемента в веществе $P$: \index{атомный вес элемента $E$ в веществе $P$, $A_r(E, P)$}
\begin{equation}  \label{ArEP}
	A_r(E, P) = \sum \chi(^iE, P)A_r(E)
\end{equation}
Здесь $\chi(^iE, P)$ — количественная доля изотопа $^iE$ в веществе $P$ (также называемая изотопным составом), \index{изотопный состав}
а суммирование проводится по всем стабильным изотопам и радиоактивным изотопам, имеющим характерные земные \emph{изотопные подписи} \cite{IPTEI} (см. также \S\ref{s:IsotopeSignature}), \index{изотопная подпись} и они перечислены в Таблице изотопных составов элементов. 
Атомный вес элемента в данном веществе можно определить, зная атомные массы изотопов и соответствующие количественные доли изотопов этого элемента в этом конкретном веществе.

{\bf Cтандартный атомный вес элемента}, $A_{r}{\circ}(E)$, представляет собой <<рекомендуемое значение атомного веса (относительно атомная масса) элемента, пересматриваемого каждые два года комиссией CIAAW и применимого к элементам в любом обычном материале 
с высоким уровнем достоверности>> \cite{IUPACGoldBook}. Он состоит либо из интервала (в настоящее время используется для 14 элементов), либо из базового значения и неопределенности (стандартная неопределенность атомного веса), которые в настоящее время используются для 71 элемента. Стандартный атомный вес определяется на основе оценки рецензируемых научных
публикации. Стандартные атомные веса согласуются со значениями атомного веса, рассчитанными по
содержание изотопов указано в столбце 9 Таблицы изотопного состава элементов \cite{IPTEI}. \index{cтандартный атомный вес элемента, $A_{r}{\circ}(E)$}

На основании отчета Подкомитета по естественной оценке Комиссии по фундаментальному пониманию изотопов, {\bf нормальный материал} - это материал, происходящий из земного источника, который
удовлетворяет следующему определению: \\ \index{нормальный материал}
К нормальным материалам относятся все вещества, кроме \\
(1) тех, кто подвергся серьезному преднамеренной, нераскрытой или непреднамеренной искусственной изотопной модификации, \\
(2) внеземные материалы и \\
(3) изотопно-аномальные образцы, такие как продукты природного ядерного реактора Окло (Габон) или другие уникальные случаи.

В отличие от предыдущего определения (1968 год), это пересмотренное определение признает тот факт, что вариации атомного веса ряда элементов обусловлены процессами изотопного фракционирования, которые действуют на различных масштабах времени. Оно также вновь вводит исключение внеземных материалов из определения стандартного атомного веса. Новое определение более всеобъемлющее, чем более ранние версии, в отношении встречающихся в природе материалов с нуклеогенными и радиогенными изотопными вариациями, примерами которых являются аргон \cite{IUPACArgon} (\S\ref{s:Argon}) и свинец \cite{IUPACLead} (\S\ref{s:Lead}).

\subsection{Классификация элементов по их атомному весу и вариациям изотопного состава}

Поскольку изменение изотопного состава элемента влияет на его атомный вес -- \S\ref{s:IsotopeCounts}, Комиссия CIAAW
проводит периодическую оценку вариаций изотопных составов в опубликованной литературе, как
через свои подкомитеты и через проекты IUPAC \cite{IUPACMS2020}. Все известные элементы можно классифицировать по к следующим ограничениям на их стандартные атомные веса: \\
(1) Элементы без стабильного изотопа и с характерным земным изотопным составом в нормальных условиях.
материалы (например, радон). Стандартный атомный вес не может быть определен, и в
Таблица стандартных атомных весов этих элементов. Эти элементы имеют белый фон
записи в Периодической таблице элементов и изотопов ИЮПАК IPTEI \cite{IPTEI} \S\ref{s:IPTEI}. \index{нестабильные элементы} \\
(2) Элементы, стандартный атомный вес которых определяется только одним изотопом (например, натрий). Стандарт атомный вес получается из атомного веса его стабильного или долгоживущего изотопа (например, висмута или протактиний). Эти элементы имеют синий фон записей в таблице IPTEI \cite{IPTEI}. \index{моноизотопные элементы} \\
(3) Элементы, стандартный атомный вес которых определяется более чем одним изотопом, показаны на
Периодическая таблица элементов и изотопов ИЮПАК с желтым фоном записей \cite{IPTEI}. Они 
подразделяются на три подкатегории: \index{многоизотопные элементы} \\
а. Элементы, у которых нет документированных свидетельств изменения атомного веса для обычных материалов, или элементы, изотопный состав которых не оценивался в рамках проекта IUPAC (например,
индий). Элементы этой подкатегории могут попасть в категорию 3b при опубликовании  измерений в рецензируемых изданиях более точного изотопного состава. \\
б. Элементы, для которых известны вариации атомного веса в обычных материалах, но эти вариации
не превышают расчетную неопределенность измерения атомного веса, полученную из <<наилучшего
измерения>> содержания изотопов элемента (например, молибдена). Элементы в этой подкатегории
могут перейти в категорию 3c или 4 по мере улучшения результатов измерений. \\
в. Элементы, которые имеют известные изменения атомного веса в обычных материалах, которые превышают
неопределенность атомного веса, полученные из «наилучшего измерения» содержания изотопов, но
при этом еще не назначен стандартный интервал атомного веса (например, медь). Элементы в этой
подкатегории могут перейти в категорию 4 по мере того, как Комиссия завершит оценку и присвоит стандартные атомные весовые интервалы. Комиссия использует обозначение <<r>> для указания элементов в этой
подкатегории, для которых стандартная неопределенность атомного веса была \emph{расширена} для учета
известной изменчивости атомного веса.

(4) Элементы с двумя или более изотопами, имеющие известные вариации атомного веса в обычных материалах, которые превышают неопределенность атомного веса, полученную из «наилучшего измерения» содержания изотопов и имеющие верхние и нижние пределы атомного веса, определенные Комиссией на основе оцененных опубликованных данных (например, по водороду) --- \S\ref{MassIntervals}. Эти элементы имеют розовый фон для каждой ячейки элемента в таблице IPTEI \cite{IPTEI}. \index{атомные массы элементов, интервальные}

Комиссия использует обозначение <<g>> для обозначения химических элементов, для которых заявленные стандартные атомные вес и связанная с ним неопределенность не включают все известные вариации (см. определение термина <<нормальный материал>>).
Например, некоторые элементы аномально обогащены изотопами от деления ядер или нуклеогенными
изотопами на природном ядерном реакторе Окло в Габоне, Африка, и атомный вес этих материалов
не учитываются при определении стандартной атомной массы (исключение (3) в определении термина
<<нормальный материал>>)). 

Для элементов категорий 3 и 4 Комиссия использует обозначение <<m>> для обозначения тех
элементы, для которых стандартный атомный вес и связанная с ним неопределенность в коммерчески доступных
материал не включает вариации из-за нераскрытого или непреднамеренного изотопного фракционирования (исключение (1) в определение термина «нормальный» материал). Незначительные периодические изменения стандартных значений атомного веса и неопределенности являются результатом улучшенных измерений атомных масс, и эти изменения в первую очередь влияют на элементы категории 2. \index{нормальный материал}

\subsection{Стандартные интервалы атомных масс.} \label{MassIntervals}

Многие элементы можно найти на Земле в самых разных веществах с существенно разным генезисом. Как
следствие, атомный вес ряда элементов значительно различается в зависимости от происхождения и возраста этих веществ. В 2009 году Комиссия ввела интервальные обозначения (\S\ref{InteAnalysis}, \S\ref{InteDataAnalysis}, \cite{InteNotation}) для тех элементов, чей атомный вес \index{обозначения в интервальном анализе}
значительно различаются по своей природе, превышая погрешность измерения $A_r$ для элемента в конкретном веществе, и где такие вариации были хорошо задокументированы. Под <<хорошо документированным>> следует понимать
публикации в рецензируемых изданиях, специализирующихся в вопросах о естественных вариациях элементов.

Интервальные обозначения (\S\ref{InteDataAnalysis})
не меняют значения стандартного атомного веса и не является <<новым определением>> стандартного атомного веса.
Скорее, это альтернативный способ выражения неопределенности этой величины. Написание
стандартный атомный вес углерода как $A_{r}{\circ}(C) = [12,0096, 12,0116]$ указывает, что при текущем состоянии знаний
атомный углерода в любом нормальном материале будет больше или равен 12.0096 и будет меньше или равен
12.0116 [18]. Таким образом, стандартный атомный вес представлен интервалом, который охватывает (покрывает) атомные
массы обычных материалов --- \S\ref{CoverMeasrSect}.

Важно отметить, что никакое конкретное значение в интервале не должно рассматриваться как
как более представительный, чем любой другой (\S\ref{InteAnalysisBasic}), и что естественная изменчивость изотопного состава является
доминирующим источником неопределенности, выраженный интервалом --- \S\ref{MeasrConstChap}.

На сегодняшний день Комиссия предоставляет стандарт
атомный вес как интервал для 14 элементов: \\
\begin{center}
аргон, бор, бром, углерод, хлор, водород, свинец, литий,
магний, азот, кислород, кремний, сера, таллий \cite{IPTEI}.
\end{center}


\subsection{Измерения величины $\delta$ изотопов.} \label{DeltaIsotopes}

Обычно измерения изотопной дельты являются основой для определения атомного веса  \cite{IUPACMS2020}.
Величина $\delta$ изотопа получается из отношения числа изотопов $R(^{i/j}E)$ в веществе $P$:
\begin{equation} \label{REP}
	R(^{i/j}E, P) = N(^iE, P)/N(^jE, P) % (3)
\end{equation}
где $N(^iE, P)$ и $N(^jE, P)$ — число атомов каждого изотопа, а $^iE$ в общем случае обозначает наибольшее
(верхний индекс $i$) и $jE$ наименьшее (верхний индекс $j$) атомные массовые числа изотопов химического элемента $E$ в вещество $P$. $^jE$ представляет эталонный изотоп, который не обязательно является изотопом с с наименьшим атомным массовым числом. 

Дельта-значение изотопов (символ $\delta$), также называемое разностью относительных изотопных отношений, представляет собой дифференциальное измерение, полученное из соотношения изотопов вещества $P$ и шкалы, представленной опорным материалом \cite{IUPAC2014}. \index{параметр $\delta$ для изотопов} 
\begin{equation} \label{DeltaIsotopesDef} % (4)
	\delta_{\tt Ref}(^{i/j}E, P)  = R(^{i/j}E, P)/R(^{i/j}E, {\tt Ref}) - 1.
\end{equation}

Дельта-значения изотопов являются небольшими числами и поэтому часто представляются кратными $10^{-3}$ или промилле (символ $\permil$). \index{промилле, $\permil$}
Чтобы согласовать дельта-шкалу изотопов элемента со шкалой количеств изотопов, необходимо вещество,
содержание изотопов и дельта-значения изотопов которых также хорошо известны .

Обычно это вещество является изотопным эталонным материалом, который служит материалом <<лучшего качества>> (эталона) для определения содержания изотопов. Например, для углерода $\chi(^{13}C)$
шкала содержания согласуется с  $\delta_{\tt VPDB}(^{13/12}C)$ посредством измерения изотопного эталона
материал NBS 19 (карбонат кальция), которому было присвоено согласованное значение $\delta_{\tt VPDB}(^{13/12}C), {\tt NBS 19}) = +1.95 \permil$. 
Отношение числа изотопов углерода NBS 19 составляет
$R(^{13/12}C, {\tt NBS 19}) = 0.011 202 \pm 0.000 028$. \index{NBS 19, карбонат кальция}
Это измерение служит <<наилучшим измерением одного наземного источника \cite{IUPAC2016}. Белемнит Vienna Peedee (VPDB) является нулевой точкой на шкале дельта-изотопов углерода и, следовательно,
$  \delta_{\tt VPDB}(^{13/12}C, {\tt VPDB}) = 0$. Поскольку $1 \permil = 0.001$, отсюда следует: \index{VPDB, венский белемнит}
\begin{equation} % (5)
	R(^{13/12}C, {\tt VPDB}) = 0.011 202/(1 + 1.95 \times 0.001) = 0.011 180
	\end{equation}
Таким образом, без учета неопределенности соотношение между значениями дельты изотопов углерода ($\delta$) и $^{13}C$ составляет доли ($\chi$) материала P
\begin{equation} %(6)
	\chi(^{13}C, P) = 1/ \left[  1 + 1/ \left\lbrace  R (^{13/12}C, {\tt VPDB} ) \times \left[  1 + \delta_{\tt VPDB}(^{13}C, P) \right]  \right\rbrace \right]  
\end{equation}

Например, рассмотрим материал с наименьшим измеренным изотопным содержанием углерода-13, кроцетан \index{кроцетан, crocetane}
(2,6,11,15-тетраметилгексадекан), образующийся в холодных сипах восточно-Алеутской зоны субдукции. 
Кроцетан имеет минимальное опубликованное значение $\delta_{\tt VPDB}(^{13}C)$ $(- 130,3 \pm 0,3)\permil$(k = 1) \cite{Crocetane}. Изотопное содержание углерода-13 этого образца определяется с помощью \eqref{REP} и составляет $\chi(^{13}C) = 0,009 630 \pm 0,000 003$. Соответственно, изотопное содержание углерода-12 равно $\chi(^{12}C) = 1 - \chi(^{13}C)$. Атомный вес углерода в этом образце определяется
используя соотношение числа изотопов углерода {\tt VPDB}, рассчитанное в \eqref{ArEP}, содержание изотопов углерода-12
и углерод-13, а также значения атомного веса изотопов углерода-12 и углерода-13 ($A_r(^{12}C) = 12$ и
$A_r(^{13}C) = 13,003 354 835 \pm 0,000 000 002$). Для этого материала $A_r(C) = 12,009 662 \pm 0,000 003 (k = 1)$.

Если материал P является нормальным материалом с наименьшим атомным весом элемента E, то нижняя граница равна
$A_r(E, P) - U[A_r(E, P)]$, где $U[A_r(E, P)]$ — расширенная неопределенность, включающая в себя неопределенность
измерение дельта-значения материала P и неопределенность связи шкалы дельта-значения с изотопа
количество фракций и весы атомного веса. Последнее представляет собой неопределенность в связи дельта-шкалы изотопов с
шкала атомного веса.
Расширенная неопределенность $U$ получается путем умножения объединенной стандартной неопределенности величины $y$,
$u_c(y)$, по коэффициенту покрытия $k$,
\begin{equation}
	U = k \times u_c(y).
\end{equation}
Значение $k$ зависит от элемента и берётся не меньше 2.

\subsection{Модель неопределённости стандартных атомных весов}

Неопределенности стандартных атомных весов оцениваются Комиссией путем оценки соответствующих
опубликованная литература. Ожидается, что атомный вес любого элемента находится в пределах интервала, указанного
неопределенность стандартного атомного веса (или в пределах явного стандартного интервала атомного веса для 14
элементов) с большой уверенностью для нормальных материалов, которые были исследованы во время составления
данные. Значения являются результатом решения, принятого после рассмотрения соответствующих данных, которые
включает в себя качество измерений.
Интервалы, в которых значения атомных масс элементов
в нормальных материалах, либо задаются сообщаемыми значениями $A_{r}{\circ}(E)$ и их неопределенностями $U[ A_{r}{\circ}(E)]$
(таблица \ref{t:TSAW2021}, столбцы 4 и 5) или как явные интервалы (для 14 элементов см. таблицу  \ref{t:TSAW2021}, столбец 4).

С точки зрения анализа данных с интервальной неопределённостью \S\ref{UnceratintyMeasurements} тем самым задана модель погрешности 	\eqref{GeneralErrorModel}  с базовым значением в виде среднего $A_{r}{\circ}(E)$ и уравновешенным интервалом погрешности 
\index{базовое значение}
\index{интервал погрешности} 
\begin{equation*} 
	 A_{r}{\circ}(E) + [ \, -U[ A_{r}{\circ}(E)], \, U[ A_{r}{\circ}(E)] \, ].  
\end{equation*} 



Сообщаемые погрешности стандартных атомных весов $U[ A_{r}{\circ}(E)]]$ таковы, что атомный вес
ожидается, что значения нормальных материалов будут лежать между $A_{r}{\circ}(E)-U[ A_{r}{\circ}(E)]$ и $A_{r}{\circ}(E) +U[ A_{r}{\circ}(E)]$ с отличным
уверенность. Отсюда, например, стандартный атомный вес иридия $192.217 \pm 0.002$ указывает, что атомный
Ожидается, что значения веса иридия в обычных материалах будут равны или выше $192.215$ долларов США и равны или равны
ниже $192.219$.
\begin{equation*} 
 Ir: 	192.217+ [ -0.002, 0.002] =  [ \, 192.215, \,  192.219 \, ].  
\end{equation*} 

Равномерное распределение на этих интервалах (заданное либо как неопределенность $U[ A_{r}{\circ}(E)]$, либо как явное
интервал) можно использовать как простую и практически полезную модель сложной реальности \cite{IUPAC2018}. Интерпретация
стандартный атомный вес в терминах равномерного распределения, сосредоточенного либо на самом данном интервале, либо на
интервал, определяемый этой неопределенностью, служит для обобщения знаний о естественной изменчивости
значения атомного веса, игнорируя анатомию этой изменчивости среди огромного набора всех нормальных
материалы \cite{IUPAC2018}. Другие статистические модели могут быть более подходящими и могут дать лучшее представление об этом.
анатомия.

Благодаря усовершенствованию аналитического оборудования за последние три десятилетия документально подтвержденные изменения в
значения атомного веса некоторых элементов в нормальных материалах превышают неопределенность атомного веса
определяется по «наилучшему измерению». Эти элементы указаны с обозначением <<r>> в Таблице IPTEI (таблица \ref{t:TSAW2021}), чтобы указать, что диапазон изотопного состава нормального материала предотвращает
присвоение меньшего значения неопределенности стандартному атомному весу этих элементов (если только они не
заданные интервалы для их стандартных значений атомного веса).

\section{Таблица стандартных атомных весов} \label{s:TableStandardAtomicWeights}

Таблица стандартных атомных масс дана в порядке возрастания атомного номера (Таблица \ref{t:TSAW2021}). Таблица предназначена для применения ко всем обычным земным материалам с небольшими исключениями 
согласно сноскам. Стандартные атомные веса не применяются ни к внеземным материалам, ни к материалам
с намеренно измененным изотопным составом, за исключением лития, для которого искусственно $^6Li$-обедненные вещества были включены в определение его значения $A_{r}{\circ}(E)$. Стандартные атомные массы даны в виде одного значения с неопределенностями или в виде интервала (таблица \ref{t:TSAW2021}, столбцы 4 и 5). 

В ходе рассмотрения последнего отчета Комиссии было отмечено, что выражение неопределенности стандартные атомные веса не соответствовали доокументам  GUM <<Руководство по выражению неопределенности в
измерении>> \cite{GUM2011}.
Например, стандартный атомный вес иридия, равный $192.217$, с погрешностью
$\pm 0.002$, будет указано в таблице как $192.217(2)$. Однако этот формат не соответствует выражению
неопределенность в GUM, поскольку этот формат предполагает, что заявленная неопределенность является стандартной неопределенностью. Основанный на о работе Подкомитета по естественной оценке фундаментального понимания изотопов \cite{IUPACMS2020},
Комиссия выбрала формат, в котором значение неопределенности обозначается символом <<$\pm$>>, например,
стандартный атомный вес иридия теперь выражается как $ 192.217 \pm 0.002 $. В таблице стандартных атомных
Веса 2021 неопределенности заносятся в новую колонку или в виде интервалов. Кроме того, на основе
сотрудничество между Подкомитетом и Комиссией, новая сноска с символом двойной крестик $\ddag$
был добавлен в Таблицу стандартных атомных весов (Таблица \ref{t:TSAW2021}), чтобы подчеркнуть, что неопределенность атомного веса
является консенсусом, основанным на экспертном заключении.

Подробности и количество значащих цифр, указанные в Таблице стандартных атомных весов (Таблица \ref{t:TSAW2021})
во многих случаях превышает потребности пользователей, а в некоторых случаях для дальнейших расчетов требуется одно значение
с учетом интервальных элементов. Поэтому были опубликованы таблицы сокращенных стандартных атомных весов.
начиная с 1981 года с расчетом на то, что пересмотр этих сокращенных значений будет минимальным, если изменения в
соответствующие стандартные атомные веса должны стать необходимыми. Кроме того, таблица сокращена до четырех
значащие цифры были опубликованы вместе со стандартной таблицей атомного веса и таблицей с условным
значения атомного веса для интервальных элементов.
Чтобы предоставить более четкие рекомендации для удовлетворения потребностей торговли, образования, промышленности и
исследования, Комиссия решила на своих заседаниях в 2015, 2017 и 2019 годах представить единую таблицу, содержащую
как стандартные атомные веса, так и сокращенные стандартные атомные веса для общего использования. CIAAW
признает, что стандартные атомные веса могут содержать слишком много деталей. Для этого сокращенный атомный
веса указаны с точностью до пяти значащих цифр (таблица \ref{t:TSAW2021}, столбец 7), за исключением случаев, когда такая точность не может быть достигнута из-за
из-за изменчивости изотопного состава в нормальных материалах или из-за ограничений измерения
способность. Консервативное значение $+/-$ (таблица \ref{t:TSAW2021}, столбец 8) дается как упрощенная мера надежности
сокращенные значения.

В этот столбец (8) были включены <<условные атомные веса>> из предыдущих отчетов для интервальных элементов.
чтобы предоставить единую таблицу для дальнейшего использования. Предыдущие «условные атомные веса» могут быть
из этой колонки, за исключением водорода, который был $1.008$, и был расширен до $1.0080$ до
обеспечивают более низкое значение неопределенности в столбце 8. Столбцы 7 и 8 предоставляют одно значение атомного веса
(столбец 7), включая неопределенность (столбец 8) для всех элементов, имеющих стандартное значение атомного веса или
стандартный интервал атомных весов по следующим правилам: \\
i) Для элементов, которые не имеют стандартной атомной массы, выраженной в виде интервала, и имеют стандартную
значение атомного веса, выраженное пятью или менее значащими цифрами, значение в столбце 7 (сокращенное
стандартный атомный вес) соответствует значению в столбце 4. Неопределенность (столбец 8) соответствует
неопределенность в столбце 5 (таблица \ref{t:TSAW2021}). \\
ii) Для элементов, стандартная атомная масса которых не выражена в виде интервала и имеет стандартную
атомный вес с более чем 5 значащими цифрами, стандартное значение атомного веса сокращается до 5
значащие цифры в столбце 7. Значение консервативной неопределенности, указанное в столбце 8 для этих элементов
соответствует разрядному значению последней округленной и, следовательно, младшей значащей цифры значения в столбце 8 обеспечение достоверности сокращенных значений. \\
iii) Для 14 элементов (аргон, бор, бром, углерод, хлор, водород, свинец, литий, магний, азот,
кислород, кремний, сера и таллий) стандартный атомный вес указан как интервал атомного веса в
столбец 4 (таблица \ref{t:TSAW2021}). Для этих элементов в столбце 7 указано одно значение, заменяющее прежнее
условных атомных весов для образования, торговли и коммерции, опубликованных в <<Таблице условных
Атомные веса>> в предыдущих отчетах. Эти значения можно использовать, если одно значение атомного веса
требуется, а значащих цифр достаточно для дальнейшего использования. Значение, указанное в столбце 7
для этих элементов не обязательно соответствует середине интервалов (в случае Li, Mg, S, Ar
и Pb), и многие из них соответствуют значениям Ar часто используемых эталонных материалов (например, NIST SRM981 для свинца; аргон в тропосферном воздухе для Ar). Соответствующее консервативное значение $+/-$ указано в столбце 8, что соответствует наименьшему симметричному числу, чтобы покрыть стандартный атомный вес
интервал. Значимое число цифр после запятой сокращенных атомных весов соответствует
к значащим числам цифр после запятой этого значения. \index{NIST SRM981}

В Табл.~\ref{t:TSAW2021} приведена часть данных из \cite{IUPAC2021}.

Приведём некоторые пояснения к обозначениям в Табл.~\ref{t:TSAW2021}. Для использования более крупного шрифта опущен столбец (1), содержащий английское название элемента. Для согласованности с исходным текстом IUPAC, нумерация столбцов начинается с 2. 


В Табл.~\ref{t:TSAW2021} приведены данные для элементов с атомными номерами 1-26. Они включают все биогенные элементы. В этом диапазоне сожержатся 11 элементов с интервальными значениями атомных весов.
Далее приведён бром с атомным номером 35 как еще один элемент с интервальным значением атомного веса.

С существенным разрывом добавлена группа элементов  с атомными номерами 77-82. В их числе два элемента таллий (81) и свинец (82) с интервальными значениями атомных весов. Кроме того, иридий (78) упоминается в тексте для пояснения обозначений неопределённости атомных весов. Данные об изотопном составе ртути приводятся в качестве иллюстрации в \S\ref{s:IPTEI}. 

\begin{table}[h!]
	{\scriptsize 
		\begin{tabular}{ll}
			3 & Атомный вес в интервальной форме или среднее значение \\
			4 & Неопределённость \\	
			5 & Комментарий \\
			g & Известны геологические и биологические материалы, в которых элемент имеет  \\
			~ & изотопный состав за пределами нормального материала. Разница между атомным  \\
			~ & весом элемента в таких материалах и указанное в таблице может превышать указанную  \\
			~ & погрешность.\\
			m & Модифицированные изотопные составы можно найти в коммерчески доступных  \\
			~ & материалах, потому что материал был подвергнут неизвестному или непреднамеренному \\
			~ & изотопному фракционированию. \\ 
			~ & Могут возникнуть существенные отклонения атомного веса элемента от \\
			~ & указанного в таблице. \\
			r & Диапазон изотопного состава обычного земного вещества не позволяет \\
			~ & получить более точный  стандартный атомный вес; \\
			~ & табличное значение и неопределенность должны быть применимы к обычному материалу.\\
			6 & Атомный вес --- среднее значение \\
			7 & Неопределённость --- округлено до $0.001$
		\end{tabular}
	}
	\caption{Таблица стандартных атомных весов 2021 --- Обозначения столбцов}
	\label{t:TSAW2021Notes}
\end{table}

\begin{table}[h!]
	{\scriptsize 
		\begin{tabular}{ccccccc}
Символ  & Атомный &  Вес  & Неопреде- & Комм-й &  Вес  & Неопреде-  \\
~ & номер &  ~  & лённость &  ~  &  ~  & лённость  \\
\hline 
 2 & 3 & 4 & 5  & 6 & 7 & 8\\
\hline 
~ & ~ & ~ & ~ & ~  & ~ & ~\\
H & 1 & [1.000784, 1.00811] & ~ & m & 1.0080 & 0.0002\\ [1mm]
He & 2 & 4.002602 & 0.00002 &  g r & 4.0026 & 0.0001 \\ [1mm]
Li & 3 & [6.938, 6.997] & ~ & m  & 6.94 & 0.06 \\ [1mm]
Be & 4 & 9.0121831 & 0.0000005 & ~  & 9.0122 & 0.0001 \\ [1mm]
B & 5 & [10.806, 10.821] & ~ & m  & 10.81 & 0.02 \\ [1mm]
C & 6 & [12.0096, 12.0116] & ~ & ~  & 12.011 & 0.002 \\ [1mm]
N & 7 & [14.00643 14.00728] & ~ & m & 14.007 & 0.001 \\ [1mm]
O & 8 & [15.99903, 15.99977] & ~ & m  & 15.999 & 0.001 \\ [1mm]
F & 9 & 18.998403162 & 0.000000005 & ~  & 18.998 & 0.0001 \\ [1mm]
Ne & 10 & 20.1797 & 0.0006 & g m  & 20.180 & 0.0001 \\ [1mm]
Na & 11 & 22.98976928 & 0.00000002 & ~  & 22.990 & 0.0001 \\ [1mm]
Mg & 12 & [24.304, 24.307] & ~ & ~  & 24.305 & 0.002 \\ [1mm]
Al & 13 & 26.9815384 & 0.0000003 & ~  & 26.982 & 0.0001 \\ [1mm]
Si & 14 & [28.084, 28.086] & ~ & ~  & 28.085 & 0.001 \\ [1mm]
P & 15 & 30.973761998 & 0.000000005 & ~  & 30.974 & 0.0001 \\ [1mm]
S & 16 & [32.059, 32.076] & ~ & ~  & 32.06 & 0.02 \\ [1mm]
Cl & 17 & [35.446, 35.457] & ~ & m  & 35.45 & 0.01 \\ [1mm]
Ar & 18 & [39.792, 39.963] & ~ & ~  & 39.95 & 0.16 \\ [1mm]
K & 19 & 39.0983 & 0.001 & g  & 39.098 & 0.001 \\ [1mm]
Ca & 20 & 40.078 & 0.004 & g  & 40.078 & 0.004\\ [1mm]
Sc & 21 & 44.955907 & 0.000004 & ~  & 44.956 & 0.001\\ [1mm]
Ti & 22 & 47.867 & 0.001 & ~  & 47.867 & 0.001\\ [1mm]
V & 23 & 50.9415 & 0.0001 & ~  & 50.942 & 0.001\\ [1mm]
Cr & 24 & 51.9961 & 0.0006 & ~  & 51.996 & 0.001\\ [1mm]
Mn & 25 & 54.938043 & 0.000002 & ~  & 54.938 & 0.001\\ [1mm]
Fe & 26 & 55.845 & 0.002 & ~  & 55.845 & 0.002\\ [1mm]
$\ldots$ & $\ldots$ & $\ldots$ & $\ldots$ & $\ldots$  & $\ldots$ & $\ldots$\\ [1mm]
Br & 35 & [79.901, 79.907] & ~ & g  & 79.904 & 0.003\\ [1mm]
$\ldots$ & $\ldots$ & $\ldots$ & $\ldots$ & $\ldots$  & $\ldots$ & $\ldots$\\ [1mm]
Ir & 77 &  192.217 & 0.002 & ~  & 192.22 & 0.01\\ [1mm]
Pt & 78 & 195.084 & 0.009 & ~  & 195.08 & 0.02\\ [1mm]
Au & 79 & 196.966570 & 0.000004 & ~  & 196.97 & 0.01\\ [1mm]
Hg & 80 & 200.952 & 0.003 & ~  & 200.59 & 0.01\\ [1mm]
Tl & 81 & [204.382, 204.385] & ~ & ~  & 204.38 & 0.01\\ [1mm]
Pb & 82 & [206.14, 207.94] & ~ & ~  & 207.2 & 1.1\\ [1mm]
$\ldots$ & $\ldots$ & $\ldots$ & $\ldots$ & $\ldots$  & $\ldots$ & $\ldots$\\ [1mm]
Th & 90 & 232.0377 & 0.0004 & g  & 232.04 & 0.01\\ [1mm]
U & 92 & 238.02891 & 0.00003 & g~m  & 238.03 & 0.01\\ [1mm]
$\ldots$ & $\ldots$ & $\ldots$ & $\ldots$ & $\ldots$  & $\ldots$ & $\ldots$\\ [1mm]
			\hline 
			\end{tabular}
	}
	\caption{Таблица стандартных атомных весов 2021}
	\label{t:TSAW2021}
\end{table} 




\chapter{Периодическая таблица элементов и изотопов} \label{s:IPTEI}



В периодической таблице Менделеева, поддерживаемой Международным союзом теоретической \index{Периодическая таблица Менделеева}
и прикладной химии IUPAC приводятся интервальные границы стабильных изотопов химических \index{стабильные изотопы}
элементов. Например, для кислорода, имеющего 3 изотопа с атомными массами 16, 17 и 18 
на стр. 1858 статьи \cite{IUPAC} приводятся данные, часть которых представлена 
в Табл.~\ref{IUPACOxygen}. 

%%%%%%%%%%%%%%%%%%%%%%%%%%%%%%%%%%%%%%%%%%%%%%%%%%%%%%%%%%%%%%%%%%%%%%%%%%%%%%%%%%%%%%%%

\begin{table}[h!]
	\centering
	\caption{Стабильные изотопы кислорода.} 
	\medskip 
	\begin{tabular}{|c|c|}
		\hline
		Стабильный  изотоп & Молярная доля \\
		\hline 
		$^{16}O$ & [0.997 38, 0.997 76] \\
		$^{17}O$ & [0.000 367, 0.000 400] \\
		$^{18}O$ & [0.001 87, 0.002 22] \\			
		\hline
	\end{tabular}
	\label{IUPACOxygen}
\end{table} 

%%%%%%%%%%%%%%%%%%%%%%%%%%%%%%%%%%%%%%%%%%%%%%%%%%%%%%%%%%%%%%%%%%%%%%%%%%%%%%%%%%%%%%%%  

Компактное представление кислорода в таблице Менделеева выглядит следующим образом.

\begin{figure}[ht] 
	\centering\small
%	\unitlength=1mm
	\includegraphics[width=0.3\textwidth]{Figures/Oxygen.png}
%	\includegraphics[width=30mm]{Figures\Oxygen.png}
	\caption{Представление кислорода в таблице Менделеева.} 
	\label{f:Oxygen}
\end{figure}	
На рисунке \ref{f:Oxygen} дано наглядное представление о распространенности изотопов кислорода в природе в форме полярной диаграммы. В нижней строчке приведён интервал атомной массы.

Для каждого стабильного изотопа приведены границы, в пределах которых данный изотоп 
встречается в различных породах, атмосфере, водной среде в различных местах Земли. 
Подробные сведения приводятся на рисунках 4.8.1-4.8.3 из работы \cite{IUPAC}. 



%Описание того, как получаются там эти интервалы, представлено на %Рис.~\ref{f:HistAtom}. 
%%%%%%%%%%%%%%%%%%%%%%%%%%%%%%%%%%%%%%%%%%%%%%%%%%%%%%%%%%%%%%%%%%%%%%%%%%%%%%%%%%%%%%%%
%\begin{figure}[ht] \label{f:HistAtom}
%	\centering\small
%	\unitlength=1mm
%	\begin{picture}(80,68)
%	\put(0,0){\includegraphics[width=80mm]{Figures\DistrPlot.eps} 
%	\put(60,24){\mbox{гистограмма частот}} 
%	\end{picture}
%	\caption{Как образуется интервал атомных весов элемента.} 
%\end{figure} 
%%%%%%%%%%%%%%%%%%%%%%%%%%%%%%%%%%%%%%%%%%%%%%%%%%%%%%%%%%%%%%%%%%%%%%%%%%%%%%%%%%%%%%%%
%На Рис.\begin{figure}[ht] 	\centering\small
	%	\unitlength=1mm
%	\includegraphics[width=0.3\textwidth]{Figures/Lead.png}
%	\caption{Представление свинца в таблице Менделеева.} 
%	\label{f:Lead2}
%\end{figure}
%\ref{f:HistAtom} по оси абсцисс отложены массы изотопов, по оси ординат %--- их распространённость в природе.

Например, в случае ртути, известны изотопы с массовыми числами от 170 до 216 (количество протонов 80, нейтронов от 90 до 136).
Природная ртуть состоит из смеси 7 стабильных изотопов:

\begin{table}[h!]
	\centering
	\caption{Стабильные изотопы ртути.} 
	\medskip 
	\begin{tabular}{cc}
		Изотоп & Распространённость \\
		\hline
		$^{196}$Hg &  0,155 \% \\
		$^{198}$Hg & 10,04 \% \\
		$^{199}$Hg  & 16,94 \% \\
		$^{200}$Hg  & 23,14 \% \\
		$^{201}$Hg  &  13,17 \% \\
		$^{202}$Hg  &  29,74 \% \\
		$^{204}$Hg  &  6,82 \% \\
		\hline
	\end{tabular} 
	\label{Sulfur}
\end{table}		

Приведённые в таблице \ref{Sulfur} величины распространённости служат исходными данными для построения гистограммы частот. %, схематично представленной на Рис.$\vdots$\ref{f:HistAtom}.
Конкретно для атомов ртути этот рисунок показан на Рис.~\ref{f:HistHg}.

\begin{figure}[ht] 
	\centering\small
	\begin{tikzpicture}
	%\draw[help lines] (0,0) grid (11,5);
	\draw[->] (0,0) -- (0,4);
	\draw[->] (0,0) -- (10,0);
	\draw (-0.5, 1) node {10 \%};
	\draw (-0.5, 2) node {20 \%};
	\draw (-0.5, 3) node {30 \%};
	\draw[red] (0,0.0155) -- (1,0.0155);
	\draw[red] (1,0.0155) -- (1,0);
	\draw[red] (1,0) -- (2,0);
	\draw[red] (2,0) -- (2,1.004);
	\draw[red] (2,1.004) -- (3,1.004);
	\draw[red] (3,1.004) -- (3,1.694);
	\draw[red] (3,1.694) -- (4,1.694);
	\draw[red] (4,1.694) -- (4,2.314);
	\draw[red] (4,2.314) -- (5,2.314);
	\draw[red] (5,2.314) -- (5,2.314);
	\draw[red] (5,2.314) -- (5,1.317);
	\draw[red] (5,1.317) -- (6,1.317);
	\draw[red] (6,1.317) -- (6,2.974);
	\draw[red] (6,2.974) -- (7,2.974);
	\draw[red] (7,2.974) -- (7,0);
	\draw[red] (7,0) -- (8,0);
	\draw[red] (8,0) -- (8,0.682);
	\draw[red] (8,0.682) -- (9,0.682);
	\draw[red] (9,0.682) -- (9,0);
	\draw (0.6,-0.5) node {196};
	\draw (2.6,-0.5) node {198};
	\draw (4.6,-0.5) node {200};
	\draw (6.6,-0.5) node {202};
	\draw (8.6,-0.5) node {204};
	\draw (10,-0.5) node {Масса изотопа};
	\draw (2,4) node {Распространённость};
	\end{tikzpicture}
	\caption{Распространённость изотопов ртути на Земле.}
	\label{f:HistHg}
\end{figure} 

Относительно характера графика, представленного на Рис.~\ref{f:HistHg}, следует заметить следующее. Согласно современным представлениям, атомное ядро составляют протоны и нейтроны (нуклоны). Характер сил, действующих между ними, таков, что для лёгких ядер количества протонов и нейтронов примерно равны, с небольшим преобладанием последних. Число стабильных изотопов при этом невелико.  В ядрах тяжёлых элементов нейтронов существенно больше, чем протонов, и количество изотопов может достигать десятков, из которых стабильна небольшая часть. При этом количество стабильных  изотопов с чётным количеством нуклонов заметно превышает количество стабильных  изотопов с нечётным количеством нуклонов. Для энергетически выгодной конфигурации количества нуклонов существуют и другие закономерности, подобные принципу заполнению электронных оболочек атомов. В целом график распределения стабильных изотопов для данного химического элемента имеет неправильную форму с возможными <<пробелами>> внутри графика, что в случае ртути имеет место для изотопов с массами 197 и 203.

В публикации \cite{IUPAC} предложена расширенная версия периодической  системы химических элементов.
Авторы пишут:  \index{Периодическая таблица элементов и изотопов} \index{Periodic Table of the Elements and Isotopes --- IPTEI}
<<Периодическая таблица элементов и изотопов (Periodic Table of the Elements and Isotopes --- IPTEI) IUPAC (Международный союз теоретической и прикладной химии) была создана для ознакомления студентов, преподавателей и непрофессионалов с существованием и важностью изотопов химических элементов.>> Они также предлагают использовать её в качестве наглядного пособия, подобно таблице периодических элементов.

\begin{figure}[ht] 
	\centering\small
	\unitlength=1mm
	\includegraphics[width=80mm]{Figures/IPTEIabcd.png} 
	\caption{Способы иллюстрации атомных весов} 
	\label{f:IPTEIabcd}
\end{figure}
\begin{itemize}
	\item a. Element (chlorine)	whose standard atomic weight is not a constant of nature and is an interval.
	\item  b. Element (mercury) whose standard atomic weight is not a constant of nature and is not an interval.
	\item  c. Element (arsenic) whose standard atomic weight is a constant of nature because it has a single
	stable isotope. 
	\item d. Element (americium) that has no stable isotopes and thus no standard atomic weight.
\end{itemize}



В целом таблица Менделеева выглядит как показано на рисунке \ref{f:PeriodicTable} \cite{IUPAC}.
Легенда цветового поля каждого элемента дана в таблице.\\
\begin{table}
%{\small
{\footnotesize 
\begin{tabular}{ll}
	Цвет фона & Пояснение \\
	\hline	
	красный & элемент имеет два или более стабильных изотопов. \\
	~ &	Соотношения изотопов различны в различных распространённых  \\
	~ & материалах. Эти вариации надёжно определены, \\
	~ &  атомный вес указывается в виде интервала, в квадратных скобках; \\
	\hline
	жёлтый & элемент имеет два или более стабильных изотопов. \\
	~ & Соотношения изотопов различны в различных распространённых  \\
	~ & материалах. При этом невозможно дать надежные оценки нижних  \\
	~& и верхних границ изменений. \\
	~ & Атомный вес даётся с неопределённостью, которая включает  \\
	~ & ошибку измерений и неопределённость вариации изотопных отношений; \\
	\hline
	голубой& элемент имеет один стабильный изотоп.  \\
	~ & Атомный вес даётся с неопределённостью, которая включает\\ 
	~ & ошибку измерений. \\
	\hline	
	белый & элемент не имеет стабильных изотопов.	\\
	~ & в распространённых материалах не содержится в таких  \\
	~ & количествах, по которым можно дать оценку изотопных отношений.\\
	\hline
\end{tabular} 
\caption{Обозначения на рис. \ref{f:PeriodicTable}.}
}
\end{table}

\begin{figure}[ht] 
	\centering\small
	\unitlength=1mm
	\includegraphics[width=120mm]{Figures/PeriodicTable2016all.png} 
	\caption{Таблица Менделеева элементов и изотопов \cite{IUPAC}.} 
	\label{f:PeriodicTable}
\end{figure}

%\end{example} 


\chapter{Интервальный анализ и интервальная статистика}\label{InteAnalData}

В этой главе даются неоходимые сведения об интервальном анализе и интервальной статистике (иначе --- анализе данных с интервальной неопределённостью). 

\emph{Интервалом} вещественной оси называется множество всех чисел, 
расположенных между заданными числами, включая их сами. То есть,  \emph{интервал} --- отрезок вещественной оси.
Интервальный анализ --- это исчисление интервалов, т.\,е. 
наука о том, как оперировать с ними наподобие того, как мы делаем это с обычными 
числами, и затем использовать построенную технику для решения различных задач, где 
интервалы встречаются в виде данных, входных или промежуточных.  \index{интервал} 

Наиболее основательный источник для изучения этой области математики --- книга С.П.Шарого <<Конечномерный интервальный анализ>> \cite{SharyBook}.
С основами теории и учебными примерами можно ознакомиться по учебному пособию \cite{Bazhenov2020}.

\section{Интервальный анализ} \label{InteAnalysis}

Рассмотрим причины, мотивирующие на использование интервалов.  
Обратимся сначала к практическому опыту. Мы многое оцениваем <<сверху>> и  <<снизу>>. Например, можно оценить продолжительность работы: <<1-2 часа>>. Высота дерева составляет <<4-5 метров>>. На полке можно разместить <<10-15 книг>>.

Перейдем от бытовых оценок с более основательным.
Математические мотивации различных интервальных арифметик подробно рассмотрены в книге \cite{SharyBook}, а сейчас мы затронем некоторые математические понятия, которые нам пригодятся при обсуждении материала.


\subsection{Базовые понятия интервального анализа} \label{InteAnalysisBasic}

Забегая вперёд, введём понятие интервала.
\emph{Интервалом} вещественной оси $[a,b]$ называется множество всех чисел, расположенных между заданными числами  $a$ и $b$ включая их самих, т.е. \index{интервал}
\begin{equation*}
	[a,b]:= \{x\, \in \,\mathbb{R}\ |\ a \leq x \leq b \}.
\end{equation*}
При этом $a$ и $b$  называются концами интервала.

Интервальную величину принято обозначать жирным шрифом, например, $\mbf{x}$. Левую границу интервала подчёркивают снизу, а правую --- сверху. Границы берутся в квадратные скобки, что передаёт идею интервала  как отрезка вещественной оси.
\begin{equation*}
	\mbf{x} = [\ \un{\mbf{x}}, \ov{\mbf{x}} \ ].
\end{equation*}
Важнейшими характеристиками интервала являются его \emph{середина} (центр) 
\begin{equation*}
	\textstyle\index{середина интервала} 
	\m\mbf{a} = \frac{1}{2}(\ov{\mbf{a}} + \un{\mbf{a}}),
\end{equation*}
и его \emph{радиус} 
\begin{equation*}
	\textstyle\index{радиус интервала} 
	\r\mbf{a} = \frac{1}{2}(\ov{\mbf{a}} - \un{\mbf{a}}).
\end{equation*} 

Радиус интервала является мерой абсолютного рассеяния точек интервала. При описании относительной погрешности в интервальном анализе приходится использовать разные меры.   \index{относительная ширина интервала} 

Полезной характеристикой интервала является так называемый функционал Рачека $\chi$: \index{функционал Рачека} 
\begin{equation*} 
	\chi(\mbf{a}) = 
	\left\{ \ 
	\begin{array}{ll}
		\un{\mbf{a}}/\ov{\mbf{a}}, & \text{ если } \;\un{\mbf{a}}\leq\ov{\mbf{a}},\\[1mm] 
		\ov{\mbf{a}}/\un{\mbf{a}}, & \text{ иначе. } 
	\end{array}
	\right. 
\end{equation*} 
Он характеризует <<относительную узость>> интервала.

Множество всех интервалов из $\mbb{R}$ обозначается символом $\mbb{IR}$. \index{классическая интервальная арифметика}
Используемая система обозначений следует неформальному международному стандарту на обозначения 
в интервальном анализе, выработанному в 2002--2010 годах \cite{InteNotation}. \label{InteNotation}
\index{ обозначения в интервальном анализе}

Неформально можно сказать так: интервалы предназначены для величин, для которых существуют двусторонние ограничения. Можно также говорить об интервальных оценках.

\subsection{Особенности интервальной арифметики}

Важной особенностью интервальной арифметики является учёт неопределёности выполнения арифметических операций. В частности, при при последовательном выполнении сложения и вычитания получается не точно 0, а величина, содержащая 0:
\begin{equation*} 
	[1,2]-[1,2] = [-1,1] \ni 0.
\end{equation*}
Таким образом, производится двустороняя оценка величины результата прямого и обратного действия. Многократное повторение этой операции приводит к увеличению границ результата
\begin{equation*} 
	\sum_{i=1}^{n} \left(  [1,2]-[1,2] \right) = n \cdot [-1,1].
\end{equation*}
То есть, имеет место эффект нарастания <<снежного кома>>, или <<обёртывания>>. Такое свойство классической интервальной арифметики отражает факт <<внешнего>> оценивания множества решений задачи. \index{эффект <<обёртывания>>} \index{внешнее оценивание}

\subsection{Интервальная арифметика Каухера}

Помимо наиболее естественного понимания интервала как отрезка вещественной оси, существуют и более сложные конструкции. В частности, очень важна полная интервальная арифметика или арифметика Каухера. Она обобщает обычную интервальную арифметику на случай, когда у интервала есть <<направление>>. Именно, в этом случае концы интервала не обязательно упорядочены от меньшего к большему. Такое свойство даёт дополнительные возможности, которые мы обсудим далее. Обозначается такая арифметика символом $\mbb{KR}$. \index{интервальная арифметика Каухера}

Символически, можно представить соотношение арифметик следующим образом
$$ \mbb{R} \subseteq \mbb{IR} \subseteq \mbb{KR}. $$
Если концы интервалов совпадают, имеем обычную вещественную арифметику.

В арифметике Каухера содержательный смысл имеет операция замены порядка следования концов интервала, при которой получается интервал, дуальный исходному: \index{дуальный интервал}
\begin{equation*}
	\label{Dualization}
	\dual\mbf{a} := [\;\ov{\mbf{a}},\,\un{\mbf{a}}\;].
\end{equation*}
В частности, дуализация даёт возможность получать при интервальных операциях точечные значения, или \emph{внутреннюю оценку}:
\begin{equation*}
	[1,2] \ominus [1,2]=0.
\end{equation*}
Символ $\ominus$ соответствует так называемому \emph{алгебраическому вычитанию}. \index{алгебраическое вычитание} 

Аналогично для деления имеем внешние и внутренние оценки: \index{внутреннее оценивание}
\begin{equation*}
	[1,2] / [1,2]= [0.5, 4], \quad [1,2] \oslash [1,2]= 1
\end{equation*}

Таким образом, в арифметике Каухера имеются гибкие арифметические возможности  оценок: от самых строгих, рассчитанных на наихудший вариант, до локализующих результат. Последнее обстоятельство особенно важно при многократных проведениях однотипных операций и построения итерационных алгоритмов. 

Приведём правила умножения в $\mathbb{KR}$ в виде так называемой таблицы Кэли.
\begin {table}[h]
\begin{small}
	\begin{center}
		\begin{tabular}{l | c c c | c}
			& $\mbf{b} \in P $ & $\mbf{b} \in Z$ & $\mbf{b} \in -P$ & $\mbf{b} \in \mathrm{dual} \ Z $ \\	
			\hline	
			$\mbf{a} \in P$  & $ [\un{a}  \un{b}, \ov{a} \ov{b}] $ & $ [\ov{a}  \un{b}, \ov{a} \ov{b}] $ & $[ \ov{a} \un{b} ,\un{a}  \ov{b}] $ & $ [\un{a}  \un{b}, \un{a} \ov{b}] $\\
			$\mbf{a} \in Z$ & $ [\un{a}  \ov{b}, \ov{a} \ov{b}] $& $[min \{\un{a}  \ov{b}, \ov{a} \un{b} \}, max\{ \un{a}  \un{b}, \ov{a} \ov{b} \}] $ & $[ \ov{a} \un{b} , \un{a}  \un{b}] $ & 0 \\
			$\mbf{a} \in -P$ & $[\un{a}  \ov{b}, \ov{a} \un{b}] $& $[\ov{a}  \un{b}, \ov{a} \ov{b}] $ & $ [ \ov{a} \ov{b}, \un{a}  \un{b} ] $ & $ [ \ov{a} \ov{b}, \ov{a}  \un{b} ] $\\
			\hline	
			$\mbf{a} \in \mathrm{dual} \ Z $ & $ [\un{a}  \un{b}, \ov{a} \un{b}] $ & 0 & $[ \ov{a} \un{b} , \un{a}  \ov{b}] $ & $[max \{\un{a}  \un{b}, \ov{a} \ov{b} \}, min\{ \un{a}  \ov{b}, \ov{a} \un{b} \}] $
		\end{tabular} 
	\end{center}
\end{small}
\caption{Интервальное умножение в полной интервальной арифметике}  
\label{t:CayleyKR}
\end{table}

Замечательно, что данные в таблице \ref{t:CayleyKR} правила верны и для классической интервальной арифметики. Её область ограничена квадратом $3 \times 3$ и не включает неправильных интервалов самой нижней строчки и самого правого столбца.


Помимо уже упомянутого, в полной интервальной арифметике всегда имеет содержательный смысл обобщение операции пересечения интервалов, взятие \emph{минимума по включению}, обозначаемому как $\wedge$. \index{минимум по включению}

Продемонстрируем этот факт на примере. Найдем пересечение двух пересекающихся интервалов [1, 3] и [2,4]:
\begin{equation*}
[1, 3] \cap [2,4] = \left\lbrace \max \min \{1,2\}, \min \max \{3,4 \}  \right\rbrace   = [2,3].
\end{equation*}
Поступим аналогичным образом с непересекающимися интервалами [1, 2] и [3,4], взяв минимум по включению:
\begin{equation*}
[1, 2] \wedge [3,4] = \left\lbrace \max \min \{1,3\}, \min \max \{2,4 \}  \right\rbrace   = [3,2].
\end{equation*}
В классической интервальной арифметике этот результат не имеет смысла, а полной имеет: это минимальный интервал, имеющий общие элементы с исходными.

Такая возможность даёт гибкость при неизбежной в экспериментальной практике работе с несовместными данными. \emph{Минимаксный подход}, свойственный полной интервальной арифметике, также обеспечивает и другие важные свойства, см. \cite{SharyBook}. \index{минимаксный подход}

%\begin{eqnarray*} \left[ \min_{x \in \mbf{x} } \max_{y \in \mbf{y} } f(x,y),  \max_{x \in \mbf{x} } \min_{y \in \mbf{y} } f(x,y) 	\right] \subseteq f(\mbf{x}, \text{dual} \ \mbf{y}), \\ 	f(\mbf{x}, \text{dual} \ \mbf{y})  \subseteq 	\left[ \max_{x \in \mbf{x} } \min_{y \in \mbf{y} } f(x,y),  \min_{x \in \mbf{x} } \max_{y \in \mbf{y} } f(x,y)	\right]. \end{eqnarray*}

\subsection{Исторические сведения}

В завершение неформального введения приведём некоторые исторические сведения.

Неолитические световые сооружения и культовые сооружения Древнего мира учитывают непостоянство положения Солнца в течение года, с тем, чтобы солнечный свет попадал на выбранные в качестве <<алтарей>> объекты \cite{OpticsHistory}.  

Античные учёные оставили нам примеры интервальных оценок. 
Аристарх Самосский, автор первой исторически известной гелиоцентрической системы, в сочинении «О величинах и расстояниях Солнца и Луны» \cite{Veselovcky1961} оценил, что отношение радиусов Солнца и Земли составляет больше чем 19 к 3, но меньше, чем 43 к 6. Эту оценку в современных обозначениях можно выразить так: \index{Аристарх Самосский}
\begin{equation*}
\frac{R_{\odot}}{R_{\text{Земли}}} = \left[  \frac{38}{6}, \frac{43}{6}\right].
\end{equation*}
Численно оценка Аристарха очень неточна, но метод её определения и представления правильны.

Трудно переоценить результат Архимеда об определении отношения длины окружности к периметру. Он использовал для оценки отношения длины окружности к диаметру периметры вписанных и описанных 96-угольников \cite{Archimedes}. Результат из его работы "$K \acute{u} \kappa \lambda o \upsilon \ \  \mu \acute{\varepsilon} \tau \rho \eta \sigma \iota \zeta$ (Измерение окружности)" : \index{Архимед}

\begin{equation*}
3\tfrac{10}{71} \leq 
\dfrac{\pi \varepsilon \rho \acute{\iota} \mu. \kappa \acute{u} \kappa \lambda o \upsilon}{\delta \iota \acute{\alpha} \mu \varepsilon \tau \rho o} 
\leq 3\tfrac{1}{7}.
\end{equation*}
%Κύκλου μέτρησις

В сочинении <<Псаммит>> (<<Исчисление песчинок>>) Архимед приводит двустороннюю оценку углового размера Солнца и обсуждает способ получения этой оценки.

Таким образом, мы имеем свидетельства применения интервальных величин, которым более 2 тысяч лет.

О предтечах интервального анализа в XIX веке и о зарождении и развитии его как самостоятельной математической дисциплины достаточно подробно написано в Главе 1 книги С.П.Шарого <<Конечномерный интервальный анализ>> \cite{SharyBook}.

\subsection{Мультиинтервалы.}\label{s:MultiIntervals}

\paragraph{Неодносвязные интервальные величины (мультиинтервалы).}

В ряде разделов науки и техники имеют место ситуации, когда исследуемая величина  содержится в неодносвязной области. %В  интервальном анализе такая ситуация описывается объектами, которые называются твинами (от англ. фразы TWice INterval, <<двойной интервал>>).

Согласно \cite{SharyBook}, мультиинтервал --- это объединение конечного числа несвязных интервалов числовой оси (Рис. \ref{f:MultiInterval} ). \index{мультиинтервал}

\begin{figure}[ht]
\centering
\includegraphics[width=0.7\textwidth]{Figures/MultiInterval.png}
%	\includegraphics[width=0.8\textwidth]{7.mps}	
\caption{Мультиинтервал в $\mathbb{R}$. Рис. 1.11 из \cite{SharyBook}.}
\label{f:MultiInterval} 
\end{figure}

Между мультиинтервалами также могут быть определены арифметические операции <<по представителям>>, аналогично тому, как это делается на множестве интервалов \cite{Iakovlev1968}.
Мультиинтервальная арифметика применяется редко ввиду серъёзных ограничений, которые возникают при алгебраических операциях с мультиинтервальными величинами и вычислительных сложностей. Тем не менее, сама по себе идея мультиинтервалов содержательна и полностью отметать её не стоит. 

В естественных науках возникновение неодносвязных областей часто связано с наличием периодичности в уравнениях или граничных условиях. Спектр таких явлений достаточно широк.

\subsection{Твины.} \label{s:Twin}

На практике концы интервалов, представляющие результаты измерений, сами могут быть 
известны неточно, так что возникает необходимость работы с интервалами, имеющими  
интервальные концы. Такие объекты известны в интервальном анализе и называются 
\textit{твинами} (по английски twin, как сокращение фразы \un{tw}ice \un{in}terval, 
<<двойной интервал>>).\index{твин} Они были введены в научный оборот в начале 
80-х годов XX века в работах испанских исследователей, и заинтересованный читатель 
может найти подробности в книге \cite{ModalIABook}. Отдельные аспекты применения 
твинов рассматриваются в статье \cite{Twins1981}. Краткое изложении основ теории 
твинов дано в статье \cite{Nesterov1997}, а развёрнутое --- в диссертации 
\cite{Nesterov1999}.  \index{твин в форме Нестерова}

Tвин, как <<интервал интервалов>>  или интервал с интервальными концами, можно 
представить как 
\begin{equation} 
\label{Twin}
\mbf{X} = 
[\mbf{a}, \mbf{b}] = \bigl[\,[\un{\mbf{a}}, \ov{\mbf{a}}], [\un{\mbf{b}}, \ov{\mbf{b}}]\,\bigr].
\end{equation}

\begin{figure}[hbt]
\centering\small 
\setlength{\unitlength}{1mm}
\begin{picture}(70,17)
	\put(0,0){\includegraphics[width=70mm]{Figures/twinfig.eps}}
	\put(-5,6.6){\vector(1,0){80}} \put(71.5,7.6){$\mbb{R}$} 
	\put(21,10){$\un{\mbf{a}}$} \put(30,10){$\ov{\mbf{a}}$} 
	\put(41,10){$\un{\mbf{b}}$} \put(50,10){$\ov{\mbf{b}}$} 
	\put(35,1){$\mbf{X}$}  
\end{picture}
\caption{Твины на вещественной оси.} 
\label{TwinsPic2} 
\end{figure}
На рисунке \ref{TwinsPic2} твин $\mbf{X}$ представлен в графической форме. Концы твина, интервалы $\mbf{a}$ и $\mbf{b}$, даны более тёмной заливкой, чем остальная часть твина.


Твин является множеством всех интервалов, больших или равных $[\un{a}, \ov{a}]$ и меньших
или равных $[\un{b}, \ov{b}]$, и точное определение зависит от смысла, который вкладывается в понятия <<больше или равно>>, <<меньше или равно>>.
Поскольку интервалы могут быть упорядочены различными способами, то существуют 
различные виды твинов. Двум основным частичным порядкам на $\mathbb{IR}$ и $\mathbb{KR}$, 
<<$\subseteq$ >> и <<$\leq$>>,  соответствуют два основных типа твинов. 

\begin{equation}\label{TwinPresentations}
\mbf{A}_{\subseteq} = \bigl[\,[2,3], [1,4 ]\,\bigr], 
\hspace{18mm} 
\mbf{A}_{\leq} = \bigl[\,[1,2], [3,4 ]\,\bigr].
\end{equation}

Разработаны 
различные операции с твинами, а также способы оценок значений функций от них. 


\section{Анализ данных с интервальной неопределённостью} \label{InteDataAnalysis}

В этой главе мы приведём ряд примеров, которые мотивируют применение интервальных подходов при анализе данных. Математически корректное обсуждение многих вопросов <<интервальной статистики>> проводится в книге \cite{BookIntStat}. \index{интервальная статистика}

\subsection{Измерения и их результаты} 
\label{MeasuResultSect} 

Основным понятием теории обработки наблюдений является понятие \emph{измерения} (\emph{наблюдения}). Слово <<измерение>> имеет много значений. Оно может 
обозначать как процесс измерения или наблюдения, так и его результат. Из контекста обычно 
бывает ясно, какое значение слова имеется в виду \cite{BookIntStat}.  
\begin{definition}       
\textsl{Измерением (замером, наблюдением)} будем называть измеренное значение величины. 
\index{измерение}\index{замер} 
\end{definition}
По способу получения результата измерения все процессы измерения разделяются 
на \emph{прямые}, \emph{косвенные} и \emph{совокупные}. 

При прямых измерениях объект исследования приводят в непосредственное взаимодействие 
со средством измерений, которое выдает результат. \index{измерения прямые}  
При косвенных измерениях значение измеряемой величины находят на основании \index{измерения косвенные} 
известной зависимости между измеряемой величиной и искомой величинами. 

При совокупных измерениях значения искомых величин определяются из системы (совокупности)  \index{измерения совокупные} уравнений. 


Приведенная классификация условна. Следует отметить, что результат измерения является \emph{итогом} какого-либо физического эксперимента, в котором находятся первичные измерения, и \emph{последующего применения некоторого способа	математической обработки} первичных измерений. 

На практике измерение (замер, наблюдение) может представлять собой вещественное число или интервал, или же составленные из них многомерные объекты (вектор, матрицу, интервальный вектор, 
интервальную матрицу и т.\,п.). 
Вещественный тип данных для измерений является традиционным. 


\subsection{Погрешность измерений}  
\label{UnceratintyMeasurements}

Интервалы в результатах измерений могут возникать в результате коррекции исходных 
точечных результатов. 

Один из распространённых способов получения интервальных результатов в первичных 
измерениях --- это <<обинтерваливание>> точечных значений, когда к точечному 
\emph{базовому значению} $\mathring{x}$, которое считывается по показаниям 
измерительного прибора, прибавляется \emph{интервал погрешности} $\mbf{\epsilon}$: 
\begin{equation} 
\label{GeneralErrorModel} 
\index{базовое значение}
\index{интервал погрешности} 
\mbf{x} = \mathring{x} + \mbf{\epsilon}.  
\end{equation} 
Интервал погрешности, вообще говоря, может быть произвольным, но если он уравновешен, 
т.\,е. 
\begin{equation*} 
\mbf{\epsilon} = [-\epsilon, \epsilon] \quad\text{ для некоторого } \epsilon > 0, 
\end{equation*} 
то иногда для прямых измерений это можно трактовать, как отсутствие систематических 
погрешностей. 

\subsection[Накрывающие и ненакрывающие измерения]% 
{Накрывающие и ненакрывающие \\* измерения} 
\label{CoverMeasrSect} 

Результат измерения интересующей нас величины может быть либо равным, либо не равным ее истинному значению.
В случае измерения непрерывных физических величин, принадлежащих вещественному типу данных, равенство 
является исключительным событием,  неустойчивым к сколь угодно малым возмущениям или погрешностям в вычислительных алгоритмах. 

Принципиально другая ситуация возникает, если результат измерения может быть интервалом. 
Невырожденный интервал по своей сути является  представительным множеством на  
вещественной оси (имеющим ненулевую меру), и оно, как правило, устойчиво к малым возмущениям и 
погрешностям вычислений. Для обработки интервальных данных 
фундаментальный характер имеет следующее определение (\cite{MetodikaBook, Enclosing2022}):
\begin{definition}
\textsl{Накрывающее измерение} (накрывающий замер) --- это интервальный результат 
измерения, который гарантированно содержит истинное значение измеряемой величины. 
Измерение, для которого нельзя утверждать, что оно содержат истинное значение 
измеряемой величины, будем называть \textsl{ненакрывающим} (рис.~\ref{PCoverMeasurPic} 
и \ref{ICoverMeasurPic}). \index{накрывающее измерение} 
\index{ненакрывающее измерение} 
\end{definition}

Отметим, что с точки зрения формальной логики понятия накрывающего и ненакрывающего 
измерений являются противоположными, но при этом не противоречащими друг другу. 

%%%%%%%%%%%%%%%%%%%%%%%%%%%%%%%%%%%%%%%%%%%%%%%%%%%%%%%%%%%%%%%%%%%%%%%%%%%%%%%%%%%%%

\begin{figure}[!ht]
\unitlength=1mm
\centering\small 
\begin{picture}(110,27) 
	\put(0,7){\includegraphics[width=60mm]{Figures/Intefig.eps}} 
	\put(62,7){\includegraphics[width=60mm]{Figures/Intefig.eps}} 
	\put(4,12.5){\vector(1,0){40}}
	\put(65,12.5){\vector(1,0){40}}
	\linethickness{.5mm}
	\put(25,9){\color{red}\line(0,1){13}}
	\put(95,9){\color{red}\line(0,1){13}}
	\put(72,5){\mbox{\small\begin{tabular}{c}Интервал \\[-1pt] измерения\end{tabular}}}
	\put(11,5){\mbox{\small\begin{tabular}{c}Интервал \\[-1pt] измерения\end{tabular}}} 
	\put(42.4,14.5){\mbox{$\mbb{R}$}}  \put(104.4,14.5){\mbox{$\mbb{R}$}}        
	\put(14,24){\small\mbox{Истинное значение}}    
	\put(82,24){\small\mbox{Истинное значение}}    
\end{picture}
\caption{Накрывающее (слева) и ненакрывающее (справа)	измерения точечного истинного значения величины }
\label{PCoverMeasurPic} 
\end{figure} 

%%%%%%%%%%%%%%%%%%%%%%%%%%%%%%%%%%%%%%%%%%%%%%%%%%%%%%%%%%%%%%%%%%%%%%%%%%%%%%%%%%%%%%  

Накрывающее измерение является гарантированной двусторонней вилкой значений 
измеряемой величины, тогда как для ненакрывающего измерения  подобное утверждать 
нельзя. При перенесении свойства накрытия истинного значения на выборки простейший путь --- объявить накрывающей выборкой совокупность накрывающих измерений, тогда как выборки, в которых присутствует хотя бы одно ненакрывающее измерение, станут ненакрывающими. 
Погрешности и выбросы (промахи) неотъемлемо присутствуют в данных, и проверка свойства <<накрытия истинного значения>> является нетривиальной. 

%%%%%%%%%%%%%%%%%%%%%%%%%%%%%%%%%%%%%%%%%%%%%%%%%%%%%%%%%%%%%%%%%%%%%%%%%%%%%%%%%%%%%

\begin{figure}[!ht]
\unitlength=1mm
\centering\small 
\begin{picture}(110,27) 
	\put(0,7){\includegraphics[width=60mm]{Figures/InteInside.eps}} 
	\put(60,7){\includegraphics[width=60mm]{Figures/InteInsidePart.eps}} 
	\put(4,12.5){\vector(1,0){40}}
	\put(65,12.5){\vector(1,0){40}}
	\linethickness{.1mm}
	\put(29,15){\line(0,1){8}}
	\put(92,15){\line(0,1){8}}
	\put(78,5){\mbox{\small\begin{tabular}{c}Интервал \\[-1pt] измерения\end{tabular}}}
	\put(18,5){\mbox{\small\begin{tabular}{c}Интервал \\[-1pt] измерения\end{tabular}}} 
	\put(42.4,14.5){\mbox{$\mbb{R}$}}  \put(104.4,14.5){\mbox{$\mbb{R}$}}        
	\put(14,24){\small\mbox{Истинное значение}}    
	\put(72,24){\small\mbox{Истинное значение}}    
\end{picture}
\caption{Накрывающее (слева) и ненакрывающее (справа)  	измерения
	интервального истинного значения величины }
\label{ICoverMeasurPic} 
\end{figure} 
%%%%%%%%%%%%%%%%%%%%%%%%%%%%%%%%%%%%%%%%%%%%%%%%%%%%%%%%%%%%%%%%%%%%%%%%%%%%%%%%%%%%%
Далее мы будем называть \textit{накрывающей выборкой}\index{накрывающая выборка} 
совокупность измерений, в которой доминирующая часть (большинство и т.\,п.) 
измерений (наблюдений) являются накрывающими. \index{включающее  измерение} 
\index{охватывающее измерение}

Напротив, выборка называется \textsl{ненакрывающей},\index{ненакрывающая выборка} если преобладающая часть входящих в нее измерений ненакрывающие. 

Данное определение нестрогое и использует расплывчатые понятия <<большинство>>, 
<<доминирующая часть>> и т.\,п., которые должны уточняться каждый раз в процессе 
применения. 


\subsection{Принцип соответствия} 
\label{CorresPrincpSect}

В методологии науки \textit{принципом соответствия} называют утверждение, что любая 
новая научная теория должна включать старую теорию и ее результаты как частный 
предельный случай. 	Далее будем использовать принцип соответствия как инструмент проверки адекватности используемых конструкций, понятий и методов обработки данных с интервальными 
неопределенностями, который позволяет исключать заведомо ненужные.
\index{принцип соответствия}

\subsection{Выбросы и промахи} 
\label{OutlierSect}

\textit{Выбросами} (или \textit{промахами}) в метрологии называются такие измерения, 
результаты которых не привносят информацию об исследуемом объекте в рамках его 
принятой модели. \index{выброс}\index{промах}

Другое определение выбросов (промахов) состоит в том, что это результаты 
измерений, которые в данных условиях резко отличаются от остальных результатов общей 
выборки. Выбросы нарушают некоторую однородность (согласованность, непротиворечивость), 
характерную для большинства наблюдений выборки по отношению к заданной математической 
модели. 
Оба приведенных определения неформальны, так как, одно формальное 
определение для данного важнейного понятия дать нельзя.  

Как правило, выбросы стремятся удалить из выборки на этапе ее предварительной 
обработки (предобработки), т.\,е. перед применением \index{предобработка} формальных 
математических методов, так как присутствие выбросов существенно искажает оценки 
истинных значений параметров. Выявление выбросов является нетривиальной и, как правило, 
трудноформализуемой процедурой, которая опирается на опыт и т.\,п. Для вероятностной 
статистики выявление выбросов является необходимой составной частью обработки данных, 
а некоторые процедуры даже рекомендованы в стандартах. 

\emph{Что считать выбросом} в случае интервальных результатов измерений? 
Из положения, что интервальное измерение не является 
накрывающим, не следует, что оно представляет выброс (промах). 
Отождествление выбросов (промахов) со свойством ненакрывания противоречит 
принципу соответствия, сформулированному \S\ref{CorresPrincpSect}. При стремлении ширины интервальных измерений к нулю они переходят в точечные 
измерения, которые, как правило, всегда ненакрывающие. 

Если априори известно, что измерение, производимое данным инструментом с помощью
некоторой определенной методики должно быть накрывающим, то получение 
ненакрывающего результата является признаком выброса (промаха). 

\subsection{Выборки интервальных данных} 
\label{InteSampleSect} 

Пусть имеется выборка измерений некоторой величины
\begin{equation}
\label{ISample} 
\mbf{x}_{1}, \mbf{x}_{2}, \ \ldots, \ \mbf{x}_{n}, 
\end{equation}                                 
или кратко $\{\,\mbf{x}_{k}\}_{k \, = \, 1}^n$, где $k$ --- номер измерения; $\mbf{x}_k$ 
--- интервальный результат измерения, полученный, к примеру, с помощью какой-либо из процедур, 
описанных ранее. Таким образом, согласно терминологии интервального 
анализа рассматриваемая выборка --- это вектор интервалов, или интервальный вектор 
$\mbf{x} = (\mbf{x}_{1}, \mbf{x}_{2}, \ \ldots, \ \mbf{x}_{n})$. Число $n$ --- размерность 
вектора данных --- будем называть \emph{длиной выборки} (или объемом 
выборки).\index{длина выборки} По интервальным результатам измерений или наблюдений 
требуется найти оценку  для интересующей нас величины. 
\index{задача измерения постоянной величины} 

Для наглядного представления выборки часто чертят образующие ее интервалы в виде 
графика, изображенного на рис.~\ref{ScatPlotPic}, который по статистической традиции 
называют \emph{диаграммой рассеяния}. 
Можно повернуть картинку и представлять \index{диаграмма рассеяния} 
интервалы данных горизонтально. 

%%%%%%%%%%%%%%%%%%%%%%%%%%%%%%%%%%%%%%%%%%%%%%%%%%%%%%%%%%%%%%%%%%%%%%%%%%%%%%%%%%%%%

\begin{figure}[htb]
\centering\small 
\unitlength=1mm
\begin{picture}(100,60)
	\put(0,0){\includegraphics[width=80mm]{Figures/PgammaPhNoTickLabels.png}}
	\put(5,55){$\mbf{x}$}	
	\put(13,3){$1$} 
	\put(25,3){\ldots}
	\put(40,3){$k$} 
	\put(55,3){\ldots}
	\put(68,3){$n$} 
	\put(64,10){\mbox{\small Номер измерения}} 
\end{picture}
\caption{Диаграмма рассеяния интервальных измерений постоянной величины}
\label{ScatPlotPic} 
\end{figure} 
%%%%%%%%%%%%%%%%%%%%%%%%%%%%%%%%%%%%%%%%%%%%%%%%%%%%%%%%%%%%%%%%%%%%%%%%%%%%%%%%%%%%%%  

Значения $\r\mbf{x}_k$, $k = 1,2,\ldots,n$ показывают величины интервальной 
неопределенности отдельных измерений выборки. Величину 
неопределенности всей выборки характеризует вектор радиусов 
\begin{equation*}
\r\mbf{x} = (\r\mbf{x}_{1}, \r\mbf{x}_{2}, \ \ldots, \ \r\mbf{x}_{n}).
\end{equation*} 
Часто такая детальность не требуется  в представлении неопределенности выборки, 
а нужна какая-либо одна величина, которая агрегированным образом представляет данную 
неопределенность. В этом случае можно взять какую-либо норму вектора $\r\mbf{x}$.

По аналогии с традиционной метрологией будем называть измерения выборки 
\textit{равноширинными}, если неопределенность всех этих измерений одинакова, т.\,е. 
$\r\mbf{x}_k = r = \const$, $k = 1, \ \ldots, \ n$. Напротив, \textit{неравноширинными} 
(разноширинными) называем измерения, в которых величина неопределенности $\r\mbf{x}_k$ 
может меняться в зависимости от измерения выборки $k = 1, \ \ldots, \ n$. Фактически эти 
термины  означают  <<имеющие равную неопределенность>>  и  <<имеющие неодинаковые 
неопределенности>>. \index{равноширинные измерения}\index{неравноширинные измерения} 



\subsection{Информационное множество} 
\label{InfoSetSect}

Информационным множеством в случае оценивания единичной постоянной величины по выборке 
интервальных данных будет также интервал, который будем называть \emph{информационным 
интервалом}. Другими словами, это интервал, содержащий значения оцениваемой величины, 
которые совместны с измерениями выборки (согласуются с данными этих измерений). 
Но конкретный смысл, вкладываемый в понятия <<совместные>> или <<согласующиеся>>, будет 
различен для разных ситуаций. В частности, он зависит от того, является ли выборка  
интервальных данных накрывающей или нет. \index{информационный интервал}

\subsection{Оценки точечные и интервальные} 
\label{EstPointInte}

Оценки величин, которые мы получаем с помощью традиционной статистики, как известно, 
могут быть \emph{точечными} или \emph{интервальными}. 

Точечные оценки, т.\,е. оценки в виде точки --- числа, вектора или матрицы --- 
соответствуют, как правило, тому типу данных, который используется в модели 
рассматриваемого объекта или явления и могут непосредственно использоваться 
при его дальнейшем исследовании, при прогнозировании его поведения и т.\,п. 

Интервальные оценки дают области возможных значений точечных оценок и нужны 
для характеризации их возможного разброса и изменчивости (которую мы будем также 
называть \emph{вариабельностью},\index{вариабельность}. В традиционной вероятностной статистике оценки 
параметров сами являются случайными величинами, а носители их вероятностных 
распределений могут быть неограниченными. По этой причине при определении интервальных 
оценок в вероятностной статистике обычно задают некоторый \emph{уровень значимости} 
или \emph{уровень доверительной вероятности}, с помощью которых выполняют усечение 
вероятностного распределения. Тем самым всегда обеспечивается ограниченность 
интервальных оценок и их практичность. 

В интервальном анализе данных оценки величин также могут быть \emph{точечными} либо 
\emph{интервальными} или даже иметь форму каких-то других множеств. Точечная оценка 
несёт тот же смысл, что и в традиционной статистике, а интервальная оценка тоже даёт 
область возможных значений точечных оценок, характеризуя их возможный разброс и 
вариабельность.

Но есть и существенные отличия от вероятностной статистики. Во-первых, задание уровня 
значимости не требуется, так как множества значений оценки, как правило, ограниченны. 
Во-вторых, интервальные оценки могут иметь различных смысл --- быть внутренними, 
внешними или какими-нибудь другими, сообразно чему их смысл различен. В-третьих, в пределах внутренней интервальной оценки 
все значения равноценны и тоже могут служить точечными оценками рассматриваемой величины. 
Напротив, в традиционной вероятностной статистике точечные значения внутри интервальной 
оценки не вполне равноценны друг другу. 

\section{Измерение постоянной величины} 
\label{MeasrConstChap}

\emph{Постоянная величина} --- это величина, которая в рассматриваемом 
процессе сохраняет свое значение неизменным. К примеру, рост человека не меняется 
заметно в процессе его измерения, и потому может считаться постоянной величиной, 
хотя на протяжении жизни человека (или даже в течение суток) его рост, конечно же, 
несколько изменяется.                             \index{постоянная величина} 

\subsection{Оценки постоянной} 
\label{EstConst}

Пусть $\{\mbf{x}_{k}\}_{k=1}^{n}$ --- накрывающая выборка интервальных данных, 
т.\,е. такая, в которой доминирующая часть (большинство и т.\,п.) измерений 
являются накрывающими --- содержат истинное значение интересующей нас величины 
(см. \S\ref{InteSampleSect}). 

Если истинное значение величины содержится в интервалах измерений выборки  $\{\,\mbf{x}_{k}\}_{k=1}^n$ за исключением, возможно, некоторых немногих, то оно 
должно принадлежать также пересечению этих интервалов. Следовательно, уточнённым 
интервалом принадлежности истинного значения можно взять 
\begin{equation} 
\label{IXInterval} 
\mbf{I}\; = \;\bigcap_{1\leq k\leq n} \mbf{x}_{k}. 
\end{equation} 
Это и будет информационное множество $\mbf{I}$ оценки измеряемой физичес\-кой величины 
(см. Рис.~\ref{EncloConstPic}). Его можно называть для краткости \emph{информационным 
интервалом}. Явные выражения для его левой (нижней) и правой (верхней) границ даются 
следующими формулами:                            \index{информационный интервал}
\begin{equation}
\label{LoUpBounds} 
\un{\mbf{I}}\, = \,\max_{1\leq k\leq n} \,\un{\mbf{x}}_{k};
\quad
\ov{\mbf{I}}\, = \,\min_{1\leq k\leq n} \,\ov{\mbf{x}}_{k}. 
\end{equation}           

\subsection{Граф совместности выборки} 
\label{CompatibilityGraph} 

Взаимное отношение интервальных измерений в выборке в смысле их совместности друг 
с другом удобно описывать \emph{графом совместности выборки}, \index{граф совместности 
выборки} а для исследования структуры выборки прибегать к развитым инструментам теории 
графов. 

\emph{Граф} представляет собой математическую абстракцию системы 
произвольной природы, в которой между составляющими её объектами возникают парные 
связи. Граф есть совокупность двух множеств --- множества объектов, называемого 
\emph{множеством вершин}, и множества их пар, именуемого \emph{множеством рёбер}. 
Наглядно граф представляется диаграммой, где вершины отображаются в виде точек, 
а рёбра --- в виде отрезков (или дуг), соединяющих пары вершин. 

В графе совместности выборки $\mbf{X} = (\mbf{x}_{1}, \mbf{x}_{2}, \ldots, \mbf{x}_{n} )$ 
множество вершин образовано интервалами $\mbf{x}_{i}$ выборки $\mbf{X}$, а множество 
рёбер составлено из пар интервалов $(\mbf{x}_{j},\mbf{x}_{k})$, которые имеют непустое 
пересечение, то есть совместны друг с другом (Рис.~\ref{CompatGraphPic}). Графы, 
определённые таким способом по набору интервалов, в теории графов принято называть 
\emph{интервальными графами} или \emph{графами интервалов}. Они неплохо изучены и находят 
много приложений. Кроме того, интервальные графы обладают 
рядом полезных свойств, которые упрощают решение различных задач.\index{интервальный граф} 
\index{граф интервалов}  

Поскольку в совместной выборке любые два интервала дают непустое пересечение, то 
в соответствующем ей графе совместности рёбрами соединены все возможные пары вершин. 
В теории графов такие графы называются \emph{полными}.\index{полный граф} Очевидно, что 
графы совместности для несовместных выборок полными не являются. В них отсутствуют рёбра, 
соединяющие пары вершин, соответствующие непересекающимся интервалам.

Предметом интереса в несовместных выборках являются максимальные совместные подвыборки. 
Термин <<максимальная>> здесь означает, что такую подвыборку невозможно расширить 
какими-либо дополнительными интервальными измерениями выборки, при сохранении свойства 
её совместности. При этом несовместная выборка может содержать несколько максимальных 
совместных подвыборок различной или одинаковой длины. 

Нетрудно заметить, что в графе совместности выборки интервальных измерений совместным 
подвыборкам соответствуют полные подграфы, а в случае, когда совместная подвыборка 
максимальна, --- максимальный полный подграф. В теории графов максимальный полный 
подграф имеет специальное название --- \emph{клика}.\index{клика} 

\subsection{Мера совместности выборки}
\label{CompatibilityMeasure} 

Для описания выборок, помимо оценок их размеров, желательно иметь дополнительную информацию о мере сходства элементов выборки. В п.~\ref{InteWidClass} был рассмотрен вопрос о классификации выборок в зависимости от соотношения ширин интервалов в выборке по отношению к их полной вариабельности.
При определении накрывающих выборок в п.~\ref{CoverMeasrSect} отмечалось, что понятие невозможно определить строго, поскольку жесткие требования к <<накрытию>> приводят к исключению из рассмотрения большинства практических ситуаций.

В различных областях анализа данных в науках о Земле, биологии, информатике используют множество мер сходства множеств \cite{Jaccard}.  
Мера сходства бинарная: $S(A, B) \rightarrow [0, 1] $ --- это вещественная функция между объектами $A, B$. \index{IoU --- Intersection over Union}
Формально  принадлежность к мерам сходства определяется системой аксиом:
\begin{itemize}
\item[---] ограниченность $0 \leq 	S(A, B)  \leq 1 $;
\item[---] симметрия $	S(A, B) = S(B,A)  \leq 1$;
\item[---] рефлексивность $	S(A, B)=1  \Longleftrightarrow A=B $;
\item[---] транзитивность $ 	A \subseteq B \subseteq C \Longrightarrow   S(A, B) \geq S(A, C)  $.
\end{itemize}
Эти свойства также называют $t$-нормой. Существуют и иные системы аксиом сходства.
\index{$t$-норма}  \index{индекс Жаккара} \index{мера сходства}
В компьютерных приложениях (обработка изображений, машинное обучение) меру сходства множеств  обозначают как \emph{IoU} (\emph{Intersection over Union}). В математике часто используется термин \emph{индекс Жаккара}, по имени математика, предложившего подобную меру.
\label{JaccardMeasure}
В процессе развития интервального анализа были введены различные определения и конструкции оценки меры совместности интервальных объектов.
Вместе с тем в практике обработки данных часто необходимо оперировать относительными величинами. В частности, это нужно в связи с необходимостью сопоставления допусков и размеров деталей, погрешности измерителей и значений измеряемых величин и т.\,п. \cite{Kabir2017}.

Представим обощение меры Жаккара на выборки интревалов \cite{Jaccard2022}.
Введем базовую конструкцию совместности для двух интервалов.
Рассмотрим  следующую числовую характеристику степени совместности  двух интервалов $\mbf{x}, \mbf{y}$:
\begin{equation}\label{Rwid}
\mathrm{JK}(\mbf{x}, \mbf{y}) = 
\frac{\w (\mbf{x} \wedge \mbf{y} )}{\w (\mbf{x} \vee \mbf{y})}.
\end{equation}
В выражении \eqref{Rwid} используется ширина интервала (см. стр. \pageref{InteWid}), а вместо операций пересечения и объединения множеств --- операции взятия  минимума ($\wedge$) \eqref{InteMinExpr} и максимума ($\vee$) \eqref{InteMaxExpr} по включению двух величин в полной интервальной арифметике Каухера. В наименовании $\mathrm{JK}(\mbf{x}, \mbf{y})$ буква $\mathrm{J}$ указывает на фамилию 
Jaccard, а $\mathrm{K}$ --- на арифметику Каухера.
В общем случае минимум по включению в выражении \eqref{Rwid} может быть неправильным интервалом. 

Рассмотренная мера обобщает обычное понятие меры совместности на различные типы взаимной совместности интервалов. 
Если пересечение интервалов $\mbf{x}, \mbf{y} $ пусто, иначе $\mbf{x} \, \cap \, \mbf{y} = \varnothing$, то $\mbf{x} \, \wedge \, \mbf{y}$ --- неправильный интервал и числитель \eqref{Rwid} имеет отрицательное значение. 
В предельном случае вещественных значений $x \neq y$ имеем
\begin{equation*}
\mathrm{JK}(x, y) =-1.
\end{equation*}
В целом получаем
\begin{equation}\label{Rwidrange}
-1 \leq \mathrm{JK}(\mbf{x}, \mbf{y}) \leq 1.
\end{equation}
Таким образом, величина $\mathrm{JK}$  непрерывно описывает ситуации от полной несовместности вещественных значений $x \neq y$ до полного перекрытия интервалов $\mbf{x} = \mbf{y}$.
Следует заметить, что в отличие от случая вещественных величин, для которых для  индекса Жаккара имеется только два значения, 0 и 1, конструкция \eqref{Rwid} даёт последовательную характеризацию различных отношений сходства интервалов.

Мера совместности, введенная  для двух интервалов  в форме \eqref{Rwid}, допускает естественное обобщение в случае интервальной выборки. 
Пусть имеется интервальная выборка  $\mbf{X} = \{ \mbf{x}_i \}, \ i=1,2, \ \ldots, \ n.$
Определим меру $\mathrm{JK}(\mbf{X}) $ как 
\begin{align} 
\mathrm{JK}(\mbf{X}) = 
\frac{\w (\bigwedge_i \mbf{x}_i )}{\w (\bigvee_i \mbf{x}_i)}. \label{RwidSetKR}
\end{align}
Важно, что выражение \eqref{RwidSetKR} переходит в случае интервальной выборки из двух элементов в выражение \eqref{Rwid}. %Таким образом, принцип соотвествия выполнен.

\begin{example}{Пример вычисления меры совместности для накрывающей выборки.}
Пусть имеется интервальная выборка из четырех элементов \eqref{ModeExampleData}, рассмотренная при вычислении интервальной моды в п.~\ref{ModeSampleSect}
\begin{equation*}
	\mbf{X}   = \{ 
	[1, 4],  [5, 9],  [1,5, 4,5],   [6, 9]   \}.
\end{equation*}	
Диаграмма рассеяния выборки $\mbf{X}$ приведена на рис. \ref{f:ModelData2hor}.
Выберем из нее накрывающую подвыборку
\begin{equation*}
	\mbf{X}_{c}   = \{  
	[5, 9],   [6, 9]  \}.
\end{equation*}	
Для выборки $\mbf{X}_{c}$  имеем согласно \eqref{RwidSetKR}
\begin{equation*}
	\mathrm{JK}(\mbf{X}_{c}) = \frac{9-6}{9-5} = 0,\!75.
\end{equation*}
Значение $\mathrm{JK}(\mbf{X}_{c})$, близкое единице, демонстрирует высокую меру сходства элементов выборки $\mbf{X}$.	
\end{example}


\subsection{Мода интервальной выборки} 
\label{ModeSampleSect} 

В традиционной статистике важной характеристикой выборки является ее \emph{мода} 
--- значение из выборки, которое встречается наиболее часто. Для непрерывного 
вероятностного распределения мода --- точка с наибольшей плотностью вероятности. 

Имеет смысл  распространить понятие моды на обработку интервальных данных, где оно 
будет обозначать интервал тех значений, которые встречаются 
в интервалах обрабатываемых данных наиболее часто. Фактически это означает, что 
точки из моды интервальной выборки накрываются наибольшим числом интервалов этой 
выборки. 
Ясно, что по своему определению понятие моды имеет  наибольший смысл для накрывающих выборок. 
\begin{definition} 
\textsl{Модой} интервальной выборки назовем интервал пересечения ее наибольшей 
совместной подвыборки.  \index{мода выборки} 
\end{definition} 

\subsection[Выборки унимодальные и мультимодальные]%
{Выборки унимодальные \\* и мультимодальные} 
\label{UniMultiModSect} 

Тот факт, что выборка не является унимодальной, может быть признаком сложной 
внутренней структуры описываемого ею явления. 
Исследуемая величина может, к примеру, не быть 
постоянной, а является композицией нескольких близких постоянных величин. 

Далее моду с наибольшей частотой 
будем называть \emph{основной модой} (или \emph{главной модой}).\index{основная мода} Прочие 
моды выборки будут называться \emph{неосновными}. \index{неосновная мода}\index{главная мода}  

%\subsection{Вариабельность оценки} \label{ConstVariabSect}  

%\subsection{Приём варьирования неопределённости}  \label{UncertAlterSect} 


\chapter{Обработка данных методами интервального анализа}\label{s:DataAnalysis}

\section{Характеризация выборки постоянной величины}

Программное обеспечение С.Жилина \cite{InteSoft}, \cite{InteKaucherSoft}.

\subsection{Органический углерод}

Данные \S\ref{s:CarbonData}, Табл.~\ref{t:OrganicCarbonVariation}

\begin{align}
\delta ^{13}\mbf{C} =	10^3 \cdot \delta ^{13}C_{VPDB} = \{ \
&	[-35, -21 ], [ -16, -9 ], [ -34, -10 ] \nonumber \\
&	[ -74.3, -2 ], [ -130.31, +7 ], [ -30, -19], \nonumber \\
& 	[ -44, -16.8], [ -32, -10.3] \	\}
\end{align}

Диаграмма рассеяния величины $\delta ^{13}\mbf{C}$ представлена на Рис.~\ref{f:OrganicCarbonVariation}.
\begin{figure}[ht] 
	\centering\small
	\unitlength=1mm
	{\includegraphics[width=80mm]{Figures/CarbonOrganicDataIUPAC2016.png}} 
	\caption{Вариации углерода в органических объектах} в единицах $10^3 \cdot \delta ^{13}C_{VPDB}$
\label{f:OrganicCarbonVariation}
\end{figure}

Интервальная мода \index{интервальная мода}  
вариаций $10^3 \cdot \delta ^{13}C_{VPDB}$ в органических объектах представлена на Рис.~\ref{f:OrganicCarbonVariationMode}
\begin{figure}[ht] 
\begin{center}
	\unitlength=1mm
	{\includegraphics[width=80mm]{Figures/CarbonOrganicDataIUPAC2016ModeArray.png}} 
	\caption{Интервальная мода 
		вариаций $10^3 \cdot \delta ^{13}C_{VPDB}$ в органических объектах} 
	\label{f:OrganicCarbonVariationMode}
\end{center}	
\end{figure}

Численное значение интервальной моды составляет 
\begin{equation}\label{OrganicCarbonVariationMode}
	\mode ( 10^3 \cdot \delta ^{13}C_{VPDB} ) = [-30, -21].
\end{equation}
Величину \eqref{OrganicCarbonVariationMode} можно использовать как \emph{внутренюю оценку}.
В качестве \emph{внешней оценки} можно взять значение
\begin{equation}\label{OrganicCarbonVariationOut}
[\min \, \delta ^{13}\mbf{\un{C}},  \max \, \delta ^{13}\mbf{\ov{C}}] = [ -130.31, +7 ].
\end{equation}

Полученные оценки \eqref{OrganicCarbonVariationMode} и \eqref{OrganicCarbonVariationOut} можно собрать в твин в форме с упорядочением по включению \eqref{TwinPresentations} \S\ref{s:Twin}:
\begin{equation}\label{OrganicCarbonVariationTwin}
\delta ^{13}\mbf{C}_{\subseteq} = \left[ \, [-30, -21], [-130.31, +7] \, \right].
\end{equation}
Твин \eqref{OrganicCarbonVariationTwin} можно использовать в различных расчётах, например атомной массы органических молекул.

\subsection{Свинец}

Данные \S\ref{s:CarbonData}, Табл.~\ref{t:204LeadVariation}

\begin{align}
\mbf{\chi} (^{204}Pb) =	 \{	& [ 0.0115, 0.0136 ], 	[ 0.0078, 0.0160 ], [ 0.0118, 0.0137 ] , \nonumber \\
& [ 0.0113, 0.0131 ], [ 0.0096, 0.0140 ] 	\} 
\end{align}

Диаграмма рассеяния 
\begin{figure}[ht] 
	\centering\small
	\unitlength=1mm
	{\includegraphics[width=80mm]{Figures/PbSRchi204DataIUPAC2021.png}} 
	\caption{Вариации $^{204}Pb$ в sedimentary rocks,  $ \chi ^{204}Pb$}
	\label{f:PbSRchi204DataIUPAC2021}
\end{figure}

Интервальная мода \index{интервальная мода}
\begin{equation}
	\mode ( \chi ^{204}Pb) = [0.0118, 0.0131].
\end{equation}

\begin{figure}[ht] 
	\begin{center}
		\unitlength=1mm
		{\includegraphics[width=80mm]{Figures/PbSRchi204DataIUPAC2021ModeArray.png}} 
		\caption{Интервальная мода 
			вариаций  $^{204}Pb$ в sedimentary rocks,  $ \chi ^{204}Pb$} 
		\label{f:PbSRchi204DataIUPAC2021Mode}
	\end{center}	
\end{figure}

\section{Применение составных интервальных объектов}

В этом разделе мы рассмотрим применение составных интервальных объектов для описания изотопных распределений в различных веществах и различных вычислениях.
Базовыми составными интервальными объектами являются твины и мультиинтрервалы, рассмотренные соответственно в \S\ref{s:Twin} и \S\ref{s:MultiIntervals}.
При вычислениях нам понадобятся различные конструкции и операции, представленные  в \S\ref{InteAnalData}.

\subsection{Твины --- расчёт молекулярной массы молекулы метана}

Твины, интервалы интервалов, см. \S\ref{s:Twin}, дают возможность более подробного описания интервальной неопределенности. При этом возможны различные способы такого описания.
С одной стороны, можно самим концам интервальных оценок приписать интервальную неопределённость. В этом  случае, можно устанавливать линейный порядок между твинами. 
С другой стороны, с помощью твинов можно строить одновременно внутренние и внешние оценки (твины Нестерова). Продемонстрируем на примере, как это можно делать.

Рассмотрим вычисление атомной массы молекулы метана $CH_4$. Для проведения вычислений будем использовать базовое программное обеспечение поддержки интервальных операций в языке Python \index{Python} {\tt intvalpy} А.Андросова \cite{intvalpy},
библиотеку работы с твинами Нестерова А.Жаворонковой {\tt twin} \cite{PythonTwin} и
программу работы с изотопными данными Т.Яворук {\tt MendeleevTwin} \cite{MendeleevTwin}.

Данные об изотопных вариациях берутся c сайта \cite{IsotopeData2018}, описания которых дано в публикации \cite{IUPACTables}.
Файл имеет формат .xlsx Excel и содержит страницы с данными для изотопов 12  элементов:
H, Li, B, C, N, O, Mg, Si, S, Cl, Br, Tl.
%['Hydrogen','Carbon']

Для превого компонента выбранного вещества, в данных выбирается закладка с именем 

\begin{table}[ht]
\caption{Нижние и верхние границы атомных масс водорода в различных категориях веществ}	
\label{IsotopeDataHydrogen}
{\small
\begin{tabular}{llcc}
Category & Subcategory & $\un{\mbf{M}}(H)$ & $\ov{\mbf{M}}(H)$ \\
\hline
Water&Sea water (deep)&1,0079814&1,0079823 \\
Water&Other (naturally occurring)&1,0079042&1,0080020\\
Water&Fruit juice and wint&1,0079750&1,0079891\\
Silicates&Pectolite&1,0079145&1,0079377\\
Silicates&Other&1,0079492&1,0079825\\
Hydroxides&Aluminum and iron&1,0079473&1,0079805\\
Organic hydrogen&Non-marine organisms&1,0079446&1,0079921\\
Organic hydrogen&Marine organisms&1,0079557&1,0079797\\
Organic hydrogen&Organic sediments&1,0079656&1,0079725\\
Organic hydrogen&Coal&1,0079564&1,0079716\\
Organic hydrogen&Crude oil&1,0079562&1,0079692\\
Organic hydrogen&Ethanol (naturally occurring)&1,0079391&1,0079504\\
Organic hydrogen&Reagents (synthetic)&1,0079598&1,0081074\\
Methane&Atmospheric&1,0079454&1,0079706\\
Methane&Other (naturally occurring)&1,0078985&1,0079609\\
Hydrogen Gas&Air&1,0079604&1,0080100\\
Hydrogen Gas&Other (naturally occurring)&1,0078507&1,0079426\\
Hydrogen Gas&Commercial tank gas&1,0078543&1,0079730\\
Hydrogen Gas&Automobile exhaust and &1,0078736&1,0079587\\
~&industrial contamination&~&~\\
\hline
\end{tabular}
}
\end{table}

Пусть источником водорода и углерода является  морская фауна. 

Представим массы компонент молекулы в виде твинов в форме Нестерова --- см. \S\ref{s:Twin}.
\begin{equation*}
	\mbf{X}_{\subseteq} = [ \, \mbf{X}^{in} \, ,  \, \mbf{X}^{out} \, ]
\end{equation*}
$\mbf{X}^{in}, \mbf{X}^{out}$ --- интервалы внутренней и внешней оценки величины.


\emph{Внешней оценкой} будет служит максимум по включению для атомов молекулы, \\
\emph{внутренней оценкой} --- мода выборки.

% внешняя оценка Hydrogen :  '[1.00012, 1.00017]'
% мода Hydrogen :  '[1.00013, 1.00015]'
%внешняя оценка Carbon :  '[12.0096, 12.0111]'
%мода Carbon :  '[12.0102, 12.011]'
%T_h= ['[1.00013, 1.00015]', '[1.00012, 1.00017]']
%T_c= ['[12.0102, 12.011]', '[12.0096, 12.0111]']
%T=T_c+4*T_h= ['[16.0109, 16.0115]', '[16.0101, 16.0118]']
Используем программное обеспечение \cite{intvalpy}, \cite{PythonTwin}, \cite{MendeleevTwin}.

	Твины атомных весов водорода и углерода
	\begin{align}
		\mbf{M}(H) = &[ \, [1.00013, 1.00015] \, ,  \, [1.00012, 1.00017] \, ], \\
		\mbf{M}(C) = & [ \, [12.0102, 12.011] \, ,  \, [12.0096, 12.0111] \, ]
	\end{align}	
	Твин веса молекулы метана $CH_4$
	\begin{equation}
		\mbf{M} (CH_4) = [ \, [16.0109, 16.0115] \, ,  \, [16.0101, 16.0118] \, ].
	\end{equation}	
	Графическое представление твина веса молекулы метана $\mbf{M} (CH_4)$
	\begin{figure}[ht] 
		\begin{center}
			\unitlength=1mm
			\begin{picture}(90,15)
				\put(10,5){\includegraphics[width=60mm]{Figures/TwinCH4.png}}
				\put(45,10){\mbox{\small $X^{in}$}} 
				\put(35,9){\vector(1,0){28}}
				\put(57,9){\vector(-1,0){28}}
				\put(10,4){\vector(1,0){60}}
				\put(60,4){\vector(-1,0){50}}
				\put(30,0){\mbox{\small $X^{out}$}}
			\end{picture}	%	{\includegraphics[width=60mm]{Figures/TwinCH4.png}} 
			\caption{Твин молекулы метана $CH_4$} 
			\label{f:TwinCH4}
		\end{center}	
	\end{figure}	

\subsection{Мультинтервалы --- пример технеция}

Правило запрета Маттауха — Щукарева: в природе не могут существовать два стабильных изобара, заряды ядра которых отличаются на единицу  \S\ref{IsobarChains}. \index{правило Щукарева — Маттауха}
Если у какого-либо химического элемента есть устойчивый изотоп, то его ближайшие соседи по таблице не могут иметь устойчивых изотопов с тем же массовым числом.  
Пример ---$_{43}Tc^{55}$.
\begin{figure}[ht] 
	\centering\small
	\unitlength=1mm
	\includegraphics[height=35mm, width=40mm]{Figures/stable_isotopes_Tc.png} 
	\caption{Число стабильных изотопов у элементов } 
	\label{f:stable_isotopes_by_elementTc}
\end{figure}	

Рассмотрим вопрос, какие изотопы имеют соседи технеция, молибден и рутений. Природное распеределение изотопов у молибдена и рутения показано на Рис.~\ref{f:MoAbudanceRuAbudance}.
\begin{figure}[ht] 
	\centering\small
	\unitlength=1mm
	\includegraphics[width=80mm]{Figures/MoAbudanceRuAbudance.png} 
	\caption{Распределение изотопов у элементов молибдена и рутения} 
	\label{f:MoAbudanceRuAbudance}
\end{figure}

Видно, что во всём диапазоне возможных изотопов технеция присутствуют изотопы соседей.
Анализ картины в виде, показанном на Рис.~\ref{f:MoAbudanceRuAbudance}, весьма непрост. По-видимому, наиболее близкими по типу математическими объектоми являются нечёткие множества \cite{Batyrshin2001}.
При этом характер распределений сложен и содержит <<разрывы>>, так что аналитическое описание функции принадлежности здесь невозможно.

Упростим максимально способ представления информации о распространённости молибдена и рутения, в виде индикаторной функции <<да-нет>>. В таком случае, можно сложную картину распределений выразить в виде мультиинтервалов с массами изотопов. 

Мультиинтревалы стабильных изтопов молибдена и рутения и наиболее долгоживущий изотоп технеция \index{технеций}

\begin{equation*}
	\begin{aligned}
		\mbf{M}_{Mo} & = [92] \cup  [94, 98] \cup [100], \\
		\mbf{M}_{Tc} &=  [ {\color{red}98}   ], \\
		\mbf{M}_{Ru} &= [96] \cup  [98, 102] \cup [104].
	\end{aligned}
\end{equation*}
или, с выделением массы 98,
\begin{equation*}
	\begin{aligned}
		\mbf{M}_{Mo} & = [92, [94,& ~{\color{red}98}]&, 100 \ ], \\
		\mbf{M}_{Tc} &=  &[ {\color{red}98} &  ], \\
		\mbf{M}_{Ru} &= [ 96,   & [{\color{red}98}&, 102], 104 \ ].
	\end{aligned}
\end{equation*}
Минимум по включению мультиинтревалов стабильных изтопов молибдена и рутения содержит значение наилучшего кандидата для технеция $_{43}Tc^{98}$ 
\begin{equation*}
	\mbf{M}_{Mo} \cap \mbf{M}_{Ru} = [96, {\color{red}98}] \supseteq \mbf{M}_{Tc}.
\end{equation*}

На Рис.~\ref{f:IsobarTriplet} показана триада изобар Mo-Tc-Ru.
\begin{figure}[ht] 
	\begin{center}
		\unitlength=1mm
		{\includegraphics[width=100mm]{Figures/MoTcRuNZmarked.png}}
	\end{center}	
			\caption{Триплет изобар Mo-Tc-Ru} 
\label{f:IsobarTriplet}
\end{figure}	

Таким образом, формально как рассмотренный  изотоп $_{43}Tc^{98}$, так и иные, кроме $_{43}Tc^{93}$, запрещены по Щукареву-Маттауху, и приведённый пример иллюстрирует простейшее  математическое описание ситуации. 

Следует заметить, что в природе дело обстоит сложнее, и изотопы с весами 97-99 весьма долгоживующие, и в искусственно синтезированном виде используются человечеством.
Подробно теория вопроса с точки зрения теории ядра рассмотрена в \cite{Technetium2017}.

\newpage
\section{Программное обеспечение для анализа данных с интервальной неопределённостью}
\index{Программное обеспечение}

\subsection{Интервальная арифметика для анализа данных}


С.И.Жилин \\
Matlab/Octave/Scilab:\\
https://github.com/szhilin/octave-interval-examples \cite{InteSoft}\\
https://github.com/szhilin/kinterval \cite{InteKaucherSoft} \\
Julia
https://github.com/szhilin/julia-interval-examples
Python: \\
базовая библиотека\\
{\tt intvalpy} --- А.Андросов \\
 https://github.com/AndrosovAS/intvalpy, \cite{intvalpy} \\
арифметика твинов Нестерова\\
{\tt twin} --- А.Жаворонкова \\
https://github.com/Zhavoronkova-Alina/twin \cite{PythonTwin}

\subsection{Специальные программы для изотопов}
Т.Яворук \\
Вычисления с изотопами\\
{\tt MendeleevTwin} ---  https://github.com/Tatiana655/MendeleevTwin \cite{MendeleevTwin}



\chapter*{Заключение}\label{Conclusion}


Выражаю благодарность участникам Всероссийского интервального веб-семинара, С.И.\,Жилину, С.И.\,Кумкову, А.В.\,Пролубникову, Е.В.\,Чаусовой  и С.П.\,Шарому, за проявленный интерес к работе и конструктивные обсуждения примеров. 





\addcontentsline{toc}{chapter}{Литература}	
\begin{thebibliography}{00}


	\bibitem{BookIntStat} А.Н.\,Баженов, С.И.\,Жилин, С.И.\,Кумков, С.П.\,Шарый. <<Обработка и анализ данных с интервальной неопределённостью>>. 2021.

	\bibitem{Bazhenov2020}
	А.Н. Баженов. Интервальный анализ. Основы теории и учебные примеры: учебное пособие. --- СПб. 2020
	\url{https://elib.spbstu.ru/dl/2/s20-76.pdf/info}
	
		\bibitem{Bazhenov2022}
	А.Н. Баженов. Обобщение мер совместности для анализа данных с интервальной неопределённостью. – СПб., 2022. – 80 с.
	\url{https://elib.spbstu.ru/dl/5/tr/2022/tr22-142.pdf/info}

	\bibitem{Bohr1928} 
	%Bohr, N. (1928). "The Quantum Postulate and the Recent Development of Atomic Theory". Nature. 121 (3050): 580–590. doi:10.1038/121580a0
	\textsc{Бор Н.} Квантовый постулат и новое развитие атомистики. УФН 8 306–337 (1928)
	%\textsc{Ельяшевич М.А.} Развитие Нильсом Бором квантовой теории атома и принципа 
	%соответствия // Успехи Физических Наук. -- 1985. -- Т.~147, вып.~2. -- С.~253--301. \  
	%DOI:  10.3367/UFNr.0147.198510c.0253 
	
	\bibitem{Veselovcky1961} \textsc{Веселовский И. Н.}
Аристарх Самосский — Коперник античного мира] // Историко-астрономические исследования, вып. VII. — М., 1961. — С. 17—70. \\
http://www.astro-cabinet.ru/library/Aristarch/Aristarch\_3.htm 
	
	
	
	\bibitem{Kantorovich}
	\textsc{Канторович Л.В.} О некоторых новых подходах к вычислительным методам и обработке 
	наблюдений // Сибирский Математический Журнал. -- 1962. -- Т. 3, №5. -- С.~701--709. 
	
	
	\bibitem{OpaevIsotope} 
	\textsc{Опаев А.} Изотопная подпись. \\	
	https://elementy.ru/problems/1523/Izotopnaya\_podpis
	
	
	\bibitem{OskorbinMaksimovZhilin}
	\textsc{Оскорбин Н.М., Максимов А.В., Жилин С.И.} 
	Построение и анализ эмпирических зависимостей методом центра неопределенности // 
	Известия Алтайского государственного университета. -- 1998. -- №1. -- С.~35--38. 
	
	\bibitem{IntervalAnalysisExamples} 
	Примеры анализа интервальных данных в Octave \\ 
	\url{https://github.com/szhilin/octave-interval-examples}
	

	\bibitem{SharyBook} 
	\textsc{Шарый С.П.} Конечномерный интервальный анализ. -- ФИЦ ИВТ: 
	Новосибирск, 2022. \     Электронная книга, доступная 
	на \url{http://www.nsc.ru/interval/Library/InteBooks/SharyBook.pdf} 

	\bibitem{Iakovlev1968}
	\textsc{Яковлев А.Г.} Машинная арифметика мультиинтервалов // Вопросы кибернетики
	(Научный Совет по компл. проблеме <<Кибернетика>>: АН СССР). – 1986. – Вып. 125. –
	С. 66–81.

	\bibitem{Iakovlev2013}  \textsc{И.Яковлев.} Изучение трофической структуры сообществ с помощью анализа стабильных изотопов. Дискуссионные лекции-семинары по эволюционной экологии, 08.11.2013 \\
	http://www.eco.nsc.ru/lectures/Iakovlev\_Stable\_Isotopes.pdf
	
\bibitem{Archimedes} \textsc{Archimedes}
(Dover Books on Mathematics). 
Archimedes, Sir Thomas Heath. Unabridged reprint of the classic 1897 edition, with supplement of 1912. - p.512.

	
	\bibitem{Bowen2010} 
	\textsc{G.J. Bowen} Isoscapes: Spatial Pattern in Isotopic Biogeochemistry.  Annu. Rev. Earth Planet. Sci. 2010. 38:161–187 \\
	http://www.iai.int/admin/site/sites/default/files/uploads/ 2010\_Bowen\_Isoscapes\_Spatial-Pattern-in-Isotopic-Biogeochemistrypdf.pdf

	\bibitem{Newsome2007} \textsc{S.D. Newsome, C.Martinez del Rio, S. Bearhop, and D.L. Phillips.}
	A niche for isotopic ecology. Front Ecol Environ 2007; 5(8): 429–436, doi:10.1890/060150.01

	\bibitem{Condie2015} 
	Condie, K. C. (2015). Earth as an evolving planetary system. (3rd ed.) Elsevier. \\ https://doi.org/10.1016/C2015-0-00179-4
	
	\bibitem{InteNotation} 
	\textsc{Kearfott, R.B., Nakao, M., Neumaier, A., Rump, S., Shary, S.P., van Hentenryck, 
		P.} Standardized notation in interval analysis // Вычислительные Технологии. -- 
	2010. -- Т.~15, №1. -- С.~7--13. 
	
	\bibitem{IUPAC} 
	\textsc{Meija, J., Coplen, T.B., Berglund, M., Brand, W.A., De Bièvre, P., 
		Gröning, M., Holden, N.E., Irrgeher, J., Loss, R.D., Walczyk, T., Prohaska, T.} 
	Atomic weights of the elements 2013 (IUPAC Technical Report) // Pure and Applied 
	Chemistry. -- 2016. -- Vol.~88, Issue~3. -- P.~265--291. \   DOI: 10.1515/pac-2015-0305 
	

	\bibitem{NguyenKreinWuXiang} 
	\textsc{Nguyen H.T., Kreinovich V., Wu B., Xiang G.} Computing Statistics 
	under Interval and Fuzzy Uncertainty. Applications to Computer Science and Engineering. 
	-- Springer, Berlin-Heidelberg, 2012. 
	
	\bibitem{ChemInternatl} 
	Standard atomic weights of 14 chemical elements revised // Chemistry International. 
	-- 2018. -- Vol.~40, Issue 4. -- P.~23--24. \  DOI: 10.1515/ci-2018-0409 


	\bibitem{Rader2017}
	\textsc{J.A. Rader, J. A., Newsome, S. D., Sabat, P., Chesser, R. T., Dillon, M. E., and Martínez del Rio, C.} (2017). Isotopic niches support the resource breadth hypothesis. J. Anim. Ecol. 86, 405–413. doi:10.1111/1365-2656.12629
	
	\bibitem{GraviConstItaly}  
	\textsc{Rosi G, Sorrentino F., Cacciapuoti L., Prevedelli M., Tino G.M.} 
	Precision measurement of the Newtonian gravitational constant using cold atoms //  
	Nature. 2014. Vol.~510, P.~518--521. \  DOI: 10.1038/nature13433  \\   
	Предварительная версия работы депонирована в репозитории arXiv.org, статья 
	\url{https://arxiv.org/pdf/1412.7954.pdf}
	
%	\bibitem{Wasserstein2019} 	\textsc{Wasserstein, R.L.,  Schirm, A.L., Lazar, N.A.} Moving to a world beyond 	<<p\,<\, 0.05>> // The American Statistician. -- 2019. -- Vol.~73. -- P.~1--19. \  	DOI: 10.1080/00031305.2019.1583913 
	
	\bibitem{Zhilin2005}  
	\textsc{Zhilin, S.I.} On fitting empirical data under interval error // 
	Reliable Computing. -- 2005. -- Vol.~11. -- P.~433--442. \ DOI: 10.1007/s11155-005-0050-3 
	
%%%%%%%%%%%%%%%%%%%%%%%%%%%%%%%%%%%%%%%%%%%%%%%%%%%%%%%%%%%%%%%%%%%%%%%%%%%%%%%%%%%%%%%%%%%%%%%%

\bibitem{Trifonov1974}
\textsc{Трифонов Д.Н., Кривомазов А.Н., Лисневский Ю.И.} Учение о периодичности и учение о радиоактивности (комментированная хронология важнейших событий). М., Атомиздат, 1974, 248 с.

\bibitem{Scerri2019}
\textsc{E. Scerri}
The Periodic Table. Its Story and Its Significance. 
2nd edition.  New York, NY : Oxford University Press, 2019 

\bibitem{Mendeleev1869ru}
\textsc{Менделеев, Д.} (1869). “Соотношение свойств с атомным весом элементов”. Журнал Русского Химического Общества. 1: 60—77.

\bibitem{Mendeleev1869}
\textsc{Mendeleev, Dmitri} (1869). “Versuche eines Systems der Elemente nach ihren Atomgewichten und chemischen Functionen”. Journal für Praktische Chemie. 106: 251.

\bibitem{StableIsotopeDeltas}
W.A. Brand,T.B. Coplen.
Stable isotope deltas: Tiny, yet robust signatures in nature. March 2012Isotopes in Environmental and Health Studies 48(3):393-409 
DOI: 10.1080/10256016.2012.666977

\bibitem{MendeleevTable}
https://www.chemistryworld.com/opinion/is-this-the-worlds-oldest-classroom-periodic-table/3009960.article

\bibitem{isotopes-use}
АО «Производственное объединение «Электрохимический завод» — предприятие Топливной компании «ТВЭЛ», входящей в Государственную корпорацию по ядерной энергии «Росатом».
https://www.ecp.ru/activity/nuclear/isotopes-use

\bibitem{AtomicWeights1969}
% INORGANIC CHEMISTRY DIVISION 
Atomic weights of the elements 1969 // Pure Appl. Chem., 1970, Vol. 21, No. 1, pp. 91-108 \\ http://dx.doi.org/10.1351/pac197021010091

\bibitem{IAEA} AMDC --- Atomic Mass Data Center
https://www-nds.iaea.org/amdc/

%%%%%%%%%%%%%%%%%%%%%%%%%%%%%%%%%%%%%%%%%%%%%%%%%%%%%%%%%%%%%%%%%%%%%%%%%%%%%%%%%%%%%%%%%%%%%%%%

\bibitem{IUPACIntervalHistory}
T.B. Coplen and N. E. Holden
Atomic Weights—No Longer Constants of Nature //
Chemistry International -- Newsmagazine for IUPAC, vol. 33, no. 2, 2011, pp. 10-15.// https://doi.org/10.1515/ci.2011.33.2.10

\bibitem{CIAAW}
Commission on Isotopic Abundances and Atomic Weights, CIAAW.
https://www.ciaaw.org/

\bibitem{IUPAC2018} Norman E. Holden, Tyler B. Coplen, John K. B\"{o}hlke, Lauren V. Tarbox, Jacqueline Benefield, John R. de Laetera, Peter G. Mahaffy, Glenda O’Connorb, Etienne Rotha,
Dorothy H. Tepper, Thomas Walczyk, Michael E. Wieser and Shigekazu Yoneda. \
IUPAC Periodic Table of the Elements and Isotopes (IPTEI) for the Education Community
(IUPAC Technical Report) \ Pure Appl. Chem. 2018; 90(12): 1833–2092
https://doi.org/10.1515/pac-2015-0703
Received August 3, 2015; accepted July 23, 2018

\bibitem{IUPACGoldBook}
IUPAC. Compendium of Chemical Terminology, Blackwell Scientific Publications, Oxford, 2nd ed. (1997), (the <<Gold Book>>).
Compiled by A. D. McNaught and A. Wilkinson, 0-9678550-9-8, Online version (2019-) created by S. J. Chalk. URL, https://doi.org/10.1351/goldbook 

\bibitem{IUPAC2021}
Thomas Prohaska, Johanna Irrgeher, Jacqueline Benefield, John K. Böhlke,
Lesley A. Chesson, Tyler B. Coplen, Tiping Ding, Philip J. H. Dunn, Manfred Gröning,
Norman E. Holden, Harro A. J. Meijer, Heiko Moossen, Antonio Possolo,
Yoshio Takahashi, Jochen Vogl, Thomas Walczyk, Jun Wang, Michael E. Wieser,
Shigekazu Yoneda, Xiang-Kun Zhu and Juris Meija. \\
Standard atomic weights of the elements 2021 (IUPAC Technical Report)
Pure Appl. Chem. 2022; 94(5): 573–600

\bibitem{IUPACUncertainty}
Antonio Possolo, Adriaan M. H. van der Veen, Juris Meija and D. Brynn Hibbert
Interpreting and propagating the uncertainty of the standard atomic weights (IUPAC Technical Report). \\
Pure Appl. Chem. 2018; 90(2): 395–424

\bibitem{IUPACCosensus}
T.B. Coplen, N.E. Holden, T. Ding, H.A.J. Meijer, J. Vogl, and X. Zhu, The Table of Standard Atomic Weights-An exercise in consensus, Rapid Commun Mass Spectrom. 2022; 36:e8864.
https://doi.org/10.1002/rcm.8864

\bibitem{IUPAC2007}
M. E. Wieser, M. Berglund. Atomic weights of the elements 2007 (IUPAC Technical Report). Pure Appl. Chem. 81, 2131 (2009).

\bibitem{IUPAC2014}
Brand, Willi A., Coplen, Tyler B., Vogl, Jochen, Rosner, Martin and Prohaska, Thomas. "Assessment of international reference materials for isotope-ratio analysis (IUPAC Technical Report)" Pure and Applied Chemistry, vol. 86, no. 3, 2014, pp. 425-467. https://doi.org/10.1515/pac-2013-1023

\bibitem{IUPAC2015} J. Meija. Chem. Int. 37(5), 26 (2015), https://doi.org/10.1515/ci-2015-0512, https://iupac.org/standard-atomic-weight-ofytterbium-revised/, (accessed Jan 21, 2021).

\bibitem{IUPAC2016}
J. Meija, T. B. Coplen, M. Berglund, W. A. Brand, P. De Bi`evre, M. Gröning, N. E. Holden, J. Irrgeher, R. D. Loss, T. Walczyk, T. Prohaska. Pure Appl. Chem. 88, 293 (2016).

\bibitem{IUPAC2016data} T. B. Coplen, Y. Shrestha. Pure Appl. Chem. 88, 1203 (2016).

\bibitem{IUPAC2018} J. Meija. Chem. Int. 40(4), 23 (2018), https://doi.org/10.1515/ci-2018-0409, https://iupac.org/standard-atomic-weights-of-14-chemical-elements-revised/, (accessed Jan 23, 2021).

\bibitem{IUPAC2019data}  T. B. Coplen, Y. Shrestha. Pure Appl. Chem. 91, 173 (2019).

\bibitem{IUPAC2020} J. Meija. Chem. Int. 42(2), 31 (2020), https://doi.org/10.1515/ci-2020-0222, https://iupac.org/standard-atomic-weight-ofhafnium-revised/, (accessed Jan 23, 2021).

\bibitem{IUPACTables}  T. B. Coplen, Y. Shrestha. Tables and Charts for Isotope-Abundance Variations and Atomic Weights of Selected Elements: 2016
(Ver. 1.1, May 2018), U.S. Geological Survey Data Release (2018).

\bibitem{CIAAWnaturalVariation} CIAAW. Commission on Isotopic Abundances and Atomic Weights. https://ciaaw.org/natural-variations.htm

\bibitem{TablesCharts}
https://www.sciencebase.gov/catalog/item/580e719ae4b0f497e794b7d8

\bibitem{IPTEI}  Isotopes Matter, https://www.isotopesmatter.com

\bibitem{GUM2011}
Joint Committee for Guides in Metrology. Evaluation of measurement data – Supplement 2 to the <<Guide to the expression of
uncertainty in measurement>> --- Extension to any number of output quantities, International Bureau of Weights and Measures (BIPM), S\`{e}vres, France (2011), URLBIPM, IEC, IFCC, ILAC, ISO, IUPAC, IUPAP and OIML, JCGM 102:2011, 
https://www.bipm.org/en/committees/jc/jcgm/publications (accessed Jan 25, 2021).

\bibitem{IUPACArgon} J. K. Böhlke. Pure Appl. Chem. 86, 1421 (2014).

\bibitem{Argon1993} J. Eikenberg, P. Signer, R. Wieler. Geochem. Cosmochim. Acta. 57, 1053 (1993).

\bibitem{IUPACLead} X. K. Zhu, J. Benefield, T. B. Coplen, Z. Gao, N. Holden. Pure Appl. Chem. 93, 155 (2021).

\bibitem{IUPACMS2020}
T. B. Coplen, N. E. Holden, T. Ding, H. A. J. Meijer, J. Vogl, X. Zhu. Rapid Commun. Mass Spectrom., e8864 (2020).

\bibitem{Crocetane}
M. Elvert, E. Suess, J. Greinert, M. J. Whiticar. Org. Geochem. 31, 1175 (2000).

\bibitem{NobleGasesTracers}
P. Burnard (ed.), The Noble Gases as Geochemical Tracers, Advances in Isotope Geochemistry,
DOI: 10.1007/978-3-642-28836-4\_2,  Springer-Verlag Berlin Heidelberg 2013

%%%%%%%%%%%%%%%%%%%%%%%%%%%%%%%%%%%%%%%%%%%%%%%%%%%%%%%%%%%%%%%%%%%%%%%%%%%%%%%%%%%%%%%%%%%%%%%%
\bibitem{Weizsacker}
%C. von Weizsacker. Metastabile Zustände der Atomkerne (англ.) // Naturwissenschaften (англ.)русск. : journal. — 1936. — Vol. 24, no. 51. — P. 813—814.
C.F.V. Weizsacker
Zur Theorie der Kernmassen. Z.Phys. 96 (1935) 431-458 DOI: 10.1007/BF01337700

\bibitem{NUBASE2020}
F.G. Kondev et al The NUBASE2020 evaluation of nuclear physics properties 
2021 Chinese Phys. C 45 030001
DOI 10.1088/1674-1137/abddae

\bibitem{NuclPhys}
В.В.Варламов, Б.С.Ишханов, С.Ю.Комаров Атомные ядра. Учебное пособие. ISBN 978-5-91304-122-72010. –М., Университетская книга, 2010.
http://nuclphys.sinp.msu.ru/anuc/index.html

\bibitem{Bekman}
Бекман, И. Н.  Атомная и ядерная физика: радиоактивность и ионизирующие излучения : учебник для вузов / И. Н. Бекман. — 2-е изд., испр. и доп. — Москва : Издательство Юрайт, 2022. — 493 с. — (Высшее образование). — ISBN 978-5-534-08692-8. 

\bibitem{Nucleosynthesis}
Чечев В.П., Иванчик А.В., Варшалович Д.А.
Синтез элементов во Вселенной: От Большого взрыва до наших дней
URSS. 2020. 304 с. ISBN 978-5-9710-7626-1.

\bibitem{ElementsOrigin}
Бедняков В. А. О происхождении химических элементов. Э. Ч. А. Я., Том 33 (2002), Часть 4 стр.914-963.

\bibitem{FermiNucl}
Ферми Э. Ядерная физика. Пер. с англ.1951. 346 с.

\bibitem{NeutronReactionsAstrophysics}
R. Reifarth, C. Lederer, F. K\"{a}ppeler.
Neutron Reactions in Astrophysics.
arXiv:1403.5670v1 [astro-ph.IM] 22 Mar 2014

\bibitem{IsotopeGeoChem}
Арбузов С.И. Геохимия изотопов. Лекции. Томский политехнический университет.
https://portal.tpu.ru/SHARED/s/SIARBUZOV/ucheb\_rabota/Geochemistry/lecture/Lecture10.pdf

\bibitem{IsotopeSignatureWiki}
https://ru.wikipedia.org/wiki/Изотопная\_подпись

\bibitem{Zhuravlev2019} Журавлев А.Ю. Сотворение Земли. Как живые организмы создали наш мир.
М.: Альпина Паблишер. ISBN 978-5-91671-902-4. 514 стр.

\bibitem{MS2013}
Масс-спектрометрия и хромато-масс-спектральный анализ : пособие / В. А. Винарский, Р. А. Юрченко. – Минск : БГУ, 2013. – 135 с.
ISBN 978-985-518-890-3.


\bibitem{Lebedev2003}
\textsc{А.Т.Лебедев.} Масс-спектрометрия в органической химии. Москва: БИНОМ. Лаборатория знаний, 2003. - 493 с., ил. - (Методы в химии).


%Твинные арифметики и их применение в методах и алгоритмах двустороннего интервального оценивания. Нестеров, Вячеслав Михайлович. дисс. д.ф.-м.н. 1999


		\bibitem{ModalIABook} 
\textsc{Sainz M.A., Armengol J., Calm R., Herrero P., Jorba L.J., Vehi J.}
Modal Interval Analysis: New Tools for Numerical Information. -- Cham, Switzerland: 
Springer, 2014. -- (\textsl{Lecture Notes in Mathematics; vol.~2091}). 


\bibitem{Twins1981}
\textsc{E. Gardenes, A.Trepat, J.M. Janer}
Approaches to simulation and to the linear problem in the SIGLA system. //
Gardenes, E., Trepat, A., and Janer, J. M.: Approaches to Simulation and to the Linear Problem in the SIGLA System, Freiburger Interval-Berichte 81 (8) (1981), pp. 1-28.
%FIB81-8_01-28.pdf


\bibitem{Nesterov1997} 
\textsc{Nesterov, V.M.} Interval and Twin Arithmetics. Reliable Computing 3, 369–380 (1997). https://doi.org/10.1023/A:1009945403631

\bibitem{Nesterov1999}
\textsc{В.М.~Нестеров} Твинные арифметики и их применение в методах и алгоритмах двустороннего интервального оценивания. дисс. д.ф.-м.н. г.Санкт-Петербург, Санкт-Петербургский институт информатики и автоматизации Российской академии наук, 1999, с. 234.

\bibitem{Technetium2017}
\textsc{E.V. Johnstone, M.A. Yates, F. Poineau, A.P. Sattelberger, and K.R. Czerwinski}
Technetium: The First Radioelement on the Periodic Table //
J. Chem. Educ. 2017, 94, 3, 320–326.
Publication Date:February 21, 2017
https://doi.org/10.1021/acs.jchemed.6b00343


\bibitem{Batyrshin2001}
\textsc{И.З. Батыршин} Основные операции нечёткой логики и их обобщения. --- Казань: Отечество, 2001. --- 100 с., ил.

\bibitem{InteSoft}	
\textsc{С.И.Жилин} Примеры анализа интевальных данных 
Matlab/Octave/Scilab:\\
https://github.com/szhilin/octave-interval-examples

\bibitem{InteKaucherSoft}	
\textsc{С.И.Жилин} Интервальная арифметика Каухера
https://github.com/szhilin/kinterval 

\bibitem{intvalpy}
\textsc{А.Андросов } Библиотека {\tt intvalpy} \\
https://github.com/AndrosovAS/intvalpy, 

\bibitem{PythonTwin} \textsc{А. Жаворонкова} Арифметика твинов Нестерова. \\
https://github.com/Zhavoronkova-Alina/twin

\bibitem{MendeleevTwin}	
\textsc{Т. Яворук}
Вычисления с изотопами\\
{\tt MendeleevTwin} ---  https://github.com/Tatiana655/MendeleevTwin	

\bibitem{IsotopeData2018}
Tables and charts for isotope-abundance variations and atomic weights of selected elements: 2016 (Ver. 1.1, May 2018)	\\
https://www.sciencebase.gov/catalog/item/580e719ae4b0f497e794b7d8



	
\end{thebibliography} 

\chapter{Приложения}


\section{Комментарии по стандартным атомным весам некоторых элементов}

В контексте изложения полезно сопоставить c   
Табл.~\ref{t:IsotopeOddEven} со сведениями по оболочечной модели ядра из раздела \S\ref{s:NuclPhys} 
и данными Табл.~\ref{t:IsotopeCounts} по распространённости элементов из раздела \S\ref{s:IsotopeCounts}.

Ядра с магическими значениями чисел нейтронов \eqref{MagicN} и протонов \eqref{MagicP} формируют ядра элементов-моноизотопов. Они, естественно, имеют наименьшую вариабельность в природе.
Эта вариабельность может быть очень малой. Дело осложняется наличием радиоактивных изотопов.
Они в ряде случаев непрерывно генерируются из-за взаимодействия космических излучений с атмосферой и, таким образом, присутствуют  в атмосфере, осадках, водах. Изучение таких \emph{радиогенных} изотопов одновременно и осложняет анализ данных о весах атомных элементов, и служит источником информации о различных процессах на нашей планете.
\index{радиогенные изотопы} 

Ядра, с числом нуклонов далёкие от магических, склонны образовывать не один изотоп. Далее мы рассмотрим примеры аргона и свинца, которые имеют сравнительно очень большие вариабельности изотопного состава на Земле.


\subsection{Данные IUPAC-2016}


IUPAC в отчете 2016 года опубликовал графики естественных вариаций содержания изотопов и атомных весов 15 элементов, 12 из которых которые имеют стандартные атомные веса, выраженные в виде интервала. Эти графики предоставляют информацию о вероятном значения атомного веса элемента в данном веществе \cite{IUPAC2016data, IUPAC2019data} и доступны в таблицах Excel \cite{IUPACTables}.

Кроме того, Комиссия предоставляет графики естественных вариаций содержания изотопов и атомных весов.
для элементов, стандартный атомный вес которых выражается в виде интервала, который можно загрузить с
веб-сайта Комиссии CIAAW \cite{CIAAWnaturalVariation} или через онлайн-периодическую таблицу элементов и изотопов IPTEI  \cite{IPTEI}.

Поскольку эти ресурсы свободно доступны, рассмотрим один пример.

\subsubsection{Углерод} \label{s:CarbonData}

На Рис.~\ref{f:CarbonVariation} представлены данные по распределению атомных весов углерода в различных химических формах и разного происхождения.
\begin{figure}[ht] 
	\centering\small
	%	\unitlength=1mm
	\includegraphics[width=100mm]{Figures/Carbon.png}
	\caption{Вариации углерода в различных объектах} 
	\label{f:CarbonVariation}
\end{figure}

На Рис.~\ref{f:CarbonVariation} приведены как атомный вес, так и данные по тяжеломы изотопу $^{13}C$: доле в веществе и величине  $\delta$ по отношению к стандарту VPDB.

Рассмотрим часть данных, относящихся к углероду природных объектов. Данные Табл.\ref{t:OrganicCarbonVariation} взяты  с веб-сайта Комиссии CIAAW \cite{CIAAWnaturalVariation}.

\begin{table}[h!]
	\begin{center}
		{\small
			\begin{tabular}{ccc}
				\hline
				Категория & Нижняя & Верхняя \\
				~ & граница & граница \\ 
				\hline
				Наземные растения (C3 метаболический процесс)	& -35 & -21 \\
				Наземные растения (C4 метаболический процесс)	& -16 & -9 \\
				Наземные растения  (CAM метаболический процесс)	& -34 & -10 \\
				\hline
				Морские организмы 	& -74.3 & -2 \\
				Морские отложения и соединения &	-130.3 & 7 \\
				\hline
				Уголь &	-30 & -19 \\
				Сырая нефть &	-44 & -16.8 \\
				Природный этанол & -32 &  -10.3 \\
				\hline
			\end{tabular}
		}
		\caption{Вариации углерода в органических объектах} в единицах $10^3 \cdot \delta ^{13}C_{VPDB}$
		\label{t:OrganicCarbonVariation}
	\end{center}
\end{table}
\index{тип фотосинтеза С3}
\index{тип фотосинтеза С4}
\index{тип фотосинтеза СAM}

Как видно из численных значений, они различны как между различными категориями веществ, так и внутри подкатегорий. Мы используем данные Табл.~\ref{t:OrganicCarbonVariation} в \S\ref{s:DataAnalysis} для демонстрации методов анализа данных с интервальной неопределённостью.



\subsection{Данные IUPAC-2021}

На основании отчета Подкомитета по естественной оценке фундаментального понимания
Изотопы Комиссии, \emph{нормальный материал} — это материал, происходящий из земного источника, который
удовлетворяет следующему определению: \\ \index{нормальный материал}
К обычным материалам относятся все вещества, за исключением 
(1) тех, которые подверглись преднамеренному, нераскрытому или непреднамеренному существенному воздействию искусственная изотопная модификация, \\
(2) внеземные материалы и \\
(3) изотопно-аномальные образцы, такие как природные продукты ядерного реактора из Окло (Габон) или другие уникальные явления.

В отличие от предыдущего определения, это пересмотренное определение признает тот факт, что изменение
атомный вес некоторых элементов обусловлен процессами изотопного фракционирования, которые действуют на многие разные масштабы времени.

Он также вновь вводит исключение внеземных материалов из определения Стандартные атомные веса. Новое определение более всеобъемлющее, чем некоторые более ранние версии, в отношении
встречающиеся в природе материалы, имеющие нуклеогенные и радиогенные изотопные вариации, на примере аргона
\cite{IUPACArgon} и свинец \cite{IUPACLead}.

\subsubsection{Аргон} \label{s:Argon}

Изотопный состав аргона в земных материалах изменчив. Эти вариации являются источником неопределенности в определении стандартных свойств аргона, но они дают полезную информацию во многих областях науки \cite{IUPACArgon}. Вариации состава стабильных изотопов и атомного веса аргона вызваны несколькими причинами, в том числе \\
(1) производство изотопов аргона из других элементов путем радиоактивного распада (радиогенные
изотопы) или другие ядерные превращения (нуклеогенные изотопы), и  \index{радиогенные изотопы}  \index{нуклеогенные изотопы} \\
(2) изотопное фракционирование с помощью физико-химических процессов, таких как диффузия или фазовое равновесие. \index{изотопное фракционирование} 

Физико-химические процессы вызывают
коррелированные зависящие от массы вариации отношений изотопов аргона $n(^{40}Ar)/n(^{36}Ar)$ и $n(^{38}Ar)/n(^{36}Ar)$,
где относительная вариация $n(^{40}Ar)/n(^{36}Ar)$ примерно вдвое превышает вариацию $n(^{38}Ar)/n(^{36}Ar)$ из-за
двухкратного различия масс изотопов. 

Напротив, процессы ядерной трансформации вызывают изменения, которые не следуют этой модели. Например, процесс рождения $^{40}Ar$, изменит отношение $n(^{40}Ar)/n(^{36}Ar)$, но не $n(^{38}Ar)/n(^{36}Ar)$; процесс рождения $^{36}Ar$ привел бы к одинаковым относительным изменениям как $n(^{40}Ar)/n(^{36}Ar)$, так и $n(^{38}Ar)/n(^{36}Ar)$.

В то время как атмосферный аргон может служить обильным и однородным изотопным эталоном, отклонения от
атмосферные изотопные отношения в других источниках аргона ограничивают точность, с которой стандартный атомный вес
можно отдать за аргон. Опубликованные данные указывают на изменение атомного веса аргона в нормальных земных условиях.
материалы между 39.792 и 39.963 \cite{IUPACArgon}. Верхний конец этого интервала соответствует атомному весу
(относительная атомная масса) аргона-40, поскольку некоторые образцы минералов, богатых калием, содержат почти чистый радиогенный аргон-40.
[15]. Атомный вес чистого аргона-40, $A_r(^{40}Ar) = 39.962 383$, был округлен до $39.963$ для получения верхней
соответствует стандартному атомному весу аргона. Нижняя граница стандартного атомного веса аргона составляет от
образец настурана (урановая руда из Саскачевана, Канада), содержащий большое количество нуклеогенных изотопов \index{настуран}
$^{36}Ar$ и $^{38}Ar$ \cite{Argon1993}. Эти измерения были откалиброваны по атмосферному аргону.

Для надёжности, если предположить
оба соотношения изотопов как независимые и имеющие расширенную относительную неопределенность на 0.5 \% для согласования с  рекомендованное в 2007 г. значениями отношения изотопов в атмосферном аргоне, получаем изотоп
отношения $R(^{38}Ar/^{36}Ar) = 2.09 \pm 0.01$ и $R(^{40}Ar/^{36}Ar) = 45.2 \pm 0.22$ и атомный вес
$A_r(Ar) = 39.7931 \pm 0.0009 (k = 2)$, что дает нижнюю границу стандартного атомного веса аргона $39.7931 - 0.0009 = 39.792$. В пределах стандартного интервала атомного веса аргона измерения различных изотопов
отношения $R(^{40}Ar/^{36}Ar)$ или $R(^{38}Ar/^{36}Ar)$ с различными уровнями точности широко используются для исследований в геохронологии,
взаимодействие воды с горными породами, эволюция атмосферы и другие области \cite{IUPACArgon}. Если требуется одно значение атомного веса,
Комиссия рекомендует использовать $39.95\pm 0.16$, что соответствует аргону в воздухе с неопределенностью покрытие нормальных материалов.

Обзор благородных газов в атмосфере дан в  \cite{NobleGasesTracers}.

% Reported historical values of the standard atomic weight of argon have been [31,39]: $1902, 39.9; 1911, 39.88; 1920, 39.9; 1925, 39.91; 1931, 39.944; 1961, 39.948; 1969, 39.948 \pm 0.003;$ and $1979, 39.948 \pm 0.001.$ 

Предлагаемая ячейка в Таблице IUPAC Periodic Table of the Elements and Isotopes \cite{IPTEI} приведена на Рис.~\ref{f:Argon}.

\begin{figure}[ht] 
	\centering\small
	%	\unitlength=1mm
	\includegraphics[width=0.3\textwidth]{Figures/Argon.png}
	\caption{Представление аргона в таблице Менделеева.} 
	\label{f:Argon}
\end{figure}


\subsubsection{Свинец} \label{s:Lead}

Изотопный состав и атомный вес свинца в земных материалах различны, поскольку три его самых тяжелых стабильных изотопа являются стабильными конечными продуктами радиоактивного распада урана 
\begin{align}
	& ^{238}U \longrightarrow ^{206}Pb,  \\ 
	& ^{235}U \longrightarrow ^{207}Pb, 
\end{align}
и тория 
\begin{equation}
	^{232}Th \longrightarrow ^{208}Pb.
\end{equation}

Эти различия в соотношениях изотопов и атомных весах дают полезную информацию во многих областях науки, включая геохронологию, археологию, исследования окружающей среды и изучение окружающей среды, криминалистике.
%Хотя элементарный свинец может служить обильным и однородным изотопным эталоном, отклонения от соотношения изотопов в других случаях свинца ограничивают точность, с которой может быть указан стандартный атомный вес свинца.

\begin{figure}[ht] 
	\centering\small
	%	\unitlength=1mm
	\includegraphics[width=0.3\textwidth]{Figures/Lead.png}
	\caption{Представление свинца в таблице Менделеева.} 
	\label{f:Lead}
\end{figure}

В ходе всестороннего обзора нескольких сотен публикаций и анализов более 8000 образцов \cite{IUPACLead},
опубликованные изотопные данные показывают, что самый низкий атомный вес свинца среди обычных земных материалов составляет
$206.1462\pm 0.0028 (k = 2)$,
определен для роста фосфатного минерала монацита из люизианских
комплекс на северо-западе Шотландии, который содержит в основном $^{206}Pb$ и почти не содержит $^{204}Pb$. Самый высокий из опубликованных
атомный вес свинца $207.9351 \pm 0.0005 (k = 2)$ для монацита из микровключения. Материал также
из люизианского комплекса на северо-западе Шотландии, содержащего практически чистый радиогенный $^{208}Pb$. \index{монацит}

Используя вышеупомянутые атомные веса свинца как нижнюю и верхнюю границы интервала, стандартный атомный вес вес свинца $A_r(Pb) = [206.14,\, 207.94]$. Если требуется одно значение атомного веса, Комиссия рекомендует
использовать интервал $207.2 \pm 1.1$, что соответствует внешней оценке неопределенности
для нормальных материалов.
\begin{equation}
	A_r(Pb) = [206.14, \, 207.94]; \quad 207.2 \pm 1.1.
\end{equation}

Предлагаемая ячейка в Таблице IUPAC Periodic Table of the Elements and Isotopes \cite{IPTEI} приведена на Рис.~\ref{f:Lead} со значением середины интервала атомного веса $207.2$, данным белым цветом. 

%Reported historical values of the standard atomic weight of lead have been [31, 39]: $1902, 206.9; 1909, 207.10; 1916, 207.20; 1937, 207.21; 1961, 207.19; and 1969, 207.2 \pm 0.1$.

На Рис.~\ref{f:204LeadVariation} представлены данные по распределению изотопа $^{204}Pb$ в различных химических формах и разного происхождения.
\begin{figure}[ht] 
	\centering\small
	%	\unitlength=1mm
	\includegraphics[width=100mm]{Figures/Pb204.png}
	\caption{Вариации изотопа $^{204}Pb$ в различных объектах} 
	\label{f:204LeadVariation}
\end{figure}

Рассмотрим часть данных, относящихся к свинцу природных объектов, sedimentary rocks. Данные Табл.\ref{t:204LeadVariation} взяты из \cite{IUPACLead}.
% infPbSRchi204 = [ 0.0115, 0.0078, 0.0118, 0.0113, 0.0096]
% supPbSRchi204 = [ 0.0136, 0.016, 0.0137, 0.0131, 0.0140]
\begin{table}[h!]
	\begin{center}
		{\small
			\begin{tabular}{ccc}
				\hline
				Категория & Нижняя & Верхняя \\
				~ & граница & граница \\ 
				\hline
				Dolomite	& 0.0115 & 0.0136 \\
				Limestone	& 0.0078 & 0.0160 \\
				Marl 	& 0.0118 & 0.0137 \\
				Mudrock 	& 0.0113 & 0.0131 \\
				Phosphorite & 0.0096 & 0.0140 \\
				\hline
			\end{tabular}
		}
		\caption{Вариации изотопа $^{204}Pb$ в sedimentary rocks,  $ \chi ^{204}Pb$}
		\label{t:204LeadVariation}
	\end{center}
\end{table}

\chapter{Трудности в использовании интервальной неопределённости}

Интервальный анализ --- молодая ветвь математики, зародившаяся во второй половине прошлого века. По этой дисциплине нет устоявшейся учебной программы, популярных учебников и задачников. Она не ходит в большинство курсов, преподаваемых в университетах.

Анализ данных с интервальной неопределённостью --- ещё более молодой раздел статистики. В нём еще не утвердилась система понятий и определений, хотя методы и практические подходы разрабатываются и используются уже несколько десятилетий.

В связи с этим, даже у математиков-профессионалов не всегда есть понимание предмета и практики использования интервальных подходов. Тем, у кого математика и анализ данных не являются сферой деятельности, ещё сложнее разобраться в том, как понимать и обращаться с интервальными данными.  

У Комиссии по содержанию изотопов и атомным весам (CIAAW) Международного Союза теоретической и прикладной химии (IUPAC) тоже есть эти проблемы.

Рассмотрим несколько документов, свидетельствующих об этих трудностях и обсудим поднятые в них вопросы. 

%\bibitem{IUPACUncertainty}
A. Possolo, A.M.H. van der Veen, J. Meija and D.B. Hibbert
<<Interpreting and propagating the uncertainty of the standard atomic weights>> (IUPAC Technical Report). \\
Pure Appl. Chem. 2018; 90(2): 395–424 \cite{IUPACUncertainty}

%\bibitem{IUPACCosensus}
T.B. Coplen, N.E. Holden, T. Ding, H.A.J. Meijer, J. Vogl, and X. Zhu, <<The Table of Standard Atomic Weights-An exercise in consensus>>, \\ Rapid Commun Mass Spectrom. 2022; 36:e8864. \cite{IUPACCosensus}


\section{Интерпретация и распространение неопределенности стандартных атомных весов}

Проблемы в интерпретации неопределённостей атомных весов и их использования в интервальной форме рассмотрены в специальном документе \cite{IUPACUncertainty}. 

<<В 2009 году Комиссия по содержанию изотопов и атомным весам (CIAAW) Международного
Союза теоретической и прикладной химии (IUPAC) ввела обозначение интервала для выражения стандартной атомной
массы элементов, изотопный состав которых в природе значительно варьируется. Однако стало очевидным
что дополнительное руководство было бы полезным в отношении того, как репрезентативные значения должны быть получены из этих интервалов,
и о том, как связанная неопределенность должна быть охарактеризована и распространена на родственные величины, например
как относительные молекулярные массы. 

Назначение подходящих распределений вероятностей интервалам атомного веса
согласуется с целью CIAAW подчеркнуть изменчивость значений атомного веса в природе.
Эти распределения, однако, не предназначены для отражения естественной изменчивости численности различных
изотопов в земной коре или в любой другой среде. Скорее, они передают состояние знания о
элементный состав «обычных» материалов вообще или о конкретных классах таких материалов. в
отсутствие подробных сведений об изотопном составе материала, или когда такие детали могут быть безопасно
игнорируется, распределение вероятностей, присвоенное стандартным интервалам атомного веса, может быть принято прямоугольным
(или униформа). Этот выбор модели является разумным и удобным выбором по умолчанию, когда представитель
значение атомного веса и связанная с ним неопределенность необходимы в расчетах, связанных с атомным и относительным
молекулярные массы. 

Когда информация о происхождении материала или другая информация о
необходимо учитывать изотопный состав, тогда это распределение может быть неравномерным. Мы представляем
несколько примеров того, каким может быть распределение вероятностей атомного веса или относительной молекулярной массы охарактеризована, а также как ее можно использовать для оценки связанной с ней неопределенности.>>

\section{Modeling the dispersion of atomic-weight values}

Table 1: Classes of sources of boron, and observed intervals of $\delta_{\tt SRM951}(^{11}/^{10}B)$ as listed by Coplen et al. [23, table 8] and Coplen and Shrestha [24, table 3].
% [23, table 8] 
% [24, table 3].

\begin{tabular}{cccc}
	Номер & Class &
	%δSRM951,j (11B/10B)/‰
	min & max \\
	\hline
\end{tabular}
1 Marine Seawater +38.4 +40.4 \\
2 Marine Evaporated seawater +36.5 +58.5\\
3 Marine Hydrothermal fluids +30.0 +36.8\\
4 Marine Evaporite minerals +18.2 +31.7\\
5 Marine Carbonates +4.0 +32.2\\
6 Non-marine Rain water +0.8 +35.0\\
7 Non-marine Brines, surface ground waters -21.3 +59.2\\
8 Non-marine Hydrothermal fluids -9.3 +29.1\\
9 Non-marine Evaporite minerals -31.3 +7.3\\
10 Rocks Igneous -17.0 -1.7\\
11 Rocks Metamorphic -34.2 -22.0\\
12 Rocks Sediments -17.0 +26.2\\
13 Organic Boron -12.0 +29.3\\

Boron isotopic reference material SRM 951 is certified by NIST as comprising 19.827 \% of 10B and 80.173 \% of 11B (amount-of-substance
fractions), both with expanded uncertainty (for 95 \% coverage) of 0.013 \%, hence RSRM951(11B/10B) = 4.044, with expanded
uncertainty (for 95 \% coverage) of 0.003.
\index{NIST SRM 951, изотопный эталон бора}

\begin{figure}[ht] 
	\centering\small
	\unitlength=1mm
	{\includegraphics[width=80mm]{Figures/BoronTable1.png}} 
	\caption {Estimate of the probability density of $Ar(B)$ (solid blue curve), and probability density of a rescaled and shifted beta distribution 	(dotted black line) that approximates that estimate. ). \cite{IUPACUncertainty}.} 
	\label{f:BoronTable1}
\end{figure}

Fig. 1: Estimate of the probability density of $Ar(B)$ (solid blue curve), and probability density of a rescaled and shifted beta distribution
(dotted black line) that approximates that estimate. This approximant is concentrated on the standard atomic weight interval for boron
and has the same mean and standard deviation of the sample of values of $A_r(B)$ produced by Algorithm S. The multiple peaks and
valleys in the blue curve are due to the fact that the corresponding distribution is a mixture of 13 different rectangular distributions.
The horizontal axis has the same endpoints as the standard atomic weight interval for boron. The total area under any curve representing
a probability density, or a smooth histogram that estimates a probability density, is always 1. The area of the region shaded pink
amounts to 95 \% of the area under the solid blue curve, therefore, its projection onto the horizontal axis (marked with a horizontal red
line segment immediately above the horizontal axis) is a 95 \% coverage interval for the true value of $A_r(B)$ .

\section{Uncertainty evaluation for relative molecular masses}

\subsection{Carbon dioxide}

\begin{figure}[ht] 
	\centering\small
	\unitlength=1mm
	{\includegraphics[width=80mm]{Figures/carbondioxide.png}} 
	\caption {Probability density characterizing the uncertainty associated with the relative molecular mass of CO2 \cite{IUPACUncertainty}.} 
	\label{f:carbondioxide}
\end{figure}

: (i) when nothing
is assumed known about the provenance of the compound (squat and blue trapezoid, with mean 44.009 40 and standard deviation
0.000 72); \\
(ii) assuming atmospheric origin (tall and red curve), with mean 44.010 78 and standard deviation 0.000 03.




\subsection{Testosterone}

\begin{figure}[ht] 
	\centering\small
	\unitlength=1mm
	{\includegraphics[width=80mm]{Figures/Testosteron.png}} 
	\caption { Probability densities of the relative molecular mass of human (trapezoidal blue line) and synthetic (bell-shaped red line) testosterone, and corresponding averages (blue and red dots). \cite{IUPACUncertainty}.} 
	\label{f:Testosteron}
\end{figure}

Fig 6 \\

\section{Mixtures of molecules}


H C N O \\
4 1 0 0 \\
6 2 0 0 \\
8 3 0 0 \\
10 4 0 0 \\
12 5 0 0 \\
0 0 2 0 \\
0 1 0 2 \\
0 0 0 2 \\


\begin{tabular}{ccccccccc}
	~ & CH4 & C2H6 & C3H8 & C4H10 & C5H12 & N2 & CO2 & O2 \\
	\hline
\end{tabular}

CH4 1.000 0.994 0.989 0.986 0.984 0.000 0.707 0.000 \\
C2H6 0.994 1.000 0.999 0.998 0.997 0.000 0.745 0.000\\
C3H8 0.989 0.999 1.000 1.000 0.999 0.000 0.756 0.000\\
C4H10 0.986 0.998 1.000 1.000 1.000 0.000 0.762 0.000\\
C5H12 0.984 0.997 0.999 1.000 1.000 0.000 0.765 0.000\\
N2 0.000 0.000 0.000 0.000 0.000 1.000 0.000 0.000\\
CO2 0.707 0.745 0.756 0.762 0.765 0.000 1.000 0.595\\
O2 0.000 0.000 0.000 0.000 0.000 0.000 0.595 1.000\\

\section{Conclusions}

The assignment of probability distributions to the values between the endpoints of the intervals that represent
the standard atomic weights of the elements in “normal” materials, is consistent with the definition
of standard atomic weights that the CIAAW has adopted. It is also a necessary step to use such intervals in
subsequent calculations, for example of relative molecular masses and of chemical compositions expressed
in terms of amounts-of-substance.

Such probability distributions generally do not aim to describe the natural abundances of the atomic
weights in “normal” materials. Instead, they convey what is known about the dispersion of the atomic
weights, either in all “normal” materials generally, or in specific materials. To this extent, these distributions
characterize both the knowledge about the natural variability and the measurement uncertainty, and may be
used in uncertainty propagation exercises.

If the knowledge in hand is no greater than what a standard atomic weight – for example [12.0096,
12.0116] for carbon – already expresses, then a rectangular distribution should be used by default. The corresponding
evaluation of uncertainty may be conservative (that is, it may be larger than it could be if more
information were used), yet it serves the very general purpose of characterizing uncertainty for uncertainty
calculations that do not require the utmost precision in the values of the atomic weights.

If specific information is available about the isotopic composition of the material of interest (say, because
its provenance is known, or because its isotopic composition was determined experimentally), then the
uncertainty associated with the corresponding value of the atomic weight may be substantially smaller than
what the rectangular distribution on the standard atomic weight interval implies.



%%%%%%%%%%%%%%%%%%%%%%%%%%%%%%%%%%%%%%%%%%%%%%%%%%%%%%%%%%%%%%%%%%%%%%%%%%%%%%%%%%%%%%%%

\addcontentsline{toc}{chapter}{Предметный указатель}
\raggedright\small\printindex   

%%%%%%%%%%%%%%%%%%%%%%%%%%%%%%%%%%%%%%%%%%%%%%%%%%%%%%%%%%%%%%%%%%%%%%%%%%%%%%%%%%%%%%%% 

\end{document} 


%%%%%%%%%%%%%%%%%%%%%%%%%%%%%%%%%%%%%%%%%%%%%%%%%%%%%%%%%%%%%%%%%%%%%%%%%%%%%%%%%%%%%%













%%%%%%%%%%%%%%%%%%%%%%%%%%%%%%%%%%%%%%%%%%%%%%%%%%%%%%%%%%%%%%%%%%%%%%%%%%%%%%%%%%%%%%%%
\addcontentsline{toc}{chapter}{Обозначения}
\chapter*{Обозначения}\label{Notation}

\begin{tabular}{ll}
	:= & левая часть равенства есть обозначение для правой \\
	\& & логическая конъюнкция, связка <<и>> \\
	$\Longrightarrow$ & логическая импликация \\
	$\Longleftrightarrow$ &логическая равносильность \\
	$\rightarrow$ & отображение множеств \\
	$\mapsto$ & правило сопоставления элементов при отображении \\
	$\leftarrow$ & оператор присваивания в алгоритмах \\
	$\circ$ & знак композиции отображений \\
	$\emptyset$ & пустое множество \\
	$x \in X $ & элемент $x$ принадлежит множеству $X$  \\
	$x \notin X$ & элемент $x$ не принадлежит множеству $X$ \\ 
	$X \ni x$ & множество $X$ содержит элемент $x$\\
	$X \not \owns x$ & множество $X$ не содержит элемент $x$\\
	$X \cup Y$ & объединение множеств $X$ и $Y$ \\
	$X \cap Y$ & пересечение множеств $X$ и $Y$\\
	$ X \ Y$ & разность множеств $X$ и $Y$\\
	$X \subseteq $ & множество $X$ есть подмножество множества $Y$ \\
	$X \subset $ & множество $X$ есть собственное подмножество множества $Y$ \\
	$X \times Y$ & прямое декартово произведение множеств $X$ и $Y$\\
	$\mathbb{N}$ & множество натуральных чисел\\
	$\mathbb{R}$ & множество действительных (вещественных) чисел\\
	%	$\mathbb{R}_{+}$ & множество неотрицательных вещественных чисел \\
	$\mathbb{IR}$ & классическая интервальная арифметика \\
	$\mathbb{ID}$  & множество интервалов, содержащихся в $ D \subseteq \mathbb{R}^n $\\
	$\mathbb{KR}$ & полная интервальная арифметика Каухера \\
	$\mathbb{R}^n $ & множество вещественных $n$-мерных векторов \\
	$\mathbb{IR}^n $ & множество вещественных $n$-мерных векторов c элементами из $\mathbb{IR}$\\
	$\mathbb{KR}^n $ & множество вещественных $n$-мерных векторов c элементами из $\mathbb{KR}$\\
	%	$\mathbb{R}^{m \times n}$ & множество вещественных $m \times n$-матриц \\
	%	$\mathbb{IR}^{m \times n}$ & множество  $m \times n$-матриц с элементами из $\mathbb{IR}$ \\
	%	$\mathbb{KR}^{m \times n}$ & множество  $m \times n$-матриц с элементами из $\mathbb{KR}$ \\
\end{tabular}




\begin{tabular}{ll}
	$\mathcal{EF}$ & семейство элементарных функций \\	
	$\sgn \ x$ & знак вещественного числа $x$ \\
	$x^{+}, x^{-}$& положительная и отрицательная части числа $x$ \\
	$\sgn \mbf{a}$ & знак интервала $\mbf{a}$ \\
	$\mbf{a}^{+}, \mbf{a}^{-}$ & положительная и отрицательная части интервала $\mbf{a}$ \\
	$\un{\mbf{a}}, \inf \mbf{a}$ & левый конец интервала $\mbf{a}$ \\
	$\ov{\mbf{a}}, \sup \mbf{a}$ & правый конец интервала $\mbf{a}$ \\
	$ \| \mbf{a} \| $ & абсолютная величина (магнитуда) интервала $\mbf{a}$ \\	
	$\abs \mbf{a} $ & интервальное расширение функции модуля \\
	$\langle \mbf{a} \rangle $ & мигнитуда интервала $\mbf{a}$ \\
	$\langle \mbf{A} \rangle $ & компарант интервальной матрицы $\mbf{A}$ \\ 
	$\m \mbf{a}$ & середина (медиана) интервала $\mbf{a}$ \\
	${\tt wid}\ \mbf{a}$ & ширина интервала $\mbf{a}$ \\
	$\r \mbf{a}$ & радиус интервала $\mbf{a}$ \\
	${\tt dev} \ \mbf{a}$ & отклонение интервала $\mbf{a}$  от нуля \\
	${\tt dual} \ \mbf{a}$ & дуальный (двойственный) к  $\mbf{a}$  интервал \\
	${\tt opp} \ \mbf{a}$ & алгебраически противоположный к $\mbf{a}$ интервал \\
	${\tt inv} \ \mbf{a}$ & алгебраически обратный к $\mbf{a}$ интервал \\
	${\tt pro} \ \mbf{a}$ & правильная проекция интервала $\mbf{a}$\\
	${\tt vert} \ \mbf{a}$ & множество крайних точек интервала $\mbf{a}$\\
	$\ominus$ & <<внутреннее>> интервальное вычитание \\
	$\oslash$ & <<внутреннее>> интервальное деление \\
	$\chi (\mbf{a})$ &функционал Рачека от интервала $\mbf{a}$ \\
	$\varXi_{uni}$ & объединённое множество решений \\
	$\varXi_{tol}$ & допусковое множество решений \\
	$\varXi_{ctl}$ & управляемое множество решений \\
	$\varXi_{\alpha \beta} $ & множество $AE$-решений типа $\alpha \beta$ \\
	$\varXi_{\mathcal{A} \beta} $ & множество $AE$-решений типа $\mathcal{A}  \beta$ \\
	%	$\mbf{A}^{-1}$ & обратная интервальная матрица\\
	%	$\mbf{A}^{c}$ & характеристическая матрица ИСЛАУ \\
	%	$\mbf{b}^{c}$ & характеристический вектор правой части ИСЛАУ \\
	${\tt dist}$ & метрика в интервальных пространствах \\
	${\tt Dist}$ & мультиметрика в интервальных пространствах \\
	${\tt sti}$ &  стандартное погружение  $\mathbb{KR}^n$ в $\mathbb{R}^{2n}$ \\
	${\tt ran}(f,\mbf{X}) $ & область значений функции $f$ на множестве $\mbf{X}$ \\
	$f^{\angle}(\tilde{x}, x)$ & наклон функции $f$ между точками $\tilde{x}$ и $x$ \\
	%	${\tt hyp}$ & подграфик функции $f$ \\
	%	${\tt epi}$ & надграфик функции $f$ \\
	%	$\partial f$ & субдифференциал функции f \\
	%	$\partial X$ &граница множества $X$ \\
	%	${\tt int} \ X$   & топологическая внутренность множества $X$ \\
	%	${\tt cl} \ X$   & топологическое замыкание множества $X$ \\
	%	${\tt ch} \ X$   & выпуклая оболочка множества $X$ \\
	$\square X$   & интервальная оболочка множества $X$ \\
	$\wedge$   & минимум в частично упорядоченном множестве \\
	$\vee$   & максимум в частично упорядоченном множестве \\
	И & условная решёточная операция \\
	$I$ & единичная матрица соответствующих размеров \\
	$Q^{\sim} $ & знаково-блочная матрица для матрицы $Q$ \\
\end{tabular}

\begin{tabular}{ll}
	$\| \cdot \|$ & векторная или матричная норма \\
	$\| \cdot \|_1$ & 1-норма векторов или подчинённая 1-норма матриц \\
	$\| \cdot \|_1$ & 2-норма векторов или подчинённая 2-норма матриц \\
	$\| \cdot \|_1$ & $\infty$-норма векторов или подчинённая $\infty$-норма матриц \\
	$\lambda(A)$ & собственное значение матрицы $A$ \\
	$\rho(A)$ & спектральный радиус матрицы $A$ \\
	$\sigma(A)$ &  сингулярное число матрицы $A$ \\
	${\tt diag} (A)$ &диагональная  матрица \\%$n \times n$-матрица	с элементами $z_1, \ldots , z_n$ по главной диагонали \\
	$\mathcal{N}(\mbf{X}, \tilde{x})$ & интервальный оператор Ньютона \\
	$\mathcal{K}(\mbf{X}, \tilde{x})$ & интервальный оператор Кравчика \\
	$\mathcal{H}(\mbf{X}, \tilde{x})$ & интервальный оператор Хансена-Сенгупты
\end{tabular}

\bigskip

Используемая система обозначений следует, в основном, неформальному международному
стандарту на обозначения в интервальном анализе, выработанному в 2002 году. В настоящее время его текст доступен в Интернете на многих сайтах, посвящённых интервальным вычислениям (к примеру, на отечественном \url{http://www.nsc.ru/interval}).

Интервалы и другие интервальные величины (векторы, матрицы и др.) всюду в тексте обозначаются жирным математическим шрифтом, например, $\mbf{A}, \mbf{B}, \mbf{C}, \ldots, \mbf{x}, \mbf{y}, \mbf{z}$, тогда как неинтервальные (точечные) величины никак специально не выделяются. Арифметические операции с интервальными величинами --- это операции
соответствующих интервальных арифметик: либо классической интервальной арифметики $\mathbb{IR}$, либо полной интервальной арифметики Каухера $\mathbb{KR}$.


\begin{table}
	\caption{Нижние и верхние границы атомных масс лития в различных категориях веществ}	
	\label{IsotopeDataLithium}
	{\small
		\begin{tabular}{llcc}
			Category & Subcategory & $\un{\mbf{M}}(H)$ & $\ov{\mbf{M}}(H)$ \\
			\hline
			Marine Sources & Sea water&	6.94223 & 6.94233 \\
			Marine Sources 	& Hydrothermal fluids	& 6.94021	& 6.94079 \\
			Marine Sources &	"Foraminifera and carbonic sediments" &	6.93931 &	6.94286 \\
			%Marine Sources &	"Foraminifera and carbonic sediments" &	6.93931 &	6.94286 \\
			Marine Sources	& Brines and pore water	& 6.94009	& 6.94378 \\
			Non-Marine Sources & Surface Water & 6.94110 & 6.94236 \\
			Non-Marine Sources & Ground and thermal water & 6.93867	& 6.94072 \\
			Non-Marine Sources	& Contaminated ground water	& 6.94072	& 6.95875 \\
			Lithium in rocks &	Basalt (unaltered) &	6.94026	& 6.94050 \\
			Lithium in rocks & Basalt (altered)	& 6.93988 &	6.94096 \\
			Lithium in rocks & Rhyolite	& 6.93978 & 6.93978 \\
			Lithium in rocks & Granite & 6.94002 & 6.94080 \\
			Lithium in rocks & Limestone & 6.93978 & 6.94230 \\
			Phosphates & LiAlFPO4 (amblygonite) &	6.93910	& 6.94054 \\
			Silicates & LiAl(SiO3)2 (spodumene)	& 6.93913 & 6.94084 \\
			Silicates &	KLiAl2Si3O10 (lepidolite) & 6.93956	& 6.94053 \\
			Reagents & 6Li depleted compounds & 6.96177	& 6.99593  \\
			Reagents & Other reagents & 6.93924	& 6.94160 \\
			\hline
		\end{tabular}
	}
\end{table}
