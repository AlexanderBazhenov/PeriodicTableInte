\documentclass[a5paper,openany]{book}
%\documentclass[11pt]{article}

\usepackage{cmap}  
\usepackage[utf8]{inputenc}
\usepackage[T2A]{fontenc} 
\usepackage{index} 
\usepackage[russian]{babel} 
\usepackage{amsmath,amssymb} 
\usepackage{euscript,upref}  
\usepackage{array,longtable}
\usepackage{indentfirst} 
\usepackage{graphicx} 
\usepackage{caption} 
\usepackage{url}
\usepackage{todonotes}
%\usepackage{circuitikz}
\usepackage{xcolor}
\usepackage{wasysym}
%%%%%%%%%%%%%%%%%%%%%%%%%%%%%%%%%%%%%%%%%%%%%%%%%%%%%%%%%%%%%%%%%%%%%%%%%%%%%%%%%%%%%%%%  
% URL проекта - https://ru.overleaf.com/project/5e954c887ac0ac0001d54ece   
%%%%%%%%%%%%%%%%%%%%%%%%%%%%%%%%%%%%%%%%%%%%%%%%%%%%%%%%%%%%%%%%%%%%%%%%%%%%%%%%%%%%%%%%
%234567890123456789012345678901234567890123456789012345678901234567890123456789012345678
%%%%%%%%%%%%%%%%%%%%%%%%%%%%%%%%%%%%%%%%%%%%%%%%%%%%%%%%%%%%%%%%%%%%%%%%%%%%%%%%%%%%%%%%

\textwidth=114truemm
\textheight=165truemm
\oddsidemargin=-1cm
\evensidemargin=\oddsidemargin
\topmargin=-1cm
\sloppy

\pagestyle{plain}
%\mathsurround=1pt
%\tolerance=400
%\hfuzz=2pt
\makeindex

\captionsetup{font=small,labelsep=period,margin=7mm} 

%%%%%%%%%%%%%%%%%%%%%%%%%%%%%%%%%%%%%%%%%%%%%%%%%%%%%%%%%%%%%%%%%%%%%%%%%%%%%%%%%%%%%%%%
%
%           Определения новых команд и макросов
%    
\newcommand{\mbf}[1]{\protect\text{\boldmath$#1$}}
\newcommand{\mbb}{\mathbb}
\newcommand{\mrm}{\mathrm}
\newcommand{\mcl}{\mathcal}
\newcommand{\msf}{\mathsf}
\newcommand{\eus}{\EuScript}
\newcommand{\ov}{\overline}
\newcommand{\un}{\underline}
\newcommand{\m}{\mathrm{mid}\;}
\newcommand{\w}{\mathrm{wid}\;}
\newcommand{\Uni}{\mathrm{Uni}\,}
\newcommand{\Tol}{\mathrm{Tol}\,} 
\newcommand{\Uss}{\mathrm{Uss}\,} 
\newcommand{\Ab}{(\mbf{A}, \mbf{b})}
\newcommand{\sgn}{\mathrm{sgn}\;}
\newcommand{\abs}{\mathrm{abs}\;}
\newcommand{\ran}{\mathrm{ran}\,}
\newcommand{\pro}{\mathrm{pro}\,}
\newcommand{\dual}{\mathrm{dual}\,} 
\newcommand{\dist}{\mathrm{dist}\,} 
\newcommand{\const}{\mathrm{const}}
\newcommand{\NExt}{_{\scalebox{0.57}{$\natural$}}}
\newcommand{\ih}{\scalebox{0.67}[0.87]{$\Box$\hspace*{1pt}}}

\renewcommand{\r}{\mathrm{rad}\;} 

\newcommand{\ii}[2]{\hspace*{-\arraycolsep}
	%	\begin{matrix}[t]{c}\\[-14pt]
	\begin{array}[t]{c}\\[-16pt]
		\mbox{\huge\textsf{И}}^{\scriptstyle #1}\\[-6pt]
		{\scriptstyle #2\phantom{A}}\end{array} \hspace{-0.8\arraycolsep}}
\newcommand{\Prop}{\noindent\textbf{Предложение}.~}
\newcommand{\Cons}{\noindent\textbf{Следствие}.~}
%%%%%%%%%%%%%%%%%%%%%%%%%%%%%%%%%%%%%%%%%%%%%%%%%%%%%%%%%%%%%%%%%%%%%%%%%%%%%%%%%%%%%%%%

\renewcommand{\textfraction}{0}
\renewcommand{\topfraction}{1}
\renewcommand{\bottomfraction}{1} 
\renewcommand{\indexname}{Предметный указатель}

%%%%%%%%%%%%%%%%%%%%%%%%%%%%%%%%%%%%%%%%%%%%%%%%%%%%%%%%%%%%%%%%%%%%%%%%%%%%%%%%%%%%%%%%
%
%           Определение счётчиков 
%  
\newcounter{DefNum}[section]
\setcounter{DefNum}{0} 
\newcounter{ExmpNum}[section]
\setcounter{ExmpNum}{0}
\newcounter{IncluDefi}
\newcounter{IreneExmp} 
\newcounter{BazheExmp}
\newcounter{ThNum}[section]
\setcounter{ThNum}{0} 
%%%%%%%%%%%%%%%%%%%%%%%%%%%%%%%%%%%%%%%%%%%%%%%%%%%%%%%%%%%%%%%%%%%%%%%%%%%%%%%%%%%%%%%%
%
%           Определение необходимых окружений          
%
\newtheorem{definition}{Определение}[section] 
\newenvironment{example}% 
{\par\addvspace{\medskipamount}\addtocounter{ExmpNum}{1} 
	\noindent\textbf{Пример {\thesection}.\arabic{ExmpNum}.}}% 
{\hfill$\blacksquare$\par\medskip} 

\newenvironment{theoremRU}% 
{\par\addvspace{\medskipamount}\addtocounter{ThNum}{1} 
	\noindent\textbf{Теорема {\thesection}.\arabic{ThNum}.}\sl}%
%{\hfill$\blacksquare$\par\medskip}

\newenvironment{Proof}% 
{\par\addvspace{\medskipamount}
	\noindent\textbf{Доказательство.}}%
{\hfill$\blacksquare$\par\medskip} 

%%%%%%%%%%%%%%%%%%%%%%%%%%%%%%%%%%%%%%%%%%%%%%%%%%%%%%%%%%%%%%%%%%%%%%%%%%%%%%%%%%%%%%%%  
%For contents 
%\renewcommand{\l@section}{\@dottedtocline{1}{0.5em}{1.5em}}
%\renewcommand{\l@subsection}{\@dottedtocline{1}{2.5em}{2.0em}}
%\makeatother
%\setlength{\marginparwidth}{2cm}

%%%%%%%%%%%%%%%%%%%%%%%%%%%%%%%%%%%%%%%%%%%%%%%%%%%%%%%%%%%%%%%%%%%%%%%%%%%%%%%%%%%%%%%%%%%




%\title{Естественно научные и технические мотивации интервального анализа.}

%\author{А.Н.\,Баженов}

%%%%%%%%%%%%%%%%%%%%%%%%%%%%%%%%%%%%%%%%%%%%%%%%%%%%%%%%%%%%%%%%%%%%%%%%%%%%%%%%%%%%%%%%

\begin{document}
	

\begin{center}
	\hfill \break
	\Large{Министерство образования науки Российской Федерации}\\
	\hfill \break
	\Large{	САНКТ-ПЕТЕРБУРГСКИЙ \\
		ПОЛИТЕХНИЧЕСКИЙ УНИВЕРСИТЕТ ПЕТРА ВЕЛИКОГО}\\ 
	%\hfill \break
	%\hfill \break
	%	\large{Институт прикладной математики и механики}\\
	%\hfill \break		
	%\hfill \break
	%	\large{Кафедра «Прикладная математика»}\\
	%\hfill \break
	%\hfill \break
	\hfill \break
	\hfill \break
	
	
	
	
	
	\Large{\it А.Н.Баженов\\
		\hfill \break		\hfill \break}
	{\huge 	Интервальная таблица Менделеева и изотопы на Земле.}\\
	\hfill \break \hfill \break
	\Large{	Учебное пособие	
	}\\
	\hfill \break \hfill \break
\end{center}

\hfill \break
%\hfill \break
%\hfill \break
\begin{center}\Large{Санкт-Петербург \\
		\hfill \break
		2022} \end{center}
\thispagestyle{empty} % выключаем отображение номера для этой страницы

\newpage
УДК 519.9
Р32

%Р е ц е н з е н т ы:\\
%Доктор физико-математических наук, профессор НГУ С.П.Шарый
А в т о р :\\
А.Н.Баженов.
Интервальная таблица Менделеева и изотопы на Земле. – СПб., 2022. – 83 с.
\hfill \break

Учебное пособие соответствует образовательному стандарту высшего
образования Федерального государственного автономного образовательного учреждения высшего образования «Санкт-Петербургский политехнический университет Петра Великого» по направлению подготовки бакалавров 01.03.02 <<Прикладная математика и информатика>>, по дисциплине «Интервальный анализ» и по направлению подготовки  магистров 01.04.02 <<Прикладная математика и информатика>> по дисциплине «Анализ данных с интервальной неопределённостью».


В издании рассмотрена современная версия таблица Менделеева с интервальными значениями атомных весов, разработанная Международным союзом теоретической 
и прикладной химии IUPAC. Также приводятся сведения об изотопных распределениях в живой и неживой природе на Земле.

Пособие адресовано всем, кто интересуется применением математики к решению практических задач.


Материал апробирован в учебных курсах для студентов 
СПбПУ Петра Великого и аспирантов первого года обучения ФТИ им. А.Ф.Иоффе РАН.
\hfill \break
\hfill \break
\hfill \break
\copyright 
Санкт-Петербургский политехнический университет Петра Великого, 2022

%	\maketitle 

% УДК  518+519.4+519.6, 658.562.012+519.254  
%\\
%\textbf{\textit{Ключевые слова: }} измерение, неопределённость, интервальный анализ.
%\end{abstract}

%%%%%%%%%%%%%%%%%%%%%%%%%%%%%%%%%%%%%%%%%%%%%%%%%%%%%%%%%%%%%%%%%%%%%%%%%%%%%%%%%%%%%%%%

\tableofcontents      %  Содержание  

%%%%%%%%%%%%%%%%%%%%%%%%%%%%%%%%%%%%%%%%%%%%%%%%%%%%%%%%%%%%%%%%%%%%%%%%%%%%%%%%%%%%%%%%


\chapter*{Введение}
\addcontentsline{toc}{chapter}{Введение}   

\bigskip
%В издании приводятся примеры применения интервальных величин в науке и технике. 

В пособии дается информация о современной версия таблица Менделеева с интервальными значениями атомных весов, разработанная Международным союзом теоретической 
и прикладной химии IUPAC. Также приводятся сведения об изотопных распределениях в живой и неживой природе на Земле.

Издание задумано как дополнение к курсам лекций автора по интервальному анализу для студентов кафедры <<Прикладная математика>> Института Прикладной Математики и Механики СПбПУ и аспирантов первого года обучения ФТИ им. А.Ф.Иоффе РАН. В значительной степени тематика этого курса представлена в учебном пособии автора <<Интервальный анализ. Основы теории и учебные примеры>> \cite{Bazhenov2020} и << Обобщение мер совместности для анализа данных с интервальной неопределённостью>> \cite{Bazhenov2022}.

Для углубленного знакомства с интервальным анализом и интервальной статистикой рекомендуются фундаментальная монография С.П.\,Шарого <<Конечномерный интервальный анализ>> \cite{SharyBook} и коллективный труд А.Н.\,Баженова, С.И.\,Жилина, С.И.\,Кумкова и С.П.\,Шарого <<Обработка и анализ данных с интервальной неопределённостью>> \cite{BookIntStat}.


\chapter{Теоретические сведения}

В этой главе даются неоходимые сведения об интервальном анализе и интервальной статистике (иначе --- анализе данных с интервальной неопределённостью). 

\section{Интервальный анализ}

Рассмотрим причины, мотивирующие на использование интервалов.  
Обратимся сначала к практическому опыту. Мы многое оцениваем <<сверху>> и  <<снизу>>. Например, можно оценить продолжительность работы: <<1-2 часа>>. Высота дерева составляет <<4-5 метров>>. На полке можно разместить <<10-15 книг>>.

Перейдем от бытовых оценок с более основательным.
Математические мотивации различных интервальных арифметик подробно рассмотрены в книге \cite{SharyBook}, а сейчас мы затронем некоторые математические понятия, которые нам пригодятся при обсуждении материала.


\subsection{Базовые понятия интервального анализа}

Забегая вперёд, введём понятие интервала.
\emph{Интервалом} вещественной оси $[a,b]$ называется множество всех чисел, расположенных между заданными числами  $a$ и $b$ включая их самих, т.е. \index{интервал}
\begin{equation*}
[a,b]:= \{x\, \in \,\mathbb{R}\ |\ a \leq x \leq b \}.
\end{equation*}
При этом $a$ и $b$  называются концами интервала.

Интервальную величину принято обозначать жирным шрифом, например, $\mbf{x}$. Левую границу интервала подчёркивают снизу, а правую --- сверху. Границы берутся в квадратные скобки, что передаёт идею интервала  как отрезка вещественной оси.
\begin{equation*}
\mbf{x} = [\ \un{\mbf{x}}, \ov{\mbf{x}} \ ].
\end{equation*}
Важнейшими характеристиками интервала являются его \emph{середина} (центр) 
\begin{equation*}
\textstyle\index{середина интервала} 
\m\mbf{a} = \frac{1}{2}(\ov{\mbf{a}} + \un{\mbf{a}}),
\end{equation*}
и его \emph{радиус} 
\begin{equation*}
\textstyle\index{радиус интервала} 
\r\mbf{a} = \frac{1}{2}(\ov{\mbf{a}} - \un{\mbf{a}}).
\end{equation*} 

Радиус интервала является мерой абсолютного рассеяния точек интервала. При описании относительной погрешности в интервальном анализе приходится использовать разные меры.   \index{относительная ширина интервала} 

Полезной характеристикой интервала является так называемый функционал Рачека $\chi$: \index{функционал Рачека} 
\begin{equation*} 
\chi(\mbf{a}) = 
\left\{ \ 
\begin{array}{ll}
\un{\mbf{a}}/\ov{\mbf{a}}, & \text{ если } \;\un{\mbf{a}}\leq\ov{\mbf{a}},\\[1mm] 
\ov{\mbf{a}}/\un{\mbf{a}}, & \text{ иначе. } 
\end{array}
\right. 
\end{equation*} 
Он характеризует <<относительную узость>> интервала.

Множество всех интервалов из $\mbb{R}$ обозначается символом $\mbb{IR}$. \index{классическая интервальная арифметика}
Используемая система обозначений следует неформальному международному стандарту на обозначения 
в интервальном анализе, выработанному в 2002--2010 годах \cite{InteNotation}. 

Неформально можно сказать так: интервалы предназначены для величин, для которых существуют двусторонние ограничения. Можно также говорить об интервальных оценках.

\subsection{Особенности интервальной арифметики}

Важной особенностью интервальной арифметики является учёт неопределёности выполнения арифметических операций. В частности, при при последовательном выполнении сложения и вычитания получается не точно 0, а величина, содержащая 0:
\begin{equation*} 
[1,2]-[1,2] = [-1,1] \ni 0.
\end{equation*}
Таким образом, производится двустороняя оценка величины результата прямого и обратного действия. Многократное повторение этой операции приводит к увеличению границ результата
\begin{equation*} 
\sum_{i=1}^{n} \left(  [1,2]-[1,2] \right) = n \cdot [-1,1].
\end{equation*}
То есть, имеет место эффект нарастания <<снежного кома>>, или <<обёртывания>>. Такое свойство классической интервальной арифметики отражает факт <<внешнего>> оценивания множества решений задачи. \index{эффект <<обёртывания>>} \index{внешнее оценивание}

\subsection{Интервальная арифметика Каухера}

Помимо наиболее естественного понимания интервала как отрезка вещественной оси, существуют и более сложные конструкции. В частности, очень важна полная интервальная арифметика или арифметика Каухера. Она обобщает обычную интервальную арифметику на случай, когда у интервала есть <<направление>>. Именно, в этом случае концы интервала не обязательно упорядочены от меньшего к большему. Такое свойство даёт дополнительные возможности, которые мы обсудим далее. Обозначается такая арифметика символом $\mbb{KR}$. \index{интервальная арифметика Каухера}

Символически, можно представить соотношение арифметик следующим образом
$$ \mbb{R} \subseteq \mbb{IR} \subseteq \mbb{KR}. $$
Если концы интервалов совпадают, имеем обычную вещественную арифметику.

В арифметике Каухера содержательный смысл имеет операция замены порядка следования концов интервала, при которой получается интервал, дуальный исходному: \index{дуальный интервал}
\begin{equation*}
\label{Dualization}
\dual\mbf{a} := [\;\ov{\mbf{a}},\,\un{\mbf{a}}\;].
\end{equation*}
В частности, дуализация даёт возможность получать при интервальных операциях точечные значения, или \emph{внутреннюю оценку}:
\begin{equation*}
[1,2] \ominus [1,2]=0.
\end{equation*}
Символ $\ominus$ соответствует так называемому \emph{алгебраическому вычитанию}. \index{алгебраическое вычитание} 

Аналогично для деления имеем внешние и внутренние оценки: \index{внутреннее оценивание}
\begin{equation*}
[1,2] / [1,2]= [0.5, 4], \quad [1,2] \oslash [1,2]= 1
\end{equation*}

Таким образом, в арифметике Каухера имеются гибкие арифметические возможности  оценок: от самых строгих, рассчитанных на наихудший вариант, до локализующих результат. Последнее обстоятельство особенно важно при многократных проведениях однотипных операций и построения итерационных алгоритмов. 

Приведём правила умножения в $\mathbb{KR}$ в виде так называемой таблицы Кэли.
\begin {table}[h]
\begin{small}
\begin{center}
	\begin{tabular}{l | c c c | c}
		& $\mbf{b} \in P $ & $\mbf{b} \in Z$ & $\mbf{b} \in -P$ & $\mbf{b} \in \mathrm{dual} \ Z $ \\	
		\hline	
		$\mbf{a} \in P$  & $ [\un{a}  \un{b}, \ov{a} \ov{b}] $ & $ [\ov{a}  \un{b}, \ov{a} \ov{b}] $ & $[ \ov{a} \un{b} ,\un{a}  \ov{b}] $ & $ [\un{a}  \un{b}, \un{a} \ov{b}] $\\
		$\mbf{a} \in Z$ & $ [\un{a}  \ov{b}, \ov{a} \ov{b}] $& $[min \{\un{a}  \ov{b}, \ov{a} \un{b} \}, max\{ \un{a}  \un{b}, \ov{a} \ov{b} \}] $ & $[ \ov{a} \un{b} , \un{a}  \un{b}] $ & 0 \\
		$\mbf{a} \in -P$ & $[\un{a}  \ov{b}, \ov{a} \un{b}] $& $[\ov{a}  \un{b}, \ov{a} \ov{b}] $ & $ [ \ov{a} \ov{b}, \un{a}  \un{b} ] $ & $ [ \ov{a} \ov{b}, \ov{a}  \un{b} ] $\\
			\hline	
		$\mbf{a} \in \mathrm{dual} \ Z $ & $ [\un{a}  \un{b}, \ov{a} \un{b}] $ & 0 & $[ \ov{a} \un{b} , \un{a}  \ov{b}] $ & $[max \{\un{a}  \un{b}, \ov{a} \ov{b} \}, min\{ \un{a}  \ov{b}, \ov{a} \un{b} \}] $
	\end{tabular} 
\end{center}
\end{small}
\caption{Интервальное умножение в полной интервальной арифметике}  
\label{t:CayleyKR}
\end{table}

Замечательно, что данные в таблице \ref{t:CayleyKR} правила верны и для классической интервальной арифметики. Её область ограничена квадратом $3 \times 3$ и не включает неправильных интервалов самой нижней строчки и самого правого столбца.


Помимо уже упомянутого, в полной интервальной арифметике всегда имеет содержательный смысл обобщение операции пересечения интервалов, взятие \emph{минимума по включению}, обозначаемому как $\wedge$. \index{минимум по включению}

Продемонстрируем этот факт на примере. Найдем пересечение двух пересекающихся интервалов [1, 3] и [2,4]:
\begin{equation*}
[1, 3] \cap [2,4] = \left\lbrace \max \min \{1,2\}, \min \max \{3,4 \}  \right\rbrace   = [2,3].
\end{equation*}
Поступим аналогичным образом с непересекающимися интервалами [1, 2] и [3,4], взяв минимум по включению:
\begin{equation*}
[1, 2] \wedge [3,4] = \left\lbrace \max \min \{1,3\}, \min \max \{2,4 \}  \right\rbrace   = [3,2].
\end{equation*}
В классической интервальной арифметике этот результат не имеет смысла, а полной имеет: это минимальный интервал, имеющий общие элементы с исходными.

Такая возможность даёт гибкость при неизбежной в экспериментальной практике работе с несовместными данными. \emph{Минимаксный подход}, свойственный полной интервальной арифметике, также обеспечивает и другие важные свойства, см. \cite{SharyBook}. \index{минимаксный подход}

%\begin{eqnarray*} \left[ \min_{x \in \mbf{x} } \max_{y \in \mbf{y} } f(x,y),  \max_{x \in \mbf{x} } \min_{y \in \mbf{y} } f(x,y) 	\right] \subseteq f(\mbf{x}, \text{dual} \ \mbf{y}), \\ 	f(\mbf{x}, \text{dual} \ \mbf{y})  \subseteq 	\left[ \max_{x \in \mbf{x} } \min_{y \in \mbf{y} } f(x,y),  \min_{x \in \mbf{x} } \max_{y \in \mbf{y} } f(x,y)	\right]. \end{eqnarray*}

\subsection{Исторические сведения}

В завершение неформального введения приведём некоторые исторические сведения.

Неолитические световые сооружения и культовые сооружения Древнего мира учитывают непостоянство положения Солнца в течение года, с тем, чтобы солнечный свет попадал на выбранные в качестве <<алтарей>> объекты \cite{OpticsHistory}.  

Античные учёные оставили нам примеры интервальных оценок. 
Аристарх Самосский, автор первой исторически известной гелиоцентрической системы, в сочинении «О величинах и расстояниях Солнца и Луны» \cite{Veselovcky1961} оценил, что отношение радиусов Солнца и Земли составляет больше чем 19 к 3, но меньше, чем 43 к 6. Эту оценку в современных обозначениях можно выразить так: \index{Аристарх Самосский}
\begin{equation*}
\frac{R_{\odot}}{R_{\text{Земли}}} = \left[  \frac{38}{6}, \frac{43}{6}\right].
\end{equation*}
Численно оценка Аристарха очень неточна, но метод её определения и представления правильны.

Трудно переоценить результат Архимеда об определении отношения длины окружности к периметру. Он использовал для оценки отношения длины окружности к диаметру периметры вписанных и описанных 96-угольников \cite{Archimedes}. Результат из его работы "$K \acute{u} \kappa \lambda o \upsilon \ \  \mu \acute{\varepsilon} \tau \rho \eta \sigma \iota \zeta$ (Измерение окружности)" : \index{Архимед}

\begin{equation*}
3\tfrac{10}{71} \leq 
\dfrac{\pi \varepsilon \rho \acute{\iota} \mu. \kappa \acute{u} \kappa \lambda o \upsilon}{\delta \iota \acute{\alpha} \mu \varepsilon \tau \rho o} 
\leq 3\tfrac{1}{7}.
\end{equation*}
%Κύκλου μέτρησις

В сочинении <<Псаммит>> (<<Исчисление песчинок>>) Архимед приводит двустороннюю оценку углового размера Солнца и обсуждает способ получения этой оценки.

Таким образом, мы имеем свидетельства применения интервальных величин, которым более 2 тысяч лет.

О предтечах интервального анализа в XIX веке и о зарождении и развитии его как самостоятельной математической дисциплины достаточно подробно написано в Главе 1 книги С.П.Шарого <<Конечномерный интервальный анализ>> \cite{SharyBook}.

\section{Мультиинтервалы.}\label{s:MultiIntervals}

\paragraph{Неодносвязные интервальные величины (мультиинтервалы).}

В ряде разделов науки и техники имеют место ситуации, когда исследуемая величина  содержится в неодносвязной области. %В  интервальном анализе такая ситуация описывается объектами, которые называются твинами (от англ. фразы TWice INterval, <<двойной интервал>>).

Согласно \cite{SharyBook}, мультиинтервал --- это объединение конечного числа несвязных интервалов числовой оси (Рис. \ref{f:MultiInterval} ). \index{мультиинтервал}

\begin{figure}[ht]
	\centering
	\includegraphics[width=0.7\textwidth]{Figures/MultiInterval.png}
	%	\includegraphics[width=0.8\textwidth]{7.mps}	
	\caption{Мультиинтервал в $\mathbb{R}$. Рис. 1.11 из \cite{SharyBook}.}
	\label{f:MultiInterval} 
\end{figure}

Между мультиинтервалами также могут быть определены арифметические операции <<по представителям>>, аналогично тому, как это делается на множестве интервалов \cite{Iakovlev1968}.
Мультиинтервальная арифметика применяется редко ввиду серъёзных ограничений, которые возникают при алгебраических операциях с мультиинтервальными величинами и вычислительных сложностей. Тем не менее, сама по себе идея мультиинтервалов содержательна и полностью отметать её не стоит. 

%	На практике концы интервалов, представляющие результаты измерений, сами могут быть известны неточно, так что возникает необходимость работы с интервалами, имеющими  интервальные концы. Такие объекты известны в интервальном анализе и называются твинами (от англ. фразы TWice INterval, <<двойной интервал>>).\index{твин} Они были введены в научный оборот в начале 80-х годов XX века в работах испанских исследователей, и заинтересованный читатель может найти подробности в книге \cite{ModalIABook}.       
% Твинные арифметики и их применение в методах и алгоритмах двустороннего интервального оценивания. Нестеров, Вячеслав Михайлович. дисс. д.ф.-м.н. 1999

В естественных науках возникновение неодносвязных областей часто связано с наличием периодичности в уравнениях или граничных условиях. Спектр таких явлений достаточно широк.

\section{Твины.}

На практике концы интервалов, представляющие результаты измерений, сами могут быть 
известны неточно, так что возникает необходимость работы с интервалами, имеющими  
интервальные концы. Такие объекты известны в интервальном анализе и называются 
\textit{твинами} (по английски twin, как сокращение фразы \un{tw}ice \un{in}terval, 
<<двойной интервал>>).\index{твин} Они были введены в научный оборот в начале 
80-х годов XX века в работах испанских исследователей, и заинтересованный читатель 
может найти подробности в книге \cite{ModalIABook}. Отдельные аспекты применения 
твинов рассматриваются в статье \cite{Twins1981}. Краткое изложении основ теории 
твинов дано в статье \cite{Nesterov1997}, а развёрнутое --- в диссертации 
\cite{Nesterov1999}. 

Tвин, как <<интервал интервалов>>  или интервал с интервальными концами, можно 
представить как 
\begin{equation} 
\label{Twin}
\mbf{X} = 
[\mbf{a}, \mbf{b}] = \bigl[\,[\un{\mbf{a}}, \ov{\mbf{a}}], [\un{\mbf{b}}, \ov{\mbf{b}}]\,\bigr].
\end{equation}

\begin{figure}[hbt]
	\centering\small 
	\setlength{\unitlength}{1mm}
	\begin{picture}(70,17)
	\put(0,0){\includegraphics[width=70mm]{Figures/twinfig.eps}}
	\put(-5,6.6){\vector(1,0){80}} \put(71.5,7.6){$\mbb{R}$} 
	\put(21,10){$\un{\mbf{a}}$} \put(30,10){$\ov{\mbf{a}}$} 
	\put(41,10){$\un{\mbf{b}}$} \put(50,10){$\ov{\mbf{b}}$} 
	\put(35,1){$\mbf{X}$}  
	\end{picture}
	\caption{Твины на вещественной оси.} 
	\label{TwinsPic2} 
\end{figure}
На рисунке \ref{TwinsPic2} твин $\mbf{X}$ представлен в графической форме. Концы твина, интервалы $\mbf{a}$ и $\mbf{b}$, даны более тёмной заливкой, чем остальная часть твина.


Твин является множеством всех интервалов, больших или равных $[\un{a}, \ov{a}]$ и меньших
или равных $[\un{b}, \ov{b}]$, и точное определение зависит от смысла, который вкладывается в понятия <<больше или равно>>, <<меньше или равно>>.
Поскольку интервалы могут быть упорядочены различными способами, то существуют 
различные виды твинов. Двум основным частичным порядкам на $\mathbb{IR}$ и $\mathbb{KR}$, 
<<$\subseteq$ >> и <<$\leq$>>,  соответствуют два основных типа твинов. Разработаны 
различные операции с твинами, а также способы оценок значений функций от них. 


\section{Анализ данных с интервальной неопределённостью}

В этой главе мы приведём ряд примеров, которые мотивируют применение интервальных подходов при анализе данных. Математически корректное обсуждение многих вопросов <<интервальной статистики>> проводится в книге \cite{BookIntStat}. \index{интервальная статистика}

\subsection{Измерения и их результаты} 
\label{MeasuResultSect} 

\subsection{Погрешность измерений} 

\subsection[Накрывающие и ненакрывающие измерения]% 
{Накрывающие и ненакрывающие \\* измерения} 
\label{CoverMeasrSect} 

\subsection{Принцип соответствия} 
\label{CorresPrincpSect}

\subsection{Выбросы и промахи} 
\label{OutlierSect}

\subsection{Выборки интервальных данных} 
\label{InteSampleSect} 

\subsection{Информационное множество} 
\label{InfoSetSect}

\subsection{Оценки точечные и интервальные} 

\chapter{Измерение постоянной величины} 
\label{MeasrConstChap}

\section{Мода интервальной выборки} 
\label{ModeSampleSect} 

\section[Выборки унимодальные и мультимодальные]%
{Выборки унимодальные \\* и мультимодальные} 
\label{UniMultiModSect} 

\section{Вариабельность оценки} 
\label{ConstVariabSect}  

\section{Приём варьирования неопределённости} 
\label{UncertAlterSect} 

%%%%%%%%%%%%%%%%%%%%%%%%%%%%%%%%%%%%%%%%%%%%%%%%%%%%%%%%%%%%%%%%%%%%%%%%%%%%%%%%%%%%

\chapter{Интервальные величины на Земле}\label{s:IntNature}

Интервальность явлений в природе проявляется начиная с самых основ строения веществ, образующих нашу планету. В этой главе мы рассмотрим Периодический закон расположения элементов, а далее рассмотрим ряд изотопных распределений на Земле.


\section{Сведения из ядерной физики}\label{s:NuclPhys} 

Вещества в природе состоят их химических элементов.
Под химическим элементом понимают совокупность атомов с одинаковым зарядом атомных ядер. 

Атомное ядро состоит из протонов, число которых равно атомному номеру элемента, и нейтронов, число которых может быть различным.  В связи с этим, для понимания закономерностей строения ядер атомов необходимы некоторые сведения из ядерной физики, см., например книгу \cite{NuclPhys}.

Химические свойства атомов определяются электрическим зарядом ядра, то есть числом протонов в ядре. При этом скорость протекания химических и физических процессов также зависит от массы ядра, то есть от числа нейтронов в ядре. Поэтому  различные изотопы атомов имеют различные свойства: с разной скоростью участвуют в химических ракциях, по-разному накапливаются в организмах, стратифицируются в зависимости от высоты земной поверхности и многое другое. Кроме того, не все изотопы ядер стабильны, и это тоже сказывается на их распределении в природе.

Обычно в ядрах нейтронов больше чем протонов. Этот факт иллюстрирует так называемая диаграмма $N-Z$ атомных ядер.

\begin{figure}[ht] 
	\centering\small
	\unitlength=1mm
	\includegraphics[width=100mm]{Figures/NZdiagram.png} 
	\caption{$N-Z$ диаграмма  атомных ядер \cite{NuclPhys}} 
	\label{f:NZdiagram}
\end{figure}
Черным цветом выделены стабильные ядра --- долина стабильности.
Справа от нее располагаются ядра, испытывающие $\beta^{-}$-распад, слева - ядра, испытывающие $\beta^{+}$-распад и $e$-захват. В области больших $A = N+Z$ находятся ядра, испытывающие $\alpha$-распад, и спонтанно делящиеся ядра. 
Линия $B_p = 0$ (proton drip-line) ограничивает
область существования атомных ядер слева, линия $B_n = 0$ (neutron drip-line) --- справа.

На Рис.~\ref{f:NZdiagram} большинство элементов лежит ниже прямой $N=Z$.
Также на Рис.~\ref{f:NZdiagram} выделены ядра с определёнными числами протонов и нейтронов. Поясним этот факт.

Cвойства ядер весьма сложны. Протоны и нейтроны участвуют во всех видах известных взаимодействий, сильных, слабых, электромагнитных и гравитационных. Поэтому нет такой теоретической модели ядер, которая бы успешно количественно объясняла все их наблюдаемые свойства.
%Весьма успешной для описания ряда явлений, таких как деление ядер, в частности, является представление об ядерном веществе как жидкости со специальными свойствами. 

Экспериментальные исследования атомных ядер выявили некоторую периодичность в изменении
индивидуальных характеристик (энергии связи, спины, магнитные моменты, четности, некоторые
особенности $\alpha$- и $\beta$-распадов) основных и возбужденных состояний атомных ядер. 

Обнаруженная периодичность подобна периодичности свойств электронных оболочек атома и определяется указанными выше \emph{магическими} числами нейтронов и протонов.

В частности, было обнаружено, что наибольшую энергию связи имеют ядра с так называемыми \emph{магическими} числами нейтронов и протонов, равными \index{магические числа нейтронов и протонов}
\begin{align*}
\begin{array}{cl}
N & 2, 8, 20, 28, 50, 82, 126, 184(?) \\
Z & 2, 8, 20, 28, 50, 82, 114(?)
\end{array}
\end{align*}

Результатом работы по систематизации и обобщения огромного
количества экспериментальных данных было создание в середине XX века
модели оболочек атомных ядер.  

\section{Атомные веса элементов}

Начнём представление интервалов в Природе с наиболее фундаментальной составляющей всего сущего --- с атомов.

Комиссия по изотопным распределениям и атомным весам (Commission on Isotopic Abundances and Atomic Weights, CIAAW) \cite{CIAAW}.
\index{Commission on Isotopic Abundances and Atomic Weights, CIAAW}

С 2009 года атомные веса некоторых элементов в периодической системе химических 
элементов Д.И.\,Менделеева стали выражаться интервалами \cite{IUPAC}. Это событие 
стало итогом длительного, продолжительностью более полувека, процесса осознания 
химиками неизбежной и неустранимой изменчивости величины атомных масс элементов 
в зависимости от того, где и как взята их проба. С середины XX века вместе 
с развитием измерительной техники и экспериментальных методик постепенно стало ясно, 
что различие результатов измерений атомных масс в различных пробах веществ носит 
принципиальный характер. 

Дело в том, что почти каждый химический элемент представлен в природе смесью своих \index{изотоп}
изотопов, т.\,е. разновидностями атомов, сходных по своим химическим свойствам 
(структуре электронных оболочек), но отличающиеся массой ядер. И относительная 
доля различных изотопов существенно меняется в зависимости от места и характера 
взятия пробы. Например, в тканях живых организмов преобладают более лёгкие изотопы 
химических элементов, нежели в неживой природе. Отличаются друг от друга 
относительные доли изотопов элементов на суше и в морях и т.\,п. 

В периодической таблице Менделеева, поддерживаемой Международным союзом теоретической \index{Периодическая таблица Менделеева}
и прикладной химии IUPAC приводятся интервальные границы стабильных изотопов химических \index{стабильные изотопы}
элементов. Например, для кислорода, имеющего 3 изотопа с атомными массами 16, 17 и 18 
на стр. 1858 статьи \cite{IUPAC} приводятся данные, часть которых представлена 
в Табл.~\ref{IUPACOxygen}. 
\index{The International Union of Pure and Applied Chemistry, IUPAC }
%%%%%%%%%%%%%%%%%%%%%%%%%%%%%%%%%%%%%%%%%%%%%%%%%%%%%%%%%%%%%%%%%%%%%%%%%%%%%%%%%%%%%%%%

\begin{table}[h!]
	\centering
	\caption{Стабильные изотопы кислорода.} 
	\medskip 
	\begin{tabular}{|c|c|}
		\hline
		Стабильный  изотоп & Молярная доля \\
		\hline 
		$^{16}O$ & [0.997 38, 0.997 76] \\
		$^{17}O$ & [0.000 367, 0.000 400] \\
		$^{18}O$ & [0.001 87, 0.002 22] \\			
		\hline
	\end{tabular}
	\label{IUPACOxygen}
\end{table} 

%%%%%%%%%%%%%%%%%%%%%%%%%%%%%%%%%%%%%%%%%%%%%%%%%%%%%%%%%%%%%%%%%%%%%%%%%%%%%%%%%%%%%%%%  

Компактное представление кислорода в таблице Менделеева выглядит следующим образом.

\begin{figure}[ht] 
	\centering\small
%	\unitlength=1mm
	\includegraphics[width=0.3\textwidth]{Figures/Oxygen.png}
%	\includegraphics[width=30mm]{Figures\Oxygen.png}
	\caption{Представление кислорода в таблице Менделеева.} 
	\label{f:Oxygen}
\end{figure}	
На рисунке \ref{f:Oxygen} дано наглядное представление о распространенности изотопов кислорода в природе в форме полярной диаграммы. В нижней строчке приведён интервал атомной массы.

Для каждого стабильного изотопа приведены границы, в пределах которых данный изотоп 
встречается в различных породах, атмосфере, водной среде в различных местах Земли. 
Подробные сведения приводятся на рисунках 4.8.1-4.8.3 из работы \cite{IUPAC}. 


Первоначально в 2009 году интервалы атомных весов были назначены для 10 химических 
элементов, но далее в 2013 и 2016 годах работа по <<интервализации>> продолжилась, 
и теперь в периодической таблице Д.И.\,Менделеева имеется 13 элементов, атомные веса 
которых выражаются интервалами. Среди них --- такие широко распространённые и важные 
элементы как водород, углерод, азот, кислород, кремний, сера и др. Интервалы дают 
двусторонние границы значений атомного веса для любой пробы ``нормального материала'' 
включающего эти элементы. При этом особо подчёркивается \cite{IUPAC}, что внутри 
заданных интервалов не предполагается наличия какого-либо вероятностного распределения. 

%Описание того, как получаются там эти интервалы, представлено на %Рис.~\ref{f:HistAtom}. 
%%%%%%%%%%%%%%%%%%%%%%%%%%%%%%%%%%%%%%%%%%%%%%%%%%%%%%%%%%%%%%%%%%%%%%%%%%%%%%%%%%%%%%%%
%\begin{figure}[ht] \label{f:HistAtom}
%	\centering\small
%	\unitlength=1mm
%	\begin{picture}(80,68)
%	\put(0,0){\includegraphics[width=80mm]{Figures\DistrPlot.eps} 
%	\put(60,24){\mbox{гистограмма частот}} 
%	\end{picture}
%	\caption{Как образуется интервал атомных весов элемента.} 
%\end{figure} 
%%%%%%%%%%%%%%%%%%%%%%%%%%%%%%%%%%%%%%%%%%%%%%%%%%%%%%%%%%%%%%%%%%%%%%%%%%%%%%%%%%%%%%%%
%На Рис.~\ref{f:HistAtom} по оси абсцисс отложены массы изотопов, по оси ординат %--- их распространённость в природе.

Например, в случае ртути, известны изотопы с массовыми числами от 170 до 216 (количество протонов 80, нейтронов от 90 до 136).
Природная ртуть состоит из смеси 7 стабильных изотопов:

\begin{table}[h!]
	\centering
	\caption{Стабильные изотопы ртути.} 
	\medskip 
	\begin{tabular}{cc}
		Изотоп & Распространённость \\
		\hline
		$^{196}$Hg &  0,155 \% \\
		$^{198}$Hg & 10,04 \% \\
		$^{199}$Hg  & 16,94 \% \\
		$^{200}$Hg  & 23,14 \% \\
		$^{201}$Hg  &  13,17 \% \\
		$^{202}$Hg  &  29,74 \% \\
		$^{204}$Hg  &  6,82 \% \\
		\hline
	\end{tabular} 
	\label{Sulfur}
\end{table}		

Приведённые в таблице \ref{Sulfur} величины распространённости служат исходными данными для построения гистограммы частот. %, схематично представленной на Рис.~\ref{f:HistAtom}.
Конкретно для атомов ртути этот рисунок показан на Рис.~\ref{f:HistHg}.

\begin{figure}[ht] 
	\centering\small
	\begin{tikzpicture}
	%\draw[help lines] (0,0) grid (11,5);
	\draw[->] (0,0) -- (0,4);
	\draw[->] (0,0) -- (10,0);
	\draw (-0.5, 1) node {10 \%};
	\draw (-0.5, 2) node {20 \%};
	\draw (-0.5, 3) node {30 \%};
	\draw[red] (0,0.0155) -- (1,0.0155);
	\draw[red] (1,0.0155) -- (1,0);
	\draw[red] (1,0) -- (2,0);
	\draw[red] (2,0) -- (2,1.004);
	\draw[red] (2,1.004) -- (3,1.004);
	\draw[red] (3,1.004) -- (3,1.694);
	\draw[red] (3,1.694) -- (4,1.694);
	\draw[red] (4,1.694) -- (4,2.314);
	\draw[red] (4,2.314) -- (5,2.314);
	\draw[red] (5,2.314) -- (5,2.314);
	\draw[red] (5,2.314) -- (5,1.317);
	\draw[red] (5,1.317) -- (6,1.317);
	\draw[red] (6,1.317) -- (6,2.974);
	\draw[red] (6,2.974) -- (7,2.974);
	\draw[red] (7,2.974) -- (7,0);
	\draw[red] (7,0) -- (8,0);
	\draw[red] (8,0) -- (8,0.682);
	\draw[red] (8,0.682) -- (9,0.682);
	\draw[red] (9,0.682) -- (9,0);
	\draw (0.6,-0.5) node {196};
	\draw (2.6,-0.5) node {198};
	\draw (4.6,-0.5) node {200};
	\draw (6.6,-0.5) node {202};
	\draw (8.6,-0.5) node {204};
	\draw (10,-0.5) node {Масса изотопа};
	\draw (2,4) node {Распространённость};
	\end{tikzpicture}
	\caption{Распространённость изотопов ртути на Земле.}
	\label{f:HistHg}
\end{figure} 

Относительно характера графика, представленного на Рис.~\ref{f:HistHg}, следует заметить следующее. Согласно современным представлениям, атомное ядро составляют протоны и нейтроны (нуклоны). Характер сил, действующих между ними, таков, что для лёгких ядер количества протонов и нейтронов примерно равны, с небольшим преобладанием последних. Число стабильных изотопов при этом невелико.  В ядрах тяжёлых элементов нейтронов существенно больше, чем протонов, и количество изотопов может достигать десятков, из которых стабильна небольшая часть. При этом количество стабильных  изотопов с чётным количеством нуклонов заметно превышает количество стабильных  изотопов с нечётным количеством нуклонов. Для энергетически выгодной конфигурации количества нуклонов существуют и другие закономерности, подобные принципу заполнению электронных оболочек атомов. В целом график распределения стабильных изотопов для данного химического элемента имеет неправильную форму с возможными <<пробелами>> внутри графика, что в случае ртути имеет место для изотопов с массами 197 и 203.

В публикации \cite{IUPAC} предложена расширенная версия периодической  системы химических элементов.
Авторы пишут:  \index{Периодическая таблица элементов и изотопов} \index{Periodic Table of the Elements and Isotopes --- IPTEI}
<<Периодическая таблица элементов и изотопов (Periodic Table of the Elements and Isotopes --- IPTEI) IUPAC (Международный союз теоретической и прикладной химии) была создана для ознакомления студентов, преподавателей и непрофессионалов с существованием и важностью изотопов химических элементов.>> Они также предлагают использовать её в качестве наглядного пособия, подобно таблице периодических элементов.

В целом таблица Менделеева выглядит как показано на рисунке ~\ref{f:PeriodicTable} \cite{IUPAC}.
Легенда цветового поля каждого элемента дана в таблице.\\
\begin{table}
{\small
\begin{tabular}{ll}
	Цвет фона & Пояснение \\
	\hline	
	красный & элемент имеет два или более стабильных изотопов. \\
	~ &	Соотношения изотопов различны в различных распространённых  \\
	~& материалах. Эти вариации надёжно определены, \\
	~&  атомный вес указывается в виде интервала, в квадратных скобках; \\
	\hline
	жёлтый & элемент имеет два или более стабильных изотопов. \\
	~ & Соотношения изотопов различны в различных распространённых  \\
	~ & материалах. При этом невозможно дать надежные оценки нижних  \\
	~& и верхних границ изменений. \\
	~ & Атомный вес даётся с неопределённостью, которая включает  \\
	~ & ошибку измерений и неопределённость вариации изотопных отношений; \\
	\hline
	голубой& элемент имеет один стабильный изотоп.  \\
	~ & Атомный вес даётся с неопределённостью, которая включает\\ 
	~ & ошибку измерений. \\
	\hline	
	белый & элемент не имеет стабильных изотопов.	\\
	~ & в распространённых материалах не содержится в таких  \\
	~ & количествах, по которым можно дать оценку изотопных отношений.\\
	\hline
\end{tabular} 
\caption{Обозначения на рис. \ref{f:PeriodicTable}.}
}
\end{table}

\begin{figure}[ht] 
	\centering\small
	\unitlength=1mm
	\includegraphics[width=110mm]{Figures/PeriodicTable2016all.png} 
	\caption{Таблица Менделеева элементов и изотопов \cite{IUPAC}.} 
	\label{f:PeriodicTable}
\end{figure}

%\end{example} 

\section{Изотопы на Земле}

Физические свойства Земли являются примером интервальной неопределённости данных. Для такого крупного объекта, как планета, трудно привести характеристику, которая имела бы точечный характер. С учётом непрерывной геологической эволюции Земли, все её параметры имеют интервальный характер. \index{физические свойства Земли }

\begin{figure}[ht] 
	\centering\small
	\unitlength=1mm
	\includegraphics[width=100mm]{Figures/SeismicRefraction.png} 
	\caption{Скорости звука в различных геологических структурах Земли \cite{Condie2015}.} 
	\label{f:SeismicRefraction}
\end{figure}

В учебнике <<Earth as an evolving planetary system>> \cite{Condie2015} приводятся значения скорости звука в различных геологических структурах Земли, с указанием диапазонов глубин от поверхности планеты для современной геологической эпохи. 

%На рисунке \ref{f:SeismicRefraction} приведены в легенде даны интервальные оценки скорости звука.
В легенде столбчатой диаграммы, приведенной на рисунке  \ref{f:SeismicRefraction}, даны интервальные оценки скорости звука.
Например, для континентальных и океанических плато скорость звука для $P$-волн составляет 
$ [6.3, 6.7 ]$ км/c.


\subsection{Изотопная подпись}\label{s:IsotopeSignature}

Представленный выше материал является основой для понимания ряда новых методик исследования неживой и живой природы. 
В последние несколько десятилетий в науку прочно вошёл новый термин <<изотопная подпись>>. \index{изотопная подпись}

Википедия определяет изотопную подпись следующим образом \cite{IsotopeSignatureWiki}. 
<<Изотопная подпись  (или изотопная сигнатура) --- специфическое соотношение нерадиоактивных <<стабильных изотопов>> или относительно стабильных радиоактивных изотопов или неустойчивых радиоактивных изотопов определённых химических элементов в исследуемом материале. Соотношения изотопов в образце исследуют при помощи изотопной масс-спектрометрии.>>

Вхождение нового термина в научное обращние стало одним из результатов длительной работы представителей различных специальностей: биологов, географов, геологов, палеонтологов. Список можно расширить.

Результатом стала не просто фиксация различных изотопных соотношений в зависимости от происхождения исследуемого материала, а использования этих соотношений как инструмента исследования.

Например, для кислорода в документе \cite{IUPAC2018} приводятся следующие данные.
На рисунке \ref{f:OxygenNature} приведены вариации атомного веса и изотопного состава ряда материалов, содержащих кислород приведены для атмосферного воздуха, воде, углекислом газе, карбонатах, оксиде азота, других химических соединениях, растениях и животных на Земле. 

\begin{figure}[ht] 
	\centering\small
	\unitlength=1mm
	\includegraphics[width=100mm]{Figures/OxygenNature.png} 
	\caption{Вариации атомного веса и изотопного состава ряда материалов, содержащих кислород \cite{IUPAC2018} на Земле.} 
	\label{f:OxygenNature}
\end{figure}


Подобные данные есть и для других биогенных материалов: водорода, углерода, азота, фосфора. 
По представленному материалу видно, что изотопные соотношения для биогенных материалов и различные комбинации  иотопных соотношений можно использовать для определения происхождения неизвестного вещества, определять среду в которой он формировался, определять, был ли это живой организм или минерал, и многое другое. \index{биогенные материалы}

В доступном для неспециалиста изложении многочисленные примеры использования изотопных подписей приводятся в книге палеонтолога А.Ю.~Журавлёва <<Сотворение Земли. Как живые организмы создали наш мир.>> \cite{Zhuravlev2019}.

Экологические материалы со ссылками на оргинальные статьи можно найти на сайте elementy.ru \cite{OpaevIsotope}.

Как правило, дефицит или избыток конкретного изотопа измеряют по отношению к общепринятому стандарту. Например,
\begin{equation*}
\delta^{13}C_{\text{sample}} = \left( \frac{^{13}C/^{12}C_{\text{sample}}}{^{13}C/^{12}C_{\text{standard}}} - 1\right) \cdot 1000 \ \permil
\end{equation*}
Используется обозначение $\delta^{\text{mass isotope}}\text{Element}$, единицей измерения служит промилле, $\permil$, одна тысячная доля.

%Также в систему современных научных представлений входит понятие <<изотопная ниша>>.

\subsection{Изотопная ниша}\label{s:IsotopeNiche}

В последние годы в практику вошёл новый термин, <<изотопная ниша>>, относящийся к применению изотопов в биологии. Он конкретизирует широко используемый термин экологическая ниша. \index{изотопная ниша}

Изотопная ниша — это пространство, занимаемое видом в многомерном пространстве признаков, которые в этом случае являются значениями индексов $\delta^{13}С, \delta^{15}N, \delta^{18}O \text{ и } \delta^{2}H$. 

В популярной статье \cite{OpaevIsotope} приводятся результаты исследований музейных экспонатов 254 особей 12 видов птиц, оригинальная публикация \cite{Rader2017}.  Группа водяных печников распространена в Южной Америке, где разные виды населяют диапазон высот от 0 до 5000 м над уровнем моря.
\begin{figure}[ht] 
	\centering\small
	\unitlength=1mm
	{\includegraphics[width=45mm]{Figures/BirdsArealLeft.png}} 
	\caption {Слева — ареалы 12 видов водяных печников (Cincloides). \cite{Rader2017}.} %Справа — представители рода Cincloides: сверху вниз — водяные печники островной C. antarcticus, белокрылый C. atacamensis и полосатокрылый C. fuscus. \cite{Rader2017}.} 
	\label{f:BirdsAreal}
\end{figure}

Фракционирование углерода происходит в природе разными путями. В частности, при фотосинтезе возможны 3 основных варианта: \index{тип фотосинтеза}
\begin{table}[h!]
	\centering
	\caption{Отношение $\delta^{13}C$ для разных механизмов фотосинтеза.} 
	\medskip 
	\begin{tabular}{|c|c|c|}
		\hline
		Тип фотосинтеза & $\delta^{13}C$ & Пример \\
		\hline 
		$C_4$ & [-16, -10] & зерно \\
		$CAM$ & [-20, -10] & фрукты\\
		$C_3$ & [-33, 24] & бобовые \\			
		\hline
	\end{tabular}
	\label{13Cplants}
\end{table} 

Было определено соотношение тяжелых и легких изотопов углерода, азота, кислорода и водорода в  перьях птиц. Птицы меняют перья во время линьки, обычно приуроченной к определенному периоду года и длящейся 1–2 месяца. Поэтому изотопный состав перьев может рассказать о том, чем птица в это время питалась. 
\begin{figure}[ht] 
	\centering\small
	\unitlength=1mm
	{\includegraphics[width=120mm]{Figures/IsotopesNiches.png}} 
	\caption{Изотопные ниши различных видов птиц по углероду и азоту (слева) и по кислороду и водороду (справа) \cite{Rader2017}.} 
	\label{f:IsotopesNiches}
\end{figure}


Изотопная ниша по углероду и азоту в какой-то степени является нишей трофической, так как характеризует питание. А ниша по кислороду и водороду — пространственная, так как зависит от местообитания (ведь изотопный состав воды — поставщика этих элементов — различается в разных местах). Чем шире изотопная ниша по углероду и азоту (то есть больше площадь соответствующего эллипса), тем больший спектр кормов потребляет данное животное, тем шире его трофическая ниша. \index{трофическая ниша}

\begin{figure}[ht] 
	\centering\small
	\unitlength=1mm
	{\includegraphics[width=45mm]{Figures/IsotopesNichesWid.png}} 
	\caption{Взаимосвязь ширины изотопной ниши по углероду и азоту (C/N) и по кислороду и водороду (O/H) для 12 видов водяных печников. Ширина ниши данного вида — это площадь соответствующего эллипса на рис. \ref{f:IsotopesNiches} \cite{Rader2017}.} 
	\label{f:IsotopesNichesWid}
\end{figure}

Аналогично, чем больше площадь эллипса по кислороду и водороду, тем в более широком спектре местообитаний можно найти особей этого вида. Оказалось, что ширина ниши (то есть площадь эллипса) по углероду и азоту, с одной стороны, и по кислороду и водороду, с другой, положительно связаны между собой.


\subsection{ Изотопные ландшафты }\label{s:IsoScapes}

Изотопные ландшафты (Isoscapes) --- это географические карты, в легенду которых входит содержание тех или иных изотопов. На карту могут быть нанесены результаты измерений или моделирования. \index{изотопные ландшафты}

Приведём примеры из публикации \cite{Bowen2010}. Для изотопа  $\delta^{15}N$ в растениях данные представлены на рисунке \ref{f:Plant15N} и для изотопа  $\delta^{18}O$ в морской воде --- на рисунке \ref{f:SeaWater18O}.
\begin{figure}[ht] 
	\centering\small
	\unitlength=1mm
	{\includegraphics[width=80mm]{Figures/Plant15N.png}} 
	\caption{Изотопное отношение для $\delta^{15}N$ в растениях \cite{Bowen2010}.} 
	\label{f:Plant15N}
\end{figure}


\begin{figure}[ht] 
	\centering\small
	\unitlength=1mm
	{\includegraphics[width=80mm]{Figures/SeaWater18O.png}} 
	\caption{Изотопное отношение для  $\delta^{18}O$ в морской воде \cite{Bowen2010}.} 
	\label{f:SeaWater18O}
\end{figure}

%\section*{Заключение}
\paragraph{Изотопные интервалы в природе.}

Развитие изотопного анализа в последние несколько десятилетий обогатило исследователей различными возможностями. Многие науки и отрасли деятельности человека существенно изменились. Появилась возможность получать принципиально новые виды информации, строить новые логические связи между явлениями. Появились новые понятия  и концепции. С получением новых данных идёт обогащение идеями, ставятся новые вопросы.

При этом математика помогает описывать данные с интервальной неопределённостью  и работать с интревальнозначными величинами. Принципиальныо новым шагом стало введение IUPAC интервальных границ стабильных изотопов химических элементов в периодической системе.  Увеличение числа исследований в естественных науках неизбежно потребует и развития математических методов для  эффективной работы с данными. 


\chapter*{Заключение}\label{Conclusion}


Выражаю благодарность участникам Всероссийского интервального веб-семинара, С.И.\,Жилину, С.И.\,Кумкову, А.В.\,Пролубникову, Е.В.\,Чаусовой  и С.П.\,Шарому, за проявленный интерес к работе и конструктивные обсуждения примеров. 





	
\begin{thebibliography}{00}


	\bibitem{BookIntStat} А.Н.\,Баженов, С.И.\,Жилин, С.И.\,Кумков, С.П.\,Шарый. <<Обработка и анализ данных с интервальной неопределённостью>>. 2021.

	\bibitem{Bazhenov2020}
	А.Н. Баженов. Интервальный анализ. Основы теории и учебные примеры: учебное пособие. --- СПб. 2020
	\url{https://elib.spbstu.ru/dl/2/s20-76.pdf/info}
	
		\bibitem{Bazhenov2022}
	А.Н. Баженов. Обобщение мер совместности для анализа данных с интервальной неопределённостью. – СПб., 2022. – 80 с.
	\url{https://elib.spbstu.ru/dl/5/tr/2022/tr22-142.pdf/info}

	\bibitem{Bohr1928} 
	%Bohr, N. (1928). "The Quantum Postulate and the Recent Development of Atomic Theory". Nature. 121 (3050): 580–590. doi:10.1038/121580a0
	\textsc{Бор Н.} Квантовый постулат и новое развитие атомистики. УФН 8 306–337 (1928)
	%\textsc{Ельяшевич М.А.} Развитие Нильсом Бором квантовой теории атома и принципа 
	%соответствия // Успехи Физических Наук. -- 1985. -- Т.~147, вып.~2. -- С.~253--301. \  
	%DOI:  10.3367/UFNr.0147.198510c.0253 
	
	\bibitem{Veselovcky1961} \textsc{Веселовский И. Н.}
Аристарх Самосский — Коперник античного мира] // Историко-астрономические исследования, вып. VII. — М., 1961. — С. 17—70. \\
http://www.astro-cabinet.ru/library/Aristarch/Aristarch\_3.htm 
	
	
	
	\bibitem{Kantorovich}
	\textsc{Канторович Л.В.} О некоторых новых подходах к вычислительным методам и обработке 
	наблюдений // Сибирский Математический Журнал. -- 1962. -- Т. 3, №5. -- С.~701--709. 
	
	
	\bibitem{OpaevIsotope} 
	\textsc{Опаев А.} Изотопная подпись. \\	
	https://elementy.ru/problems/1523/Izotopnaya\_podpis
	
	
	\bibitem{OskorbinMaksimovZhilin}
	\textsc{Оскорбин Н.М., Максимов А.В., Жилин С.И.} 
	Построение и анализ эмпирических зависимостей методом центра неопределенности // 
	Известия Алтайского государственного университета. -- 1998. -- №1. -- С.~35--38. 
	
	\bibitem{IntervalAnalysisExamples} 
	Примеры анализа интервальных данных в Octave \\ 
	\url{https://github.com/szhilin/octave-interval-examples}
	

	\bibitem{SharyBook} 
	\textsc{Шарый С.П.} Конечномерный интервальный анализ. -- ФИЦ ИВТ: 
	Новосибирск, 2022. \     Электронная книга, доступная 
	на \url{http://www.nsc.ru/interval/Library/InteBooks/SharyBook.pdf} 

	\bibitem{Iakovlev1968}
	\textsc{Яковлев А.Г.} Машинная арифметика мультиинтервалов // Вопросы кибернетики
	(Научный Совет по компл. проблеме <<Кибернетика>>: АН СССР). – 1986. – Вып. 125. –
	С. 66–81.

	\bibitem{Iakovlev2013}  \textsc{И.Яковлев.} Изучение трофической структуры сообществ с помощью анализа стабильных изотопов. Дискуссионные лекции-семинары по эволюционной экологии, 08.11.2013 \\
	http://www.eco.nsc.ru/lectures/Iakovlev\_Stable\_Isotopes.pdf
	
\bibitem{Archimedes} \textsc{Archimedes}
(Dover Books on Mathematics). 
Archimedes, Sir Thomas Heath. Unabridged reprint of the classic 1897 edition, with supplement of 1912. - p.512.

	
	\bibitem{Bowen2010} 
	\textsc{G.J. Bowen} Isoscapes: Spatial Pattern in Isotopic Biogeochemistry.  Annu. Rev. Earth Planet. Sci. 2010. 38:161–187 \\
	http://www.iai.int/admin/site/sites/default/files/uploads/ 2010\_Bowen\_Isoscapes\_Spatial-Pattern-in-Isotopic-Biogeochemistrypdf.pdf

	\bibitem{Newsome2007} \textsc{S.D. Newsome, C.Martinez del Rio, S. Bearhop, and D.L. Phillips.}
	A niche for isotopic ecology. Front Ecol Environ 2007; 5(8): 429–436, doi:10.1890/060150.01

	\bibitem{Condie2015} 
	Condie, K. C. (2015). Earth as an evolving planetary system. (3rd ed.) Elsevier. \\ https://doi.org/10.1016/C2015-0-00179-4
	
	\bibitem{InteNotation} 
	\textsc{Kearfott, R.B., Nakao, M., Neumaier, A., Rump, S., Shary, S.P., van Hentenryck, 
		P.} Standardized notation in interval analysis // Вычислительные Технологии. -- 
	2010. -- Т.~15, №1. -- С.~7--13. 
	
	\bibitem{IUPAC} 
	\textsc{Meija, J., Coplen, T.B., Berglund, M., Brand, W.A., De Bièvre, P., 
		Gröning, M., Holden, N.E., Irrgeher, J., Loss, R.D., Walczyk, T., Prohaska, T.} 
	Atomic weights of the elements 2013 (IUPAC Technical Report) // Pure and Applied 
	Chemistry. -- 2016. -- Vol.~88, Issue~3. -- P.~265--291. \   DOI: 10.1515/pac-2015-0305 
	

	\bibitem{NguyenKreinWuXiang} 
	\textsc{Nguyen H.T., Kreinovich V., Wu B., Xiang G.} Computing Statistics 
	under Interval and Fuzzy Uncertainty. Applications to Computer Science and Engineering. 
	-- Springer, Berlin-Heidelberg, 2012. 
	
	\bibitem{ChemInternatl} 
	Standard atomic weights of 14 chemical elements revised // Chemistry International. 
	-- 2018. -- Vol.~40, Issue 4. -- P.~23--24. \  DOI: 10.1515/ci-2018-0409 


	\bibitem{Rader2017}
	\textsc{J.A. Rader, J. A., Newsome, S. D., Sabat, P., Chesser, R. T., Dillon, M. E., and Martínez del Rio, C.} (2017). Isotopic niches support the resource breadth hypothesis. J. Anim. Ecol. 86, 405–413. doi:10.1111/1365-2656.12629
	
	\bibitem{GraviConstItaly}  
	\textsc{Rosi G, Sorrentino F., Cacciapuoti L., Prevedelli M., Tino G.M.} 
	Precision measurement of the Newtonian gravitational constant using cold atoms //  
	Nature. 2014. Vol.~510, P.~518--521. \  DOI: 10.1038/nature13433  \\   
	Предварительная версия работы депонирована в репозитории arXiv.org, статья 
	\url{https://arxiv.org/pdf/1412.7954.pdf}
	
%	\bibitem{Wasserstein2019} 	\textsc{Wasserstein, R.L.,  Schirm, A.L., Lazar, N.A.} Moving to a world beyond 	<<p\,<\, 0.05>> // The American Statistician. -- 2019. -- Vol.~73. -- P.~1--19. \  	DOI: 10.1080/00031305.2019.1583913 
	
	\bibitem{Zhilin2005}  
	\textsc{Zhilin, S.I.} On fitting empirical data under interval error // 
	Reliable Computing. -- 2005. -- Vol.~11. -- P.~433--442. \ DOI: 10.1007/s11155-005-0050-3 
	
	%%%%%%%%%%%%%%%%%%%%%%%%%%%%%%%%%%%%%%%%%%%%%%%%%%%%%%%%%%%%%%%%%%%%%%%%%%%%%%%%%
	%%  Поставить в тексте ссылку на источник и найти ему место в алфавитном порядке 
	%%%%%%%%%%%%%%%%%%%%%%%%%%%%%%%%%%%%%%%%%%%%%%%%%%%%%%%%%%%%%%%%%%%%%%%%%%%%%%%%%
%	\bibitem{Eurachem2007}  	\textsc{Ellison S.L.R.,\ Williams A. (Eds.)} EURACHEM/CITAC Guide: Use of uncertainty information in compliance assessment. First edition, 2007. \\	\url{https://www.eurachem.org/images/stories/Guides/pdf/Interpretation_with_expanded_uncertainty_2007_v1.pdf}

\bibitem{CIAAW}
Commission on Isotopic Abundances and Atomic Weights, CIAAW.
https://www.ciaaw.org/

\bibitem{IUPAC2018} Norman E. Holden, Tyler B. Coplen, John K. B\"{o}hlke, Lauren V. Tarbox, Jacqueline Benefield, John R. de Laetera, Peter G. Mahaffy, Glenda O’Connorb, Etienne Rotha,
Dorothy H. Tepper, Thomas Walczyk, Michael E. Wieser and Shigekazu Yoneda. \
IUPAC Periodic Table of the Elements and Isotopes (IPTEI) for the Education Community
(IUPAC Technical Report) \ Pure Appl. Chem. 2018; 90(12): 1833–2092
https://doi.org/10.1515/pac-2015-0703
Received August 3, 2015; accepted July 23, 2018

\bibitem{NuclPhys}
В.В.Варламов, Б.С.Ишханов, С.Ю.Комаров Атомные ядра. Учебное пособие. ISBN 978-5-91304-122-72010. –М., Университетская книга, 2010.
http://nuclphys.sinp.msu.ru/anuc/index.html

\bibitem{IsotopeSignatureWiki}
https://ru.wikipedia.org/wiki/Изотопная\_подпись

\bibitem{Zhuravlev2019} Журавлев А.Ю. Сотворение Земли. Как живые организмы создали наш мир.
М.: Альпина Паблишер. ISBN 978-5-91671-902-4. 514 стр.





\bibitem{Lebedev2003}
\textsc{А.Т.Лебедев.} Масс-спектрометрия в органической химии. Москва: БИНОМ. Лаборатория знаний, 2003. - 493 с., ил. - (Методы в химии).


%Твинные арифметики и их применение в методах и алгоритмах двустороннего интервального оценивания. Нестеров, Вячеслав Михайлович. дисс. д.ф.-м.н. 1999


		\bibitem{ModalIABook} 
\textsc{Sainz M.A., Armengol J., Calm R., Herrero P., Jorba L.J., Vehi J.}
Modal Interval Analysis: New Tools for Numerical Information. -- Cham, Switzerland: 
Springer, 2014. -- (\textsl{Lecture Notes in Mathematics; vol.~2091}). 


\bibitem{Twins1981}
\textsc{E. Gardenes, A.Trepat, J.M. Janer}
Approaches to simulation and to the linear problem in the SIGLA system. //
Gardenes, E., Trepat, A., and Janer, J. M.: Approaches to Simulation and to the Linear Problem in the SIGLA System, Freiburger Interval-Berichte 81 (8) (1981), pp. 1-28.
%FIB81-8_01-28.pdf


\bibitem{Nesterov1997} 
\textsc{Nesterov, V.M.} Interval and Twin Arithmetics. Reliable Computing 3, 369–380 (1997). https://doi.org/10.1023/A:1009945403631

\bibitem{Nesterov1999}
\textsc{В.М.~Нестеров} Твинные арифметики и их применение в методах и алгоритмах двустороннего интервального оценивания. дисс. д.ф.-м.н. г.Санкт-Петербург, Санкт-Петербургский институт информатики и автоматизации Российской академии наук, 1999, с. 234.

%Твинные арифметики и их применение в методах и алгоритмах двустороннего интервального оценивания. Нестеров, Вячеслав Михайлович. дисс. д.ф.-м.н. 1999


	
\end{thebibliography} 



%\todo[inline]{ДЗ Баженову, Жилину: Ссылки и приложения про данные о  массе нейтрино.} 



%%%%%%%%%%%%%%%%%%%%%%%%%%%%%%%%%%%%%%%%%%%%%%%%%%%%%%%%%%%%%%%%%%%%%%%%%%%%%%%%%%%%%%%%

\addcontentsline{toc}{chapter}{Предметный указатель}
\raggedright\small\printindex   

%%%%%%%%%%%%%%%%%%%%%%%%%%%%%%%%%%%%%%%%%%%%%%%%%%%%%%%%%%%%%%%%%%%%%%%%%%%%%%%%%%%%%%%% 

\end{document} 


%%%%%%%%%%%%%%%%%%%%%%%%%%%%%%%%%%%%%%%%%%%%%%%%%%%%%%%%%%%%%%%%%%%%%%%%%%%%%%%%%%%%%%













%%%%%%%%%%%%%%%%%%%%%%%%%%%%%%%%%%%%%%%%%%%%%%%%%%%%%%%%%%%%%%%%%%%%%%%%%%%%%%%%%%%%%%%%
\chapter*{Обозначения}\label{Notation}

\begin{tabular}{ll}
	:= & левая часть равенства есть обозначение для правой \\
	\& & логическая конъюнкция, связка <<и>> \\
	$\Longrightarrow$ & логическая импликация \\
	$\Longleftrightarrow$ &логическая равносильность \\
	$\rightarrow$ & отображение множеств \\
	$\mapsto$ & правило сопоставления элементов при отображении \\
	$\leftarrow$ & оператор присваивания в алгоритмах \\
	$\circ$ & знак композиции отображений \\
	$\emptyset$ & пустое множество \\
	$x \in X $ & элемент $x$ принадлежит множеству $X$  \\
	$x \notin X$ & элемент $x$ не принадлежит множеству $X$ \\ 
	$X \ni x$ & множество $X$ содержит элемент $x$\\
	$X \not \owns x$ & множество $X$ не содержит элемент $x$\\
	$X \cup Y$ & объединение множеств $X$ и $Y$ \\
	$X \cap Y$ & пересечение множеств $X$ и $Y$\\
	$ X \ Y$ & разность множеств $X$ и $Y$\\
	$X \subseteq $ & множество $X$ есть подмножество множества $Y$ \\
	$X \subset $ & множество $X$ есть собственное подмножество множества $Y$ \\
	$X \times Y$ & прямое декартово произведение множеств $X$ и $Y$\\
	$\mathbb{N}$ & множество натуральных чисел\\
	$\mathbb{R}$ & множество действительных (вещественных) чисел\\
	%	$\mathbb{R}_{+}$ & множество неотрицательных вещественных чисел \\
	$\mathbb{IR}$ & классическая интервальная арифметика \\
	$\mathbb{ID}$  & множество интервалов, содержащихся в $ D \subseteq \mathbb{R}^n $\\
	$\mathbb{KR}$ & полная интервальная арифметика Каухера \\
	$\mathbb{R}^n $ & множество вещественных $n$-мерных векторов \\
	$\mathbb{IR}^n $ & множество вещественных $n$-мерных векторов c элементами из $\mathbb{IR}$\\
	$\mathbb{KR}^n $ & множество вещественных $n$-мерных векторов c элементами из $\mathbb{KR}$\\
	%	$\mathbb{R}^{m \times n}$ & множество вещественных $m \times n$-матриц \\
	%	$\mathbb{IR}^{m \times n}$ & множество  $m \times n$-матриц с элементами из $\mathbb{IR}$ \\
	%	$\mathbb{KR}^{m \times n}$ & множество  $m \times n$-матриц с элементами из $\mathbb{KR}$ \\
\end{tabular}




\begin{tabular}{ll}
	$\mathcal{EF}$ & семейство элементарных функций \\	
	$\sgn \ x$ & знак вещественного числа $x$ \\
	$x^{+}, x^{-}$& положительная и отрицательная части числа $x$ \\
	$\sgn \mbf{a}$ & знак интервала $\mbf{a}$ \\
	$\mbf{a}^{+}, \mbf{a}^{-}$ & положительная и отрицательная части интервала $\mbf{a}$ \\
	$\un{\mbf{a}}, \inf \mbf{a}$ & левый конец интервала $\mbf{a}$ \\
	$\ov{\mbf{a}}, \sup \mbf{a}$ & правый конец интервала $\mbf{a}$ \\
	$ \| \mbf{a} \| $ & абсолютная величина (магнитуда) интервала $\mbf{a}$ \\	
	$\abs \mbf{a} $ & интервальное расширение функции модуля \\
	$\langle \mbf{a} \rangle $ & мигнитуда интервала $\mbf{a}$ \\
	$\langle \mbf{A} \rangle $ & компарант интервальной матрицы $\mbf{A}$ \\ 
	$\m \mbf{a}$ & середина (медиана) интервала $\mbf{a}$ \\
	${\tt wid}\ \mbf{a}$ & ширина интервала $\mbf{a}$ \\
	$\r \mbf{a}$ & радиус интервала $\mbf{a}$ \\
	${\tt dev} \ \mbf{a}$ & отклонение интервала $\mbf{a}$  от нуля \\
	${\tt dual} \ \mbf{a}$ & дуальный (двойственный) к  $\mbf{a}$  интервал \\
	${\tt opp} \ \mbf{a}$ & алгебраически противоположный к $\mbf{a}$ интервал \\
	${\tt inv} \ \mbf{a}$ & алгебраически обратный к $\mbf{a}$ интервал \\
	${\tt pro} \ \mbf{a}$ & правильная проекция интервала $\mbf{a}$\\
	${\tt vert} \ \mbf{a}$ & множество крайних точек интервала $\mbf{a}$\\
	$\ominus$ & <<внутреннее>> интервальное вычитание \\
	$\oslash$ & <<внутреннее>> интервальное деление \\
	$\chi (\mbf{a})$ &функционал Рачека от интервала $\mbf{a}$ \\
	$\varXi_{uni}$ & объединённое множество решений \\
	$\varXi_{tol}$ & допусковое множество решений \\
	$\varXi_{ctl}$ & управляемое множество решений \\
	$\varXi_{\alpha \beta} $ & множество $AE$-решений типа $\alpha \beta$ \\
	$\varXi_{\mathcal{A} \beta} $ & множество $AE$-решений типа $\mathcal{A}  \beta$ \\
	%	$\mbf{A}^{-1}$ & обратная интервальная матрица\\
	%	$\mbf{A}^{c}$ & характеристическая матрица ИСЛАУ \\
	%	$\mbf{b}^{c}$ & характеристический вектор правой части ИСЛАУ \\
	${\tt dist}$ & метрика в интервальных пространствах \\
	${\tt Dist}$ & мультиметрика в интервальных пространствах \\
	${\tt sti}$ &  стандартное погружение  $\mathbb{KR}^n$ в $\mathbb{R}^{2n}$ \\
	${\tt ran}(f,\mbf{X}) $ & область значений функции $f$ на множестве $\mbf{X}$ \\
	$f^{\angle}(\tilde{x}, x)$ & наклон функции $f$ между точками $\tilde{x}$ и $x$ \\
	%	${\tt hyp}$ & подграфик функции $f$ \\
	%	${\tt epi}$ & надграфик функции $f$ \\
	%	$\partial f$ & субдифференциал функции f \\
	%	$\partial X$ &граница множества $X$ \\
	%	${\tt int} \ X$   & топологическая внутренность множества $X$ \\
	%	${\tt cl} \ X$   & топологическое замыкание множества $X$ \\
	%	${\tt ch} \ X$   & выпуклая оболочка множества $X$ \\
	$\square X$   & интервальная оболочка множества $X$ \\
	$\wedge$   & минимум в частично упорядоченном множестве \\
	$\vee$   & максимум в частично упорядоченном множестве \\
	И & условная решёточная операция \\
	$I$ & единичная матрица соответствующих размеров \\
	$Q^{\sim} $ & знаково-блочная матрица для матрицы $Q$ \\
\end{tabular}

\begin{tabular}{ll}
	$\| \cdot \|$ & векторная или матричная норма \\
	$\| \cdot \|_1$ & 1-норма векторов или подчинённая 1-норма матриц \\
	$\| \cdot \|_1$ & 2-норма векторов или подчинённая 2-норма матриц \\
	$\| \cdot \|_1$ & $\infty$-норма векторов или подчинённая $\infty$-норма матриц \\
	$\lambda(A)$ & собственное значение матрицы $A$ \\
	$\rho(A)$ & спектральный радиус матрицы $A$ \\
	$\sigma(A)$ &  сингулярное число матрицы $A$ \\
	${\tt diag} (A)$ &диагональная  матрица \\%$n \times n$-матрица	с элементами $z_1, \ldots , z_n$ по главной диагонали \\
	$\mathcal{N}(\mbf{X}, \tilde{x})$ & интервальный оператор Ньютона \\
	$\mathcal{K}(\mbf{X}, \tilde{x})$ & интервальный оператор Кравчика \\
	$\mathcal{H}(\mbf{X}, \tilde{x})$ & интервальный оператор Хансена-Сенгупты
\end{tabular}

\bigskip

Используемая система обозначений следует, в основном, неформальному международному
стандарту на обозначения в интервальном анализе, выработанному в 2002 году. В настоящее время его текст доступен в Интернете на многих сайтах, посвящённых интервальным вычислениям (к примеру, на отечественном \url{http://www.nsc.ru/interval}).

Интервалы и другие интервальные величины (векторы, матрицы и др.) всюду в тексте обозначаются жирным математическим шрифтом, например, $\mbf{A}, \mbf{B}, \mbf{C}, \ldots, \mbf{x}, \mbf{y}, \mbf{z}$, тогда как неинтервальные (точечные) величины никак специально не выделяются. Арифметические операции с интервальными величинами --- это операции
соответствующих интервальных арифметик: либо классической интервальной арифметики $\mathbb{IR}$, либо полной интервальной арифметики Каухера $\mathbb{KR}$.


